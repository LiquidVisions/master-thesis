\section{Research Problem: The plethora of proposed design \\ principles}
\label{sec:research_problem}

Design principles, patterns and theorems are, on top of all additional measures to enhance
the modularity, stability and evolvability of software artifacts. This thesis focuses on
two prevalent approaches in order to research the hypothesized convergence that are
introduced in \ref{sec:into_ca} and \ref{sec:inro_ns}. 

There is a wide variety of proposed design principles available for the challenges that
occur in modular and evolvable software architecture. There are plentiful experiences
documented throughout professional and personal blog posts on the internet. Many of those
experiences have mixed outcomes, some are opinionated and results are sometimes based on
improper interpretations of the proposed solutions.

Deciding on the best fit for one of the solutions is a recurring and often difficult task
for software architects. A popular and widely accepted solution from software engineering
literature is Clean Architecture. There is a wide supporting community and many corporate
solutions move toward architectures that have similarities with the Clean Architecture
approach. 

An architecture that derives from science and empirical evidence is Normalized Systems
\parencite{mannaert_normalized_2009,mannaert_normalized_2016}. Deciding between the two
approaches can be a challenging task on which there is very little documentation and
research. Could it even be the case that combining the two approaches can be an approach
that leads to a highly modular and evolvable software artifact? Let's start with a small
introduction to both approaches.