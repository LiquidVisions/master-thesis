\subsubsection*{Separation of State}

\gls{sos} is a theorem that is based on the idea that processing functions should not
contain any state information but instead should rely on external data structures to store
state information. By separating state information from processing functions, Normalized
Systems can achieve a higher level of flexibility and adaptability. External data
structures can be updated or replaced without affecting the processing functions
themselves, which greatly reduces the change of unwanted ripple effects. This theorem is
described as followed: \parencite[258]{mannaert_normalized_2016}.

\begin{tcolorbox}[boxrule=0.1pt, colback=mygray, title=Theorem IV,colbacktitle=gray]
        \textit{Calling a processing function within another processing function, needs to exhibit state keeping in order to achieve stability.}
\end{tcolorbox}