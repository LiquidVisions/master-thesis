\section{Software Transformation Requirements} \label{sec_requirements_transformation}

We study stability and evolvability by investigating potential combinatorial effects in
\gls{ca} artifacts. Therefore, during the implementation, we will apply parts of the
Functional-Construction software Transformation from
\textcite[251]{mannaert_normalized_2016} by using the following five proposed Functional
Requirements Specifications. \textcite[254-261]{mannaert_normalized_2016} have defined
them as follows.

\begin{enumerate}[leftmargin=*]
    \item An information system needs to be able to represent instances of
    data entities. A data entity consists of several data fields. Such a field may be a basic
    data field representing a value of a reference to another data entity.
    
    \item An information system needs to be able to execute processing actions on
    instances of data entities. A processing action consists of several consecutive processing
    tasks. Such a task may be a basic task, i.e., a unit of processing that can change
    independently or an invocation of another processing action.
    
    \item An information system needs to be able to input or output values
    of instances of data entities through connectors.
    
    \item An existing information system representing a set of data entities needs to be
    able to represent a new version of a data entity that corresponds to including an
    additional data field and an additional data entity.
    
    \item An existing information system providing a set of processing actions needs to
    be able to provide a new version of a processing task, whose use may be mandatory, a
    new version of a processing action, whose use may be mandatory, an additional
    processing task, and an additional processing action
    
\end{enumerate}