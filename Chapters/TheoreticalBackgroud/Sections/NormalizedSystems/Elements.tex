\subsection{Normalized Elements} \label{subsec_ns_elements} 

In the context of the \gls{ns} Theory approach, the goal is to design evolvable software,
independent of the underlying technology. Nevertheless, when implementing the software and
its components, a particular technology must be chosen. For Object Oriented Programming
Languages like Java, the following Normalized Elements have been proposed
\parencite{mannaert_normalized_2016}[363-398].

This research's artifacts utilized C\# as the primary programming language. It is essential
to recognize that different programming languages may necessitate alternative constructs
\parencite{mannaert_normalized_2016}[364]. Given the strong similarities between C\# .NET
and Java, it is assumed that the same Normalized Elements are applicable for C\# .NET
implementation of this research's artifacts.

\subsubsection{The Data Element}
This is an object that represents a piece of data in the system. Data elements are used to
pass information between processing functions and other objects. In \gls{ns},
data elements are typically standardized to ensure consistency across the system.

\subsubsection{The Task Element}
This is an object that represents a specific task or action in the system. Tasks can be
composed of one or more processing functions and can be used to represent complex
operations within the system.

\subsubsection{The Connector Element}
This object is used to connect different parts of the system together. Connectors can be
used to link processing functions, data elements, and other objects, allowing them to work
together seamlessly.

\subsubsection{The Flow Element}
This object represents the flow of control through the system. It determines the order in
which processing functions are executed and can be used to handle error conditions or
other exceptional cases.

\subsubsection{The Trigger Element}
a trigger element is an object that reacts to specific events or changes in the system
by executing predefined actions.