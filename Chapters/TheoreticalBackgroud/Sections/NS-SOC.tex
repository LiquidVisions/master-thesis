\subsubsection{Separation of Concerns}
Since the early years of software engineering, \gls{soc} has been one of the most
fundamental software engineering principles. The principle has first been mentioned by
\citeauthor{dijkstra_selected_1982}\footnote{\url{https://en.wikipedia.org/wiki/Separation_of_concerns}}
as the crucial principle to design modular software architecture
\parencite[]{dijkstra_selected_1982}. The concept itself was introduced by
\citeauthor{parnas_criteria_1972} in his book \citetitle*{parnas_criteria_1972}.

This principle promotes the idea that a program should be divided into distinct sections,
each addressing a separate concern or aspect of the problem. This allows for a more
organized and maintainable source code, as changes to one concern do not affect the
others. \gls{soc} should be applied at the level of individual modules, rather that the level
of an entire program.

The \gls{soc} had its effect on later versions of software engineering principles like SOLID.
The principles of 'Single Responsibility' and 'Interface Segregation' are directly derived
from \gls{soc}. It also affected the theorems of Normalized Systems, although it has a more
strict definition of this principle. In the book of \citeauthor{mannaert_normalized_2016}
it is described as followed: 

\begin{center}
    \textbf{Theorem I}\\
    \textit{A processing function can only contain a single task to achieve stability.}    
\end{center}

There are various manifestations of \gls{soc} observable in the artifacts. One of which is
the separation of each task that is needed in order to generate the Clean Architecture
Expander. In total there are 19 tasks performed in order to fully expand the expanded
artifact. Most of them are concerned with generating templates based on the model of the
expanded artifact. Each of those tasks