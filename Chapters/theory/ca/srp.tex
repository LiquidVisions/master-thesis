\subsubsection{The Single Responsibility Principle} \label{subsubsec_srp}

The \gls{srp} is one of the fundamental design principles of \gls{ca}. The principle
advocates designing systems with high cohesion and low coupling. The \gls{srp} states that
each module in a system should have only one reason to change. That is, it should have a
single responsibility. No matter the granularity of the module, so implementations of
methods, classes, libraries and architecture layers should adhere to \gls{srp}. By
adhering to the principle, each module becomes highly cohesive, meaning that its
responsibilities are closely related and well-defined, while also being decoupled from
other modules \parencite[81]{robert_c_martin_clean_2018}.

The final statement of \gls{srp} is as followed
\parencite[82]{robert_c_martin_clean_2018}.

\mycolorbox{A module should	be responsible to one, and only one, actor.}{\acrlong{srp}}

\gls{srp} is closely related to the concept of \gls{soc}, which also advocates separating
a system into distinct parts. Although not that clearly stated in the literature,
\citeauthor{robert_c_martin_clean_2018} argues that \gls{soc} intends to have a separation
on a functional level and architectural level. This divides a system into different layers
or components based on their functional roles. \gls{srp} is concerned with separating the
responsibilities of individual modules regardless of the granularity of the module.
\parencite[205]{robert_c_martin_clean_2018}.With this in mind, \gls{srp} adheres more to
the definition of \gls{soc} from \gls{ns}. \textit{A processing function can only contain
a single task to achieve stability.} (see chapter \ref{subsubsec_soc}
\nameref{subsubsec_soc}).

There are various manifestations of \gls{srp} implemented in the artifacts. One of which is
already mentioned in Figure \ref{fig_modulair_components}, where \gls{srp} is applied to
separate the domain logic from the application, infrastructure and presentation logic. One
could argue that this manifestation is more related to \gls{soc}, considering the high
granularity of the components.

A better example is the separation of handlers that are part of the \gls{ca}
Expander. Each of those handlers executes an isolated part of the expanding process.
Consider the Listing \ref{list_entityexpander} \nameref{list_entityexpander}
\parencite{koks_expandentitieshandlerinteractor_2023} for example. This Handler is solely
responsible for the generation of data entities. 

\begin{figure}[H]
    \centering
    \includegraphics[width=0.6\textwidth]{figures/expander_handlers.pdf}
    \caption[handlers]{Each of the handlers handles an isolated part of the expanding process.}
    \label{fig_handlers}
\end{figure}

\lstinputlisting[
    caption={The \citetitle{koks_expandentitieshandlerinteractor_2023}},
    label={list_entityexpander}]
    {Snippets/ExpandEntitiesHandlerInteractor.cs}