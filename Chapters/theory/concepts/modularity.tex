\subsection{Modularity}

The challange with the concept of modularity lies in the definition of the term. Both
\gls{ca} and \gls{ns} use a slightly different definition.
\textcite[82]{robert_c_martin_clean_2018} describes a module as a piece of code that is 
encapsulated in a source file and a cohesive set of functions and data structures.
\textcite[22]{mannaert_normalized_2016} explains a module as a part of a system,
that exhibits a high degree of cohesion and operates independently of other parts of the
system.

Whilst both design approaches agree on the cohesiveness of the internal parts of a module,
it seems that there is a slight difference in granularity in their definitions.