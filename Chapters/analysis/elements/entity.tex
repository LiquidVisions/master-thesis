\subsection{The Entity Element}

\evaluateElementTable{Entity}{tab_convergence_entity}{ \addEvalRow{Data & \fullAlignment &
    Both elements represent data objects that are part of the ontology or data schema of
    the application, and typically include attributes and relationship information. While
    both can contain a full set of attributes and relationships, the Data Element of
    \gls{ns} may also be tailored to serve a specific set of information that is required
    for a single task or use case. In \gls{ca} these type of Data Elements are specified
    explicitly as ViewModels, RequestModels or ResponseModels. Code Listing
    \ref{list_entity} illustrates the similarities of between an \glspl{ca} Entity and the
    Data Element from \gls{ns} }

    \addEvalRow{Task & \noAlignment & The Entity has no alignment with the Task element of
        \gls{ns}. However, the Tasks element might operate on entities to perform business
        logic.}

    \addEvalRow{Flow & \noAlignment & The Entity and Flow elements are not aligned, as the
        Flow element represents the control between Tasks in \gls{ns}, while the Entity in
        \gls{ca} represents domain objects.}

    \addEvalRow{Connector & \noAlignment & The Entity and Connector element from \gls{ns}
        are not aligned, as the Connector element in \gls{ns} is involved in the
        communication separation of between components, while the Entity in \gls{ca}
        represents domain objects.}
    
    \addEvalRow{Trigger & \noAlignment & The Entity element and Trigger element are not convergent,
        as the Trigger element in \gls{ns} is about event-based execution of Tasks, while
        the Entity in \gls{ca} represents domain objects.}
}