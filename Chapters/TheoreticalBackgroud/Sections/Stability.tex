\section{Towards stable software architectures} \label{sec:on_stability}


\gls{ns} originated in the field of software engineering, aiming to achieve modular and
stable software artifacts. However, the underlying theory of \gls{ns} can be applied to
various other domains, such as Enterprise Engineering, Business Process Modeling, and
document management. This research acknowledges the software engineering background of
gls{ns}. It consistently refers to software and Information Systems when referring to
\enquote*{artifacts}. However, the reader should realize that the concepts and artifacts
are not restricted to software artifacts alone.

In several disciplines stability has been defined as \emph{Bounded Input Bounded Output}
(BIBO). It is the fundamental property of a system when subjected to bounded input
disturbances. BIBO stability ensures that the output of a system will also be bounded,
preventing uncontrolled or unexpected behavior \parencite[270]{mannaert_normalized_2016}. 

A real-world example of the importance of stability is the Tacoma Narrows Bridge in
Washington State, USA. The bridge, depicted in figure \ref*{fig:bridge}, collapsed on
November the 7th, 1940. This was caused due to wind-induced oscillations called
aeroelastic flutter. The wind (Input) induced oscillations in the bridge, causing it to
start swaying back and forth (Output). These oscillations were initially small, but as
they continued, they began to increase in amplitude or magnitude, causing the bridge to
collapse.

\begin{figure}[H]
    \centering
    \includegraphics[width=0.6\textwidth]{Figures/bridge.pdf}
    \caption[TNB]{Tacoma Narrows Bridge (Galloping Gertie)}
    \label{fig:bridge}
\end{figure}

Stability can also be used in the context of software engineering. In the context of
\gls{ns}, it is considered a critical property that ensures that the software is not
excessively sensitive to small changes \parencite[270]{mannaert_normalized_2016}. New
functional requirements should only lead to fixed, and an expected amount of changes in
the source code. Conversely, instabilities occur when the total number of modifications
relies on the size of the software artifact. The number of changes will grow over time in
parallel with the growth of the software artifact. These instabilities are referred to as
combinatorial effects \parencite[270]{mannaert_normalized_2016}. when combinatorial effects
are absent, the software artifact can be considered evolvable.