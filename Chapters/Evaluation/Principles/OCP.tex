\subsection{Converging the Open/Closed Principle}

\begin{table}[H]
    \begin{tabular}{ l | c | p{0.80\linewidth}}
        \toprule
        \gls{soc} & \conv & The \gls{ocp} converges with the \gls{soc} theorem.
        \gls{ocp} states that software implementations should be open for extension but
        closed for modification. When applying \gls{ocp} correctly, modifications are
        separated from the original implementations. For example by creating a new
        implementation of an interface or a base class. Conversely, adhering to \gls{soc}
        does not guarantee the fulfillment of \gls{ocp}, as \gls{soc} focuses on
        modularization and encapsulation, rather than the extensibility of modules. \\
        \midrule
        \gls{dvt} & \partconv & The \gls{ocp} supports the \gls{dvt} theorem. While
        \gls{dvt} aims to handle changes in data structures without impacting the system,
        \gls{ocp} focuses on the extensibility of software entities. \gls{ocp} does not
        explicitly address data versioning, and thus does not guarantee full convergence
        with \gls{dvt}. However, by designing modules that follow \gls{ocp}, developers
        can create components that are more adaptable to changes in data structures. \\
        \midrule
        \gls{avt} & \conv & The \gls{ocp} converges with the \gls{avt} theorem. Both
        principles emphasize the importance of allowing changes or extensions to actions
        or operations without modifying existing implementations. By adhering to
        \gls{ocp}, developers can create modules that can be extended to accommodate new
        actions or changes in existing ones, effectively achieving \gls{avt}. \\
        \midrule
        \gls{sos} & \noconv & The \gls{ocp} has an indirect relationship with the
        Separation of States (SoSt) theorem. However, not on a level where we can speak of
        convergence. \gls{sos} emphasizes isolating different states within a system.
        Adhering to \gls{ocp} alone does not guarantee full separation of states. \\
        \bottomrule
    \end{tabular}
    \caption{Converge \gls{ocp} with \gls{ns}}
    \label{tab_convergence_ocp}
\end{table}