\section{Converging principles} \label{sec_converging_principles}

In order to address \nameref{rq3} outlined in Chapter \ref{sec_research_questions}, a
comprehensive analysis of the existing literature on the \gls{solid} principles and
\gls{ns} theorems have been conducted. Furthermore, the development of the artifact has
provided valuable insights into this subject matter.

In the following sections, a systematic cross-referencing approach has been applied to
assess the level of convergence between each of the \gls{solid} principles of \gls{ca}
and the \gls{ns} theorems. Along with a brief explanation, the level of convergence is
denoted as follows:

\begin{table}[H]
    \begin{tabular}{ l l p{0.57\linewidth}} Fully converges & \converges & This indicates
        a high degree of alignment between the respective \gls{solid} principle and
        \gls{ns} theorem. The application of either principle or theorem results in a
        similar impact on the software design. \\
        Supports convergence & \supports & In this case, the \gls{solid} principle
        assists in implementing the \gls{ns} theorem through specific design choices.
        However, it is essential to note that applying the principle does
        not inherently ensure adherence to the corresponding theorem. \\
        No convergence & \diverges & This denotation signifies a lack of alignment between
        the \gls{solid} principle and the corresponding theorem. \\
    \end{tabular}
\end{table}
