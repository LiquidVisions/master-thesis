\subsection{Converging the Interface Segregation Principle}

\begin{table}[H]
    \begin{tabular}{ l | c | p{0.80\linewidth}}
        \toprule
        \gls{soc} & \converges & The \gls{isp} converges with the \gls{soc} theorem, as
        both principles emphasize the importance of modularity and the separation of
        concerns. \gls{isp} states that clients should not be forced to depend on
        implementation they do not use, promoting the creation of smaller, focused
        interfaces. By adhering to \gls{soc} and designing target interfaces, inherently
        support \gls{soc}, leading to more evolvable software systems. \\
        \midrule
        \gls{dvt} & \supports & The \gls{isp} has an indirect relationship with the
        \gls{dvt} theorem. When adhering to the \gls{isp}, a developer can create a
        specific interface supporting a specific version of data entities. Although this
        approach can help minimize the impact of data structure changes on the system, it
        does not guarantee full \gls{dvt} information system. \\
        \midrule
        \gls{avt} & \supports & The \gls{isp} supports the \gls{avt} theorem. Both
        principles emphasize the importance of separating actions within a system.
        \gls{isp} promotes the creation of interfaces for each (version of an) action,
        which aligns with the core concept of \gls{avt}. This alignment allows for a more
        manageable system, where changes in one action do not lead to ripple effects
        throughout the entire system. Although the adherence to \gls{isp} does not
        guarantee full compliance with the \gls{avt} \\
        \midrule
        \gls{sos} & \diverges &  There is no alignment between \gls{isp} and \gls{sos}.
        Mostly because \gls{isp} focuses on the separation of interfaces and abstract
        classes, while \gls{sos} emphasizes isolating different states within a system.
        This is an implementation concern. By no means the application of \gls{isp} could
        lead to a separation of state. \\
        \bottomrule
    \end{tabular}
    \caption{Converge \gls{isp} with \ns}
    \label{tab_convergence_isp}
\end{table}