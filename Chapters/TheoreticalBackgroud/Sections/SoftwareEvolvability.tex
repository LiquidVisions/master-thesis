\section{Software evolvability} \label{software_evolvability}

<<inleiding en benoemen dat de onderstaande onderwerpen van invloed zijn op de
evolvability van software systemen. Per onderwerp een paragraaf of twee.>>

\subsection{Stability}
<<verwijzing naar positive feedback loops uit de theorie van NS en de effect op
evolueerbaarheid van software.>>

\subsection{Combinatorial effects \& anticipated change drivers.}
<<heeft relatie tot hoofdstuk evaluation en toelichtingen dat software aanpassingen niet
zouden moeten leiden tot een toename in combinatorial effects>>

\subsection{Modularity}
<<toelichten waarop modularity invloed heeft op de evolueerbaarheid van de software>>

\subsection{On Cohesion}
<<toelichten waarop Cohesion invloed heeft op de evolueerbaarheid van de software>>

Cohesion in software engineering refers to the degree to which the different structural
parts of a software system work together to achieve a single and well-defined purpose, a
common goal. 

There is a considerable body of scientific evidence supporting the importance of software
cohesion. Several studies correlate high levels of cohesion with fewer defects and are
likely to be more maintainable. It has shown that software engineering artifacts are more
open to change. Having a high cohesion is often referred to have a positive effect on
software quality attributes like reliability, maintainability, reusability and thus the
evolvability of the software artifact. 

\subsection{Coupling}
<<toelichten waarop Coupling invloed heeft op de evolueerbaarheid van de software>>