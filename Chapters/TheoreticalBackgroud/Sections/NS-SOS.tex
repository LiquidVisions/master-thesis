\subsubsection{Separation of State}

\gls{sos} is a theorem that is based on the idea that processing functions should not
contain any state information but instead should rely on external data structures to store
state information. By separating state information from processing functions, Normalized
Systems can achieve a higher level of flexibility and adaptability. External data
structures can be updated or replaced without affecting the processing functions
themselves, which greatly reduces the change of unwanted ripple effects. This theorem is
described as followed: \parencite[258]{mannaert_normalized_2016}.

\begin{center}
    \textbf{Theorem IV}\\
    \textit{Calling a processing function within another processing function, needs to exhibit state keeping in order to achieve stability.}
\end{center}

\gls{sos} fits very well in distributed information systems with asynchronous calls. The
expanded artifact is designed in a manner that all external process functions are executed
asynchronously. \todo{snippet toevoegen.}.

A simpler manifestation of the \gls{sos} theorem involves the use of Resources as an
integral part solution
\footnote{url{https://learn.microsoft.com/en-us/dotnet/core/extensions/resources}}. In
addition to enabling the localization of strings, this approach offers the advantage of
centralized management, thereby exhibiting \gls{sos}. For instance, when the functional
requirements evolve, the name of a template in the expander artifact is likely to change.
As the name of the template is used in multiple class instances, a centralized approach to
managing the template name can mitigate the risk of excessive modifications when a name
change is mandated.

Another example. The state of the model (see chapter \ref{sec:artifact_model}) is
currently persisted in an Azure SQL Database.

\lstinputlisting[
    caption={The \citetitle{koks_genericrepository_2023} \parencite{koks_genericrepository_2023}}]
    {Snippets/GenericRepository.cs}

The state of of the expander artifact model as described in \ref{sec:artifact_model} for the expander in the 
voorbeeld met seerders...
voorbeeld met repositories....