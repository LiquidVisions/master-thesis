\section{Normalized Systems: Impacting software stability} \label{sec_ns_theory}

\gls{ns} is a software development approach that prioritizes achieving software stability
through the use of standardized, modular components and interfaces. This theory is
informed by several scientific disciplines, including systems theory, mathematics, and
computer science, as well as some other software development approaches, such as agile
development and domain-driven design.

\gls{ns} originated in the field of software engineering. However, the underlying theory
of \gls{ns} can be applied to various other domains, such as Enterprise Engineering,
Business Process Modeling, and document management. This research acknowledges the
software engineering background of \gls{ns}. It consistently refers to software and
Information Systems when referring to \enquote*{artifacts.} However, the reader should
realize that the concepts and artifacts are not restricted to software artifacts alone.

\subsection{The study of Stability, Evolvability, and Combinatorics} \label{subsec_on_stability}

The \gls{ns} Theory considers stability a crucial property derived from the concept
\gls{bibo}: A bounded functional change must result in a bounded amount of work,
independent of the size of the system. Instabilities occur when the total number of
changes relies on the size of the system. The bigger the size of the system, the more
changes are required to implement the new requirement.
\textcite[271]{mannaert_normalized_2016} refer to these \enquote*{instabilities} as
Combinatorial Effects. Conversely, stability is achieved when a system is free from these
so-called Combinatorial Effects. Based on the concept of stability,
\textcite{mannaert_towards_2012} require information systems to be stable with respect to
a set of anticipated changes to exhibit high evolvability.
\subsection{Evolvability} \label{sec_on_evolvability}

In section \ref{subsec_on_stability}, it was highlighted that stability in a system is
achieved when it exhibits low sensitivity to minor changes, regardless of the system's
size. On the other hand, evolvability refers to a system's ability to adapt and adjust to
changing requirements continuously. Time is a critical factor here, as systems are often
easier to maintain during the initial stages. Evolvability, however, relates to the
system's ability to evolve and adapt independently of time.
\subsection{Modularity} \label{subsec_modularity}

Both CA and NS use a slightly different definitions for the concept of modularity.
\textcite[82]{robert_c_martin_clean_2018} describes a module as a piece of code
encapsulated in a source file with a cohesive set of functions and data structures.
According to \textcite[22]{mannaert_normalized_2016}, modularity is a hierarchical or
recursive concept that should exhibits a high degree of cohesion. While both design
approaches agree on the cohesiveness of a module's internal parts, there seems to be a
slight difference in granularity in their definitions.
\chapter{Component Cohesion Principles} \label{appendix_cohesion_principles}

\begin{table}[H]
    \small
    \begin{tabular}{ P{0.25\linewidth} | p{0.69\linewidth}} 
        \hline
        \textbf{Name} & \textbf{Description} \\ \hline
        \acrlong{rep} & \acrshort{rep} is a concept related to software development that
        refers to the balance between reusing existing software components and releasing
        new ones to ensure the efficient use of resources and time
        \parencite[104]{robert_c_martin_clean_2018}.\\ \midrule 
        
        \acrlong{ccp} & In the context of Clean Architecture, the \acrshort{ccp} states
        that classes or components that change together should be packaged together. In
        other words, if a group of classes is likely to be affected by the same kind of
        change, they should be grouped into the same package or module. This approach
        enhances the maintainability and modularity of the software
        \parencite[105]{robert_c_martin_clean_2018}.\\ \midrule 
        
        \acrlong{crp} & \acrshort{crp} states that classes or components that are reused
        together should be packaged together. It means that if a group of classes tends to
        be used together or has a high level of cohesion, they should be grouped into the
        same package or module. This approach aims to make it easier for developers to
        reuse components and understand their relationships
        \parencite[107]{robert_c_martin_clean_2018}.\\

        \bottomrule
    \end{tabular}
    \caption{The Component Cohesion Principles}
    \label{appendix_tab_cohesion_principles}
\end{table}

Cohesion facilitates the reduction of complexity and interdependence among the components
of a system, thereby contributing to a more efficient, maintainable, and reliable system.
By organizing components around a shared purpose or function or by standardizing their
interfaces, data structures, and protocols, cohesion can offer the following benefits:

\begin{itemize}
    \item \textbf{Reduce redundancy and duplication of effort}: \\
    Cohesion ensures that components are arranged around a common purpose or function,
    reducing duplicates or redundant code. This simplifies system comprehension,
    maintenance, and modification.
    \item \textbf{Promoting code reuse:}\\
    Cohesion facilitates code reuse by making it easier to extract and reuse components
    designed for specific functions. This saves time and effort during development and
    enhances overall system quality.
    \item \textbf{Enhance maintainability:}\\
    Cohesion decreases the complexity and interdependence of system components, making it
    easier to identify and rectify bugs or errors in the code. This improves system
    maintainability and reduces the risk of introducing new errors during maintenance.
    \item \textbf{Increase scalability:}\\
    Cohesion improves a system's scalability by enabling it to be extended or modified
    effortlessly to accommodate changing requirements or conditions. By designing
    well-organized and well-defined components, developers can easily add or modify
    functionality as needed without disrupting the rest of the system.  
\end{itemize}
\subsection{Low Coupling} \label{subsec_on_coupling}

Coupling is an essential concept in software engineering related to the degree of
interdependence among software modules and components. High coupling between modules
indicates the strength of their relationship, whereby a high level of coupling implies a
significant degree of interdependence. Conversely, low coupling signifies a weaker
relationship between modules, where modifications in one module are less likely to impact
others. Although not always possible, the level of coupling between the various modules of
the system should be kept to a bare minimum. Both \textcite[23]{mannaert_normalized_2016}
and \textcite[130]{robert_c_martin_clean_2018} agree with the idea that modules should be
coupled as loosely as possible
\subsection{Expansion} \label{subsec_expansion}

Creating and maintaining a stable and evolvable system is, according to
\textcite[403]{mannaert_normalized_2016}, a particularly challenging, repetitive, and
meticulous engineering job. Developers must have a sound knowledge of NS while
implementing new requirements in an always consistent manner.
\textcite[403]{mannaert_normalized_2016} propose to automate the instantiation process of
software structures by using code generation for recurring tasks. This process is referred
to expansion.


\subsection{The Design Theorems} \label{subsec_ns_desing_theorems}

In the following table we will describe the Design Theorems of \gls{ns}, firstly presented
by \textcite[111-119]{mannaert_normalized_2009}. They are known as \gls{soc}, \gls{dvt},
\gls{avt} and \gls{sos}.

\begin{table}[H]
    \begin{tabular}{ p{0.15\linewidth} p{0.75\linewidth}}
        \hline
        \textbf{Principle} & \textbf{Definition} \\ 
        \hline
        \gls{soc} & \gls{ns} has adopted the \gls{soc} principle. However,
        \textcite[112]{mannaert_normalized_2009} defined a more strict definition. \emph{A
        processing function can only contain a single task to achieve stability.} \\
        
        \gls{dvt} &  A data structure that is passed through the interface of a processing function needs to
        exhibit version transparency to achieve stability.\\
        
        \gls{avt} & A processing function that is called by another processing function, needs to exhibit version
        transparency to achieve stability.\\
        
        \gls{sos} & Calling a processing function within another processing function, needs to exhibit state
        keeping to achieve stability.\\
        
        \bottomrule
    \end{tabular}
    \caption{The Design Theorems of Normalized Systems.}
    \label{ns_principles}
\end{table}
\input{chapters/theoreticalbackgroud/sections/normalizedsystems/soc}
\input{chapters/theoreticalbackgroud/Sections/normalizedsystems/dtv}
\input{chapters/theoreticalbackgroud/sections/normalizedsystems/avt}
\input{chapters/theoreticalbackgroud/sections/normalizedsystems/sos}
\section{An analysis of Elements} \label{sec_converging_elements}

In this section, we will apply a systematic cross-referencing approach to each of the
elements of \gls{ca} with \gls{ns}. By cross-referencing these elements, we aim to
uncover the degree of convergence between \gls{ca} and \gls{ns} from a theoretical
perspective, supported by examples from the artifacts. Along with this explanation, the
level of convergence is denoted as follows:

\begin{table}[H]
    \begin{tabular}{ l l p{0.57\linewidth}} 
        
    Strong convergence & \fullConvergence & Both elements have a high level of similarity or
    are closely related in terms of their purpose, structure, or functionality.\\

    Supports convergence & \npartialConvergence &  Both elements have some similarities or share
    certain aspects in their purpose, structure, or functionality, but they are not identical
    or directly interchangeable.\\

    No or weak convergence & \noConvergence &  The elements are unrelated or have no significant
    similarities in terms of purpose, structure, or functionality.\\
    \end{tabular}
\end{table}

\subsection{The Entity Element}

\evaluateElementTable{Entity}{tab_convergence_entity}{ \addEvalRow{Data & \fullConvergence
    & Both elements represent data objects that are part of the ontology or data schema of
    the  and typically include attributes and relationship information. While both can
    contain a complete set of attributes and relationships, the Data Element of \gls{ns}
    may also be tailored to serve a specific set of information required for a single task
    or use case. In \gls{ca}, these type of Data Elements are explicitly specified as
    ViewModels, RequestModels or ResponseModels. Code Listing \ref{list_entity}
    illustrates an example of an Entity in a way that is very similar to the Data Element
    from \gls{ns} \parencite{koks_entity_2023}}

    \addEvalRow{Task & \noConvergence & There is no convergence between the Entity element of
    \gls{ca} and the Connector element of \gls{ns}, and no manifestations are found in the
    artifact.}

    \addEvalRow{Flow & \noConvergence & There is no convergence between the Entity element of
    \gls{ca} and the Flow element of \gls{ns}, and no manifestations are found in the
    artifact.}

    \addEvalRow{Connector & \noConvergence & There is no convergence between the Entity
    element of \gls{ca} and the Connector element of \gls{ns}, and no manifestations are
    found in the artifact.}
    
    \addEvalRow{Trigger & \noConvergence & There is no convergence between the Entity element
    of \gls{ca} and the Trigger element of \gls{ns}, and no manifestations are found in the
    artifact.} }
\subsection{The Interactor Element}

\evaluateElementTable{Interactor}{tab_convergence_interactor}{ \addEvalRow{Data &
    \noAlignment &  There is no alignment between the Interactor element of \gls{ca} and
    the Data element of \gls{ns} and no manifestations are found in the Artifact.}

    \addEvalRow{Task & \fullAlignment &  The Interactor element of \gls{ca} has a strong
    alignment with the Task element of \gls{ns}, as both encapsulate the execution
    of business rules. This is illustrated in Code Listing \ref{list_interactor_task}
    which aligns with an implementation of a Task, having a single execution of a business
    rule \parencite{koks_createentityvalidator_2023}. }

    \addEvalRow{Flow & \fullAlignment & The Interactor has a strong alignment with the
        Flow element of \gls{ns} as the both elements can orchestrates the flow of execution for
        a use case, which can involve multiple Tasks in \gls{ns}. This is clearly
        illustrated in Code Listing \ref{list_interactor_flow}, where the Interactor
        handles Validation, mapping and persistance of the Entity \parencite*{koks_createentityinteractor_2023}. }

    \addEvalRow{Connector & \noAlignment & There is no alignment between the Interactor
    element of \gls{ca} and the Connector element of \gls{ns} and no manifestations are
    found in the Artifact.}
    
    \addEvalRow{Trigger & \noAlignment & There is no alignment between the Interactor
    element of \gls{ca} and the Trigger element of \gls{ns} and no manifestations are
    found in the Artifact.} }
\input{chapters/analysis/elements/requestmodel}
\input{chapters/analysis/elements/responsemodel}
\input{chapters/analysis/elements/viewmodel}
\subsection{The Controller Element} \label{converging_controller_element}

\evaluateElementTable{Controller}{tab_convergence_controller}{ \addEvalRow{Data &
    \noConvergence & There is no convergence between the Controller element of \gls{ca} and
    the Data element of \gls{ns} and no manifestations are found in the Artifact.}

    \addEvalRow{Task & \noConvergence &  There is no convergence between the Controller
    element of \gls{ca} and the Task element of \gls{ns} and no manifestations are found
    in the Artifact.}

    \addEvalRow{Flow & \noConvergence &  There is no convergence between the Controller
    element of \gls{ca} and the Flow element of \gls{ns} and no manifestations are found
    in the Artifact.}

    \addEvalRow{Connector & \npartialConvergence & Although the Controller of \gls{ca}
    supports the intent of the Connector element of \gls{ns}, they are only partially
    interchangeable. While both elements are involved in the interaction between
    components, the Controller element from \gls{ca} primarily intends to interact with
    external systems using a specific protocol or technology involving user or web
    interfaces. An example of such a Controller is illustrated in Code Listing
    \ref{list_entitycontroller}. \parencite{koks_entitycontroller_2023}. In this example,
    the Controller exposes a Restful interface.}
    
    \addEvalRow{Trigger & \npartialConvergence & Although the Controller of \gls{ca} supports
    the intent of the Trigger element of \gls{ns}, they are only partially interchangeable.
    While both elements are involved in receiving events from external systems, a Controller
    is also able to initiate communication with the same external systems using specific
    protocols or technologies involving user or web interfaces. } }
\subsection{The Gateway Element}

\evaluateElementTable{Gateway}{tab_convergence_gateway}{
    \addEvalRow{ Data & \noAlignment &  The Gateway and Data element are not convergent. Nevertheless,
        the Gateway element might interact with data entities when providing access to
        external resources or systems.}

    \addEvalRow{Task & \noAlignment &  The Gateway and Task element are not convergent. Nevertheless,
        the Task elements might use gateways when interacting with external resources or systems
        during the execution of business logic.}

    \addEvalRow{Flow & \noAlignment & The Gateway and Flow element are not directly related, as the Flow
        element represents the orchestration between Tasks in \gls{ns}, whilst the Gateway
        element in \gls{ca} provide access to external resources or systems.}

    \addEvalRow{Connector & \fullAlignment & The Gateway and Connector element have a strong convergence, as both
        are involved in communication between components and provide interfaces for
        accessing external resources or systems.}
    
    \addEvalRow{Trigger & \noAlignment & The Gateway and Trigger element are not convergent, as the
        Trigger in \gls{ns} is about event-based execution of Tasks, whilst the Gateway in
        \gls{ca} provide access to external resources or systems.}
}
\subsection{The Presenter Element}

\evaluateElementTable{Presenter}{tab_convergence_presenter}{ \addEvalRow{Data &
    \noConvergence &  There is no convergence between the Presenter element of \gls{ca} and
    the Data element of \gls{ns}, and no manifestations are found in the artifact.}

    \addEvalRow{Task & \npartialConvergence &  The Presenter is responsible for preparing the
    ViewModel on behalf of the Controller and can be considered a task Element with a
    narrow scope. Because of this narrow scope, the elements are not fully
    interchangeable. Code Listing \ref{list_createentitypresenter} illustrated the inner
    workings of a Presenter \parencite{koks_createentitypresenter_2023}. }

    \addEvalRow{Flow & \npartialConvergence & Presenters can handle multiple Tasks when this
    is required. in this case there is also some convergence between the Presenter Element
    of \gls{ca} with the Flow Element of \gls{ns}.}

    \addEvalRow{Connector & \noConvergence & There is no convergence between the Presenter
    element of \gls{ca} and the Connector element of \gls{ns}, and no manifestations are
    found in the artifact.}
    
    \addEvalRow{Trigger & \noConvergence & There is no convergence between the Presenter
    element of \gls{ca} and the Trigger element of \gls{ns}, and no manifestations are
    found in the artifact.} }
\subsection{The Boundary Element}

\evaluateElementTable{Boundary}{tab_convergence_boundary}{ \addEvalRow{Data & \noAlignment
    &  There is no alignment between the Boundary element of \gls{ca} and the Data
    element of \gls{ns} and no manifestations are found in the Artifact.}

    \addEvalRow{ Task & \noAlignment &  There is no alignment between the Boundary element
    of \gls{ca} and the Task element of \gls{ns} and no manifestations are found in the
    Artifact.}

    \addEvalRow{ Flow & \noAlignment & There is no alignment between the Boundary
    element of \gls{ca} and the Flow element of \gls{ns} and no manifestations are found
    in the Artifact.}

    \addEvalRow{Connector & \fullAlignment & The Boundary element of \gls{ca} has a strong
    alignment with the Connector element of \gls{ns}, as both are involved in communication
    between components and help ensure loose coupling between these components. However,
    the Boundary element's scope seems narrower, as this element usually separates
    architectural boundaries within the application or component. In the Code Listing
    Example \ref{list_CreateEntityBoundary} we can notice that the main purpose of the
    Boundary is to separate the inner parts of the Application Layer from the Presentation
    Layer, which aligns with the goal of the Connector Element of \gls{ns}}
    
    \addEvalRow{ Trigger & \noAlignment & There is no alignment between the Boundary
    element of \gls{ca} and the Task element of \gls{ns} and no manifestations are found
    in the Artifact.} }
\section*{Evaluating the findings}

\begin{table}[!ht]
    \centering
    \begin{tabular}{lcccc}
    \toprule
     & SoC & DVT & AVT & SoS \\
    \midrule
    SRP & \fullAlignment & \partialAlignment & \partialAlignment & \noAlignment \\
    OCP & \fullAlignment & \noAlignment & \fullAlignment & \noAlignment \\
    LSP & \fullAlignment & \noAlignment & \partialAlignment & \noAlignment \\
    ISP & \fullAlignment & \partialAlignment & \partialAlignment & \noAlignment \\
    DIP & \fullAlignment & \partialAlignment & \partialAlignment & \noAlignment \\
    \bottomrule
    \end{tabular}
    \caption{Convergence between the SOLID and \gls{ns} principles}
    \label{tab_convergence_principles_summarized}
    \end{table}

    \begin{table}[!ht]
        \centering
        \begin{tabular}{lccccc}
        \toprule
         & Data & Task & Flow & Connector & Trigger \\
         \midrule
        Entity & \fullAlignment & \noAlignment & \noAlignment & \noAlignment & \noAlignment \\
        Interactor & \noAlignment & \fullAlignment & \partialAlignment & \noAlignment & \noAlignment \\
        RequestModel & \partialAlignment & \noAlignment & \noAlignment & \noAlignment & \noAlignment \\ 
        ResponseModel & \partialAlignment & \noAlignment & \noAlignment & \noAlignment & \noAlignment \\
        ViewModel & \partialAlignment & \noAlignment & \noAlignment & \noAlignment & \noAlignment \\
        Controller & \noAlignment & \noAlignment & \noAlignment & \partialAlignment & \partialAlignment \\
        Gateway & \noAlignment & \noAlignment & \noAlignment & \fullAlignment & \noAlignment \\
        Presenter & \noAlignment & \noAlignment & \noAlignment & \noAlignment & \noAlignment \\
        Boundary & \noAlignment & \noAlignment & \partialAlignment & \fullAlignment & \noAlignment \\ \bottomrule
        
        \end{tabular}
        \caption{Convergence between the SOLID and \gls{ns} elements}
        \end{table}

