\section{Introducing Normalized Systems Theorems}

Normalized Systems theorems is a scientific approach to creating software systems based on
the laws for software evolvability. These theorems have resulted in a documented track
record of achieving software stability. Effectively, it reduces the number and impact of
combinatorial effects in the source code. Combinatorial effects occur when the impact of a
change is dependent on the size of the information systems.
\parencite[]{mannaert_normalized_2009}. 

Normalized Systems formulate their theorem as prescriptive structures (elements)
that will lead to a modular architecture with low coupling and high cohesion. The
software architecture will be designed to cope with future change
\parencites[]{mannaert_normalized_2009}.         