\section{Converging principles} \label{sec_converging_principles}

In order to address the goal of this research outlined in Chapter
\ref{sec_research_questions}, a comprehensive analysis of the existing literature on the
\gls{solid} principles and \gls{ns} theorems have been conducted. Furthermore, the
development of the artifact has provided valuable insights into this subject matter.

In the following sections, a systematic cross-referencing approach has been applied to
assess the level of convergence between each of the \gls{solid} principles of \gls{ca}
and the \gls{ns} theorems. Along with a brief explanation, the level of convergence is
denoted as follows:

\begin{table}[H]
    \begin{tabular}{ l l p{0.57\linewidth}} Fully converges & \converges & This indicates
        a high degree of alignment between the respective \gls{solid} principle and
        \gls{ns} theorem. The application of either principle or theorem results in a
        similar impact on the software design. \\
        Supports convergence & \supports & In this case, the \gls{solid} principle
        assists in implementing the \gls{ns} theorem through specific design choices.
        However, it is essential to note that applying the principle does
        not inherently ensure adherence to the corresponding theorem. \\
        No convergence & \diverges & This denotation signifies a lack of alignment between
        the \gls{solid} principle and the corresponding theorem. \\
    \end{tabular}
\end{table}

\subsection{Converging the Single Responsibility Principle}

\begin{table}[H]
    \begin{tabular}{ l | c | p{0.78\linewidth}}
        \toprule
        \gls{soc} & \converges & \gls{srp} and \gls{soc} share a common objective:
        facilitating evolvable software systems through the promotion of modularity, low
        coupling, and high cohesion. While there may be some differences in granularity
        when applying both principles according to the original definition of SOC, the
        more stringent definition of Separation of Concerns, offered by the \ns
        Theorems \ref{subsubsec_soc} minimizes these differences. As a result, the two
        principles can be regarded as practically interchangeable. In conclusion, SRP and
        SOC exhibit full convergence, as they both emphasize encapsulating
        'responsibilities' or 'concerns' within modular components of a software system.
        \\
        \midrule
        \gls{dvt} & \supports & While not immediately apparent, \gls{srp} offers supports
        for the \gls{dvt} theorem. While \gls{srp} emphasizes limiting the responsibility of
        each module, it does not explicitly require handling changes in data structures.
        However, following gls{srp} can still indirectly contribute to achieving \gls{dvt}
        by promoting the Law of Demeter
        \footnote{\url{https://en.wikipedia.org/wiki/Law_of_Demeter}}, which encourages
        modules to interact with each other only through well-defined interfaces. This
        approach can minimize the impact of data structure changes, although it does not
        guarantee full convergence with \gls{dvt}. \\
        \midrule
        \gls{avt} & \supports & Although not that apparent, \gls{srp} supports the \gls{dvt}
        theorem. While \gls{srp} emphasizes limiting the responsibility of each module, it
        does not explicitly require handling changes in data structures. However, following
        gls{srp} can still indirectly contribute to achieving \gls{dvt} by promoting the Law
        of Demeter \footnote{\url{https://en.wikipedia.org/wiki/Law_of_Demeter}}, which
        encourages modules to interact with each other only through well-defined interfaces.
        This approach can minimize the impact of data structure changes, although it does not
        guarantee full convergence with \gls{dvt}. \\
        \midrule
        \gls{sos} & \diverges & The convergence between \gls{srp} and the \gls{sos} theorem is
        not as direct as with other theorems. \gls{srp} focuses on assigning a single
        responsibility to each module but does not explicitly address state management.
        Nevertheless, by following \gls{srp}, developers can create modules that manage their
        state, which indirectly contributes to \gls{sos}. \\
        \bottomrule
    \end{tabular}
    \caption{Converge \gls{srp} with \ns}
    \label{tab_convergence_srp}
\end{table}

\subsection{Converging the Open/Closed Principle}

\begin{table}[H]
    \begin{tabular}{ l | c | p{0.78\linewidth}}
        \toprule
        \gls{soc} & \converges & The \gls{ocp} converges with the \gls{soc} theorem.
        \gls{ocp} states that software implementations should be open for extension but
        closed for modification. When applying \gls{ocp} correctly, modifications are
        separated from the original implementations. For example by creating a new
        implementation of an interface or a base class. Conversely, adhering to \gls{soc}
        does not guarantee the fulfillment of \gls{ocp}, as \gls{soc} focuses on
        modularization and encapsulation, rather than the extensibility of modules. \\
        \midrule
        \gls{dvt} & \supports & The \gls{ocp} supports the \gls{dvt} theorem. While
        \gls{dvt} aims to handle changes in data structures without impacting the system,
        \gls{ocp} focuses on the extensibility of software entities. \gls{ocp} does not
        explicitly address data versioning, and thus does not guarantee full convergence
        with \gls{dvt}. However, by designing modules that follow \gls{ocp}, developers
        can create components that are more adaptable to changes in data structures. \\
        \midrule
        \gls{avt} & \converges & The \gls{ocp} converges with the \gls{avt} theorem. Both
        principles emphasize the importance of allowing changes or extensions to actions
        or operations without modifying existing implementations. By adhering to
        \gls{ocp}, developers can create modules that can be extended to accommodate new
        actions or changes in existing ones, effectively achieving \gls{avt}. \\
        \midrule
        \gls{sos} & \diverges & The \gls{ocp} has an indirect relationship with the
        Separation of States (SoSt) theorem. However, not on a level where we can speak of
        convergence. \gls{sos} emphasizes isolating different states within a system.
        Adhering to \gls{ocp} alone does not guarantee full separation of states. \\
        \bottomrule
    \end{tabular}
    \caption{Converge \gls{ocp} with \gls{ns}}
    \label{tab_convergence_ocp}
\end{table}
\subsection{Converging the Liskov Substitution Principle}

\begin{table}[H]
    \begin{tabular}{ l | c | p{0.80\linewidth}}
        \toprule
        \acrshort*{soc} & \conv & Adhering to \gls{lsp} in the software design leads to a more
        modular design and separation of specific concerns. Therefore we state that \gls{lsp}
        converges with \gls{soc}. \gls{lsp} states that objects of a derived class should be
        able to replace objects of the base class without affecting the correctness of the
        program. This can only be achieved by a strict separation of concerns in combination
        with Action version Transparent implementations of the signature.
        \\
        \midrule
        \acrshort*{dvt} & \noconv & \gls{lsp} has a limited alignment with the \gls{dvt}
        theorem. While \gls{lsp} focuses on the substitutability of objects in class
        hierarchies, \gls{dvt} aims to handle changes in data structures without impacting
        the system. By following \gls{lsp}, developers can ensure that derived classes can
        be substituted for their base classes, which may help reduce the impact of data
        structure changes on the system. However, \gls{lsp} does not explicitly address
        data versioning and thus does not guarantee full convergence with \gls{dvt}. \\
        \midrule
        \acrshort*{avt} & \partconv & The \gls{lsp} supports the \gls{avt} theorem. Both principles
        emphasize the importance of allowing the extensibility of the system, without
        negatively impacting the desired requirements. By adhering to \gls{lsp}, developers
        can create class hierarchies that can be easily extended to accommodate new actions or
        changes in existing ones, which may contribute to achieving \gls{avt}. However,
        adhering to \gls{lsp} alone may not guarantee full convergence with \gls{avt}. 
        \\
        \midrule
        \acrshort*{sos} & \noconv & By designing class hierarchies according to \gls{lsp},
        developers can create components that are less prone to side effects caused by shared
        states. However, the alignment between \gls{lsp} and \gls{sos} is very weak, and
        adhering to \gls{lsp} alone may not guarantee full separation of states. 
        \\
        \bottomrule
    \end{tabular}
    \caption{Converge \gls{lsp} with \gls{ns}}
    \label{tab_convergence_lsp}
\end{table}
\subsection{Converging the Interface Segregation Principle}

\begin{table}[H]
    \begin{tabular}{ l | c | p{0.80\linewidth}}
        \toprule
        \gls{soc} & \converges & The \gls{isp} converges with the \gls{soc} theorem, as
        both principles emphasize the importance of modularity and the separation of
        concerns. \gls{isp} states that clients should not be forced to depend on
        implementation they do not use, promoting the creation of smaller, focused
        interfaces. By adhering to \gls{soc} and designing target interfaces, inherently
        support \gls{soc}, leading to more evolvable software systems. \\
        \midrule
        \gls{dvt} & \supports & The \gls{isp} has an indirect relationship with the
        \gls{dvt} theorem. When adhering to the \gls{isp}, a developer can create a
        specific interface supporting a specific version of data entities. Although this
        approach can help minimize the impact of data structure changes on the system, it
        does not guarantee full \gls{dvt} information system. \\
        \midrule
        \gls{avt} & \supports & The \gls{isp} supports the \gls{avt} theorem. Both
        principles emphasize the importance of separating actions within a system.
        \gls{isp} promotes the creation of interfaces for each (version of an) action,
        which aligns with the core concept of \gls{avt}. This alignment allows for a more
        manageable system, where changes in one action do not lead to ripple effects
        throughout the entire system. Although the adherence to \gls{isp} does not
        guarantee full compliance with the \gls{avt} \\
        \midrule
        \gls{sos} & \diverges &  There is no alignment between \gls{isp} and \gls{sos}.
        Mostly because \gls{isp} focuses on the separation of interfaces and abstract
        classes, while \gls{sos} emphasizes isolating different states within a system.
        This is an implementation concern. By no means the application of \gls{isp} could
        lead to a separation of state. \\
        \bottomrule
    \end{tabular}
    \caption{Converge \gls{isp} with \gls{ns}}
    \label{tab_convergence_isp}
\end{table}
\subsection{Converging the Interface Segregation Principle}

\begin{table}[H]
    \begin{tabular}{ l | c | p{0.79\linewidth}}
        \toprule
        \gls{soc} & \converges & This principle converges with the \gls{soc} theorem.
        \gls{dip} states that high-level modules should not depend on low-level modules.
        When unavoidable, high-level modules should depend on abstractions of low-level
        modules and abstractions should not depend on details. By adhering to \gls{dip}
        correctly, developers can create modular and decoupled software systems, which
        aligns to break down a system into evolvable components. This convergence enables
        developers to create maintainable, scalable, and adaptable software systems that
        effectively manage complexity. Dependency Injection is a valuable (but not the
        only or mandatory) aspect of \gls{dip}. We have observed that the claim that the
        technique of Dependency Injection solves coupling between classes in an
        application is dangerous and in some cases wrong
        \parencite[215]{mannaert_normalized_2016}. Nevertheless, this technique, when
        applied correctly (see \ref{subsubsec_dip}), the artifact have pointed out that it
        has been a great asset in creating evolvable software, especially in the aspect of
        \gls{soc}.\\
        \midrule
        \gls{dvt} & \supports &  Adhering to the \gls{dip} can indirectly support the
        \gls{dvt} theorem. By adhering to the \gls{dip}, developers can promote
        implementations that encourage modules to interact with each other only through
        well-defined interfaces or abstractions. This approach can help minimize the
        impact of data structure changes on the system but does not guarantee full
        compliance with \gls{dvt}. \\
        \midrule
        \gls{avt} & \supports & The \gls{dip} can support the \gls{avt} theorem. Both
        principles emphasize the importance of isolating actions or operations within a
        system. By adhering to \gls{dip}, developers can create modular components that
        interact through abstractions, which may contribute to achieving AvT. However, the
        alignment between \gls{dip} and \gls{avt} less strong than with \gls{soc}, and
        adhering to \gls{dip} alone will not guarantee a system that entirely complies to
        \gls{avt}. \\ 
        \midrule
        \gls{sos} & \diverges & There is a very weak alignment between \gls{dip} and
        \gls{sos}. Although Developers can create components that are less prone to side
        effects caused by a shared state, this can hardly be contributed to the \gls{dip}.
        Therefore it is stated that there is no convergence between the two principles.\\
        \bottomrule
    \end{tabular}
    \caption{Converge \gls{dip} with \ns}
    \label{tab_convergence_dip}
\end{table}
