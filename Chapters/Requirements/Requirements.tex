\chapter{Requirements} \label{chap_requirements} 

This Chapter will specify the actual requirements of the software artifacts in Section
\ref{sec_artifact_requirements}. During the Design Cycle of the artifacts we will apply
the 5 basic functional requirements. These requirements are defined by as part of a
Functional-Construction Software Transformation. They are used to study the evolvability
of software systems in general \parencite[251]{mannaert_normalized_2016}.
\textcite[254-261]{mannaert_normalized_2016} defnined them as followed.

\begin{itemize}
    \item An information system needs to be able to represent instances of
    data entities. A data entity consists of several data fields. Such a field may be a basic
    data field representing a value of a reference to another data entity.
    
    \item An information system needs to be able to execute processing actions on
    instances of data entities. A processing action consists of several consecutive processing
    tasks. Such a task may be a basic task, i.e., a unit of processing that can change
    independently, or an invocation of another processing action.
    
    \item An information system needs to be able to input or output values
    of instances of data entities through connectors.
    
    \item An existing information system representing a set of data entities, needs
    to be able to represent: 
    \begin{itemize}
        \item a new version of a data entity that corresponds to including an additional
        data field
        \item an additional data entity 
    \end{itemize}
    
    \item An existing information system providing a set of processing
    actions, needs to be able to provide:
    \begin{itemize}
        \item a new version of a processing task, whose use may be mandatory
        \item a new version of a processing action, whose use may be mandatory
        \item an additional processing task
        \item an additional processing action
    \end{itemize}
    
\end{itemize}


\section{Artifact requirements} \label{sec_artifact_requirements}

Chapter \fullref{sec_research_objectives} outlines the construction of two artifacts. Both
of these artifacts will be meticulously designed and developed in accordance with the
design philosophy and principles of \gls{ca}, ensuring strict adherence to the following
requirements:

\subsection{Component Architecture Requirements}
 
The following requirements are applied to the Component Architecture of both the Generator
Artifact and the Generated Artifact.

\begin{table}[H]
    \begin{tabular}{@{\makebox[2em][c]{\rownumber\space}}  p{0.87\linewidth}}
        \multicolumn{1}{@{\makebox[2em][c]{Nr.}}  p{0.87\linewidth}}{Requirement}\\ 
    \hline
    The solution is organized into separate Visual Studio projects for the Domain,
    Application, Infrastructure, and Presentation layers of the component. A detailed
    description of these layers can be found in Section \fullref{subsec_layers}.
    \\
    The Visual Studio projects representing the component layers comply with the naming
    conventions outlined in the appendix \fullref{appendix_component_naming_convention} \\
       
    The dependencies between the component layers must follow an inward direction towards
    the higher-level components as illustrated in Figure \ref{fig_modulair_components}
    schematically, and cannot skip layers. \\
       \hline
    \end{tabular}
\caption{The Component Architecture Requirements}
\label{table_component_requirements}
\end{table}

\begin{table}[H]
    \begin{tabular}{@{\makebox[2em][c]{\rownumber\space}}  p{0.87\linewidth}}
        \multicolumn{1}{@{\makebox[2em][c]{Nr.}}  p{0.87\linewidth}}{Technology Requirement}\\ 
    \hline
       The Domain and Application layers have no dependencies on any infrastructure technologies, like web- or database
       technologies. 
       \\
       The Presentation Layer relies on various infrastructure technologies for
       facilitating end-user interaction. Examples of such technologies include Command
       Line Interfaces (CLIs), RESTful APIs, and web-based solutions. Each dependency is
       isolated and managed in separate Visual Studio Projects to ensure the stability and
       evolvability of the system. 
       \\
       The Infrastructure Layer may rely on other infrastructure components, such
       as databases or filesystems. Each infrastructure dependency is isolated and managed
       in separate Visual Studio Projects to promote stability and evolvability. 
       \\
       All Component Layers utilize the C\# programming languages, explicitly targeting
       the .NET 7.0 framework. 
       \\
       Reusing existing functionality or technology (packages) is permitted only when
       adhering to the \gls{lsp} and utilizing the open-source package manager, \gls{nuget}.
        \\
       \hline
    \end{tabular}
\caption{Application Layer Technology Requirements}
\label{table_requirements_application_layer_technology}
\end{table}
\subsection{Software Architecture Requirements} \label{software_requirements}

Figure \ref{fig_design} illustrates the generic Software Architecture of the Artifacts.
Each instantiated element adheres to the Element Naming Convention outlined in Appendix
\ref{appendix_element_naming_convention}. In addition, the following tables detail the
requirements specific to each element.

\begin{figure}[H]
    \centering
    \includegraphics[width=1\textwidth]{figures/generic_design.pdf}
    \caption[Generic architecture]{The Generic architecture of the artifacts}
    \label{fig_design}
\end{figure}


\subsubsection*{Presentation Layer}
\begin{table}[H]
    \begin{tabular}{@{\makebox[2em][c]{\rownumber\space}}  p{0.87\linewidth}}
        \multicolumn{1}{@{\makebox[2em][c]{Nr.}}  p{0.87\linewidth}}{ViewModel Requirement}\\ 
        \hline
        The ViewModel consists of data attributes representing fields from the
        corresponding Entity. In addition, it may contain information specific to the user
        interface. \\

        The ViewModel has no external dependencies on other objects within the
        architecture. \\
       
       \hline
    \end{tabular}
\caption{ViewModel Requirements}
\label{table_requirements_viewmodel}
\end{table}

\begin{table}[H]
    \begin{tabular}{@{\makebox[2em][c]{\rownumber\space}}  p{0.87\linewidth}}
        \multicolumn{1}{@{\makebox[2em][c]{Nr.}}  p{0.87\linewidth}}{Presenter Requirement}\\ 
    \hline
    The Presenter Implementation is derived from the IPresenter interface and follows the
    specified implementation. The IPresenter interface can be found in the Application
    Layer. \\   
    
    The Presenter is responsible for creating the Controller's Response by instantiating
    the ViewModel, constructing the HTTP Response message, or combining both elements as
    needed. \\
    
    When required, the Presenter utilizes the IMapper interface without depending on
    specific implementations of the IMapper interface. \\
    
    The Presenter has an internal scope and cannot be instantiated outside of the
    Presentation layer. \\
    \hline
    \end{tabular}
\caption{Presenter Requirements}
\label{table_requirements_presenter}
\end{table}

\begin{table}[H]
    \begin{tabular}{@{\makebox[2em][c]{\rownumber\space}}  p{0.87\linewidth}}
        \multicolumn{1}{@{\makebox[2em][c]{Nr.}}  p{0.87\linewidth}}{ViewModelMapper Requirement}\\ 
    \hline
    The ViewModelMapper is derived from the IMapper interface and follows the specified
    implementation. The IMapper interface can be found in the Application Layer. \\

    The ViewModelMapper is responsible for mapping the values of the necessary data
    attributes from the ResponseModel to the ViewModel. \\
    
    The ViewModelMapper has an internal scope and cannot be instantiated outside of the
    Presentation layer. \\
    \hline
    \end{tabular}
\caption{ViewModelMapper Requirements}
\label{table_requirements_viewModelMapper}
\end{table}

\begin{table}[H]
    \begin{tabular}{@{\makebox[2em][c]{\rownumber\space}}  p{0.87\linewidth}}
        \multicolumn{1}{@{\makebox[2em][c]{Nr.}}  p{0.87\linewidth}}{Controller Requirement}\\ 
    \hline
    The Controller is responsible for receiving external requests and forwarding the
    request to the appropriate Boundary within the Application Layer. \\

    The Controller relies on the IBoundary interface without depending on specific
    implementations of the IBoundary interface. \\
    \hline
    \end{tabular}
\caption{Controller Requirements}
\label{table_requirements_controlle}
\end{table}

\subsubsection*{Application Layer}

\begin{table}[H]
    \begin{tabular}{@{\makebox[2em][c]{\rownumber\space}}  p{0.87\linewidth}}
        \multicolumn{1}{@{\makebox[2em][c]{Nr.}}  p{0.87\linewidth}}{IBoundary Requirements}\\ 
    \hline
    The IBoundary interface establishes the contract for its derived Boundary implementations. \\

    The IBoundary interface has public scope within the system. \\
    \hline
    \end{tabular}
\caption{IBoundary Requirements}
\label{table_requirements_iboundary}
\end{table}

\begin{table}[H]
    \begin{tabular}{@{\makebox[2em][c]{\rownumber\space}}  p{0.87\linewidth}}
        \multicolumn{1}{@{\makebox[2em][c]{Nr.}}  p{0.87\linewidth}}{Boundary Implementation Requirements}\\ 
    \hline
    A Boundary implementation is derived from the IBoundary interface and follows the
    specified implementation. \\

    The Boundary implementation serves as a separation between the internal aspects of the
    Application Layer and the other layers within the Component. \\
    
    Each Boundary implementation handles a single task, which is then
    executed using the IInteractor interface. \\
    
    Boundary implementations have an internal scope and cannot be instantiated outside the
    Application Layer. \\
    \hline
    \end{tabular}
\caption{Boundary Implementation Requirements}
\label{table_requirements_boundary}
\end{table}

\begin{table}[H]
    \begin{tabular}{@{\makebox[2em][c]{\rownumber\space}}  p{0.87\linewidth}}
        \multicolumn{1}{@{\makebox[2em][c]{Nr.}}  p{0.87\linewidth}}{IInteractor Requirements}\\ 
    \hline
    The IInteractor interface establishes the contract for its derived Interactor
    implementations. \\

    The IInteractor has an internal scope and cannot be implemented outside the Application
    Layer. \\
    \hline
    \end{tabular}
\caption{IInteractor Requirements}
\label{table_requirements_iinteractor}
\end{table}

\begin{table}[H]
    \begin{tabular}{@{\makebox[2em][c]{\rownumber\space}}  p{0.87\linewidth}}
        \multicolumn{1}{@{\makebox[2em][c]{Nr.}}  p{0.87\linewidth}}{Interactor Implementation Requirements}\\ 
    \hline
    An Interactor implementation is derived from the IInteractor interface and follows the
    specified implementation. \\

    The Interactor implementation executes a single task or orchestrates a series of
    tasks. Each of these tasks is implemented in separate Interactors. Alternatively, a
    Gateway is used for Tasks with Infrastructure dependencies, such as data persistence
    in a database. \\
    
    Depending on the Task, the Interactor implementation orchestrates the mapping from
    RequestModels to Entities, or from Entities to ResponseModels, utilizing the IMapper
    interface. \\
    
    Interactor implementations have an internal scope and cannot be implemented outside
    the Application Layer. \\
    \hline
    \end{tabular}
\caption{Interactor Implementation Requirements}
\label{table_requirements_interactor}
\end{table}

\begin{table}[H]
    \begin{tabular}{@{\makebox[2em][c]{\rownumber\space}}  p{0.87\linewidth}}
        \multicolumn{1}{@{\makebox[2em][c]{Nr.}}  p{0.87\linewidth}}{IMapper Requirements}\\ 
    \hline
    The IMapper interface establishes the contract for its derived Mapper implementations.
    \\

    The IMapper interface has a public scope within the system. \\
    \hline
    \end{tabular}
\caption{IMapper Requirements}
\label{table_requirements_imapper}
\end{table}

\begin{table}[H]
    \begin{tabular}{@{\makebox[2em][c]{\rownumber\space}}  p{0.87\linewidth}}
        \multicolumn{1}{@{\makebox[2em][c]{Nr.}}  p{0.87\linewidth}}{RequestModelMapper Requirement}\\ 
    \hline
    The RequestModelMapper is derived from the IMapper interface and follows the specified
    implementation. \\

    The RequestModelMapper is responsible for mapping the values of the necessary data
    attributes from the RequestModel to an Entity. \\
    
    The RequestModelMapper has an internal scope and cannot be implemented outside
    the Application Layer. \\
    \hline
    \end{tabular}
\caption{RequestModelMapper Requirements}
\label{table_requirements_requestmodelmapper}
\end{table}

\begin{table}[H]
    \begin{tabular}{@{\makebox[2em][c]{\rownumber\space}}  p{0.87\linewidth}}
        \multicolumn{1}{@{\makebox[2em][c]{Nr.}}  p{0.87\linewidth}}{ResponseModelMapper Requirement}\\ 
    \hline
    The RequestModelMapper is derived from the IMapper interface and follows the specified
    implementation. \\

    The RequestModelMapper is responsible for mapping the values of the necessary data
    attributes from the RequestModel to an Entity. \\
    
    The RequestModelMapper has an internal scope and cannot be implemented outside the
    Application Layer. \\
       \hline
    \end{tabular}
\caption{ResponseModelMapper Requirements}
\label{table_requirements_responsemodelmapper}
\end{table}

\begin{table}[H]
    \begin{tabular}{@{\makebox[2em][c]{\rownumber\space}}  p{0.87\linewidth}}
        \multicolumn{1}{@{\makebox[2em][c]{Nr.}}  p{0.87\linewidth}}{IPresenter Requirements}\\ 
    \hline
    The IPresenter interface establishes the contract for its derived Presenter
    implementations, typically implemented as part of the Presentation Layer. \\

    The IPresenter interface has a public scope within the system. \\
    \hline
    \end{tabular}
\caption{IPresenter Requirements}
\label{table_requirements_ipresenter}
\end{table}

\begin{table}[H]
    \begin{tabular}{@{\makebox[2em][c]{\rownumber\space}}  p{0.87\linewidth}}
        \multicolumn{1}{@{\makebox[2em][c]{Nr.}}  p{0.87\linewidth}}{I\textit{[\gls{verb}]}Gateway Requirements}\\ 
    \hline
    The \textit{[\gls{verb}]Gateway} interface establishes the contract for its derived Gateway
    implementations, which are typically implemented in the Infrastructure Layer. \\

    The \textit{[\gls{verb}]}Gateway interface has a public scope within the system. \\
    
    Each task is represented in the naming convention of the interface. As an example, the
    basic \gls{crud} actions result in a total of five IGateway interfaces: ICreateGateway,
    IGetGateway, IGetByIdGateway, IUpdateGateway, and IDeleteGateway. \\
    \hline
    \end{tabular}
\caption{\textit{[\gls{verb}]}Gateway Requirements}
\label{table_requirements_IGateway}
\end{table}

\begin{table}[H]
    \begin{tabular}{@{\makebox[2em][c]{\rownumber\space}}  p{0.87\linewidth}}
        \multicolumn{1}{@{\makebox[2em][c]{Nr.}}  p{0.87\linewidth}}{ResponseModel Requirement}\\ 
    \hline
    The ResponseModel consists primarily of data attributes representing the fields of the
    corresponding Entity. Additionally, the ResponseModel may contain data specific to the
    output of the Interactor. \\

    The ResponseModel does not depend on external objects within the architecture. \\
    \hline
    \end{tabular}
\caption{ResponseModel Requirements}
\label{table_requirements_responsemodel}
\end{table}

\begin{table}[H]
    \begin{tabular}{@{\makebox[2em][c]{\rownumber\space}}  p{0.87\linewidth}}
        \multicolumn{1}{@{\makebox[2em][c]{Nr.}}  p{0.87\linewidth}}{RequestModel Requirement}\\ 
    \hline
    The RequestModel consists primarily of data attributes representing the fields of the
    corresponding Entity. Additionally, the RequestModel may contain data specific to the
    input of the Interactor. \\

    The RequestModel does not depend on external objects within the architecture. \\
       
       \hline
    \end{tabular}
\caption{RequestModel Requirements}
\label{table_requirements_requestmodel}
\end{table}

\subsubsection*{Domain Layer}

\begin{table}[H]
    \begin{tabular}{@{\makebox[2em][c]{\rownumber\space}}  p{0.87\linewidth}}
        \multicolumn{1}{@{\makebox[2em][c]{Nr.}}  p{0.87\linewidth}}{Data Entity Requirement}\\ 
    \hline
    The Data Entity consists solely of attributes representing the corresponding data
    fields. \\

    The Data Entity does not rely on external objects within the architecture. \\
    
    The Application Layer is the only layer that utilizes the Data Entity. \\
    \hline
    \end{tabular}
\caption{Data Entity Requirements}
\label{table_requirements_data_entity}
\end{table}

\subsubsection*{The Infrastructure Layer}

\begin{table}[H]
    \begin{tabular}{@{\makebox[2em][c]{\rownumber\space}}  p{0.87\linewidth}}
        \multicolumn{1}{@{\makebox[2em][c]{Nr.}}
        p{0.87\linewidth}}{\textit{[\gls{verb}]}Gateway Implementation Requirement}\\ 
    \hline
    The [\textit{\gls{verb}}]Gateway Implementation derives from the I[\textit{\gls{verb}}]Gateway interface
    and adheres to the specified implementation. \\

    The [\textit{\gls{verb}}]Gateway Implementation is responsible for the interaction
    associated with the specific task, utilizing the infrastructure technology of the
    specific layer (e.g., a SQL database or a filesystem).
    
    The [\textit{\gls{verb}}]Gateway Implementation has an internal scope and cannot be
    instantiated outside of the Layer. \\
    \hline
    \end{tabular}
\caption{\textit{[\gls{verb}]}Gateway Implementation Requirements}
\label{table_requirements_gatewayimplementation}
\end{table}

\subsubsection*{Design Principles compliancy}
Each architectural pattern adheres to at least one of the SOLID principles to ensure that
none of the implementations violate these principles.





\subsection*{Design Principles compliancy}
Each architectural pattern follows at least one of the SOLID principles, ensuring that
none of the implementations violate these principles. 


    \textcolor{red}{TODO: requirement mbt code expansion toevoegen (pag 403)}