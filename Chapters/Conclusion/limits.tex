\section{Limitations}

In this research, the artifacts created to demonstrate the alignment between \gls{ca} and
\gls{ns} have shown promising results. However, it is essential to recognize these
artifacts' limitations, particularly in implementing \gls{ns} principles such as the
Separation of State Principle and the Trigger Element. These limitations must be
acknowledged to guide future work and refinement of the combined architectural design
approaches.

One of the primary limitations of the artifacts lies in their incomplete representation of
the Separation of State principle. This principle is crucial in \gls{ns} to ensure proper
handling of state changes while achieving stability and evolvability. While the artifacts
incorporate some aspects of state management, they fall short of fully implementing the
Separation of State principle as \gls{ns} prescribes. 

The other limitation of the artifacts is their lack of a comprehensive Trigger Element, an
essential element of \gls{ns}. The Trigger Element manages external triggers while
ensuring that software remains stable and evolvable. In the artifacts, incorporating the
Trigger Element is limited, primarily relying on the Controller element from \gls{ca}.
While this approach may be sufficient for web-enabled environments such as websites and
RESTful APIs, it may not be adequate for a broader range of requirements.