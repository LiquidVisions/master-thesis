\chapter{Introduction} \label{introduction}

\section{Preamble} \label{sec:preamble}

One of the constants in life is the notion that everything in life is constantly in
motion, nothing ever stays the same. Plato states that \enquote*{Patha Rhei} is one of the
famous philosophical statements first described by the Greek philosopher
\emph{Heraclitus}\footnote{\url{https://plato.stanford.edu/entries/process-philosophy/}}.
His statement unambiguously describes the dynamics of everything that exists. The
\enquote{flux}, or \emph{change} is one of the constants in life. 

According to most scholars, this most certainly also applies to modern corporate
environments where change is continuously introduced at an ever-increasing pace. These
changes lead to an evolution of requirements impacting the evolvability, stability and
quality of software artifacts like Information Systems.

The laws of software evolution refer to a series of laws that describes the balance
between the forces driving new requirements on the one hand, and the forces that slow down
progress on the other hand. In other words: changing software leads to the deterioration of
stability and evolvability, possibly with a negative effect on the desired quality of
these software systems. \citeauthor*{lehman_programs_1980}

More than a half of century of software engineering-, and architecture practices show that
the complexity of these software artifacts gradually increases over time. Eventually, this
will render most of the software artifacts obsolete, according to
\citeauthor{lehman_programs_1980}.

Historically, a lot of research has been done to solve the problem of the deterioration of
Software Artifact. McIlroy proposed a vision where the systematic reuse of software
building blocks leads to negative programming practices where software changes eventually
lead to a reduction of complexity \parencite{p_naur_nato_1968}. 

Dijkstra argued against the use of unstructured control flow in programming and advocated
for the use of structured programming constructs to improve the clarity and
maintainability of code \parencite{dijkstra_letters_1968}. Later in his career, he
advocated structured programming techniques that improved the modularity and evolvability
of software artifacts.

Parnas continued with the principle of information hiding. He stated that design decisions
that are used multiple times by a software artifact should be modularized in order to
reduce complexity \parencite{parnas_criteria_1972}. 

Over the last decades, several programming paradigms had had a significant effect on the
modularity, stability and evolvability of software systems. Procedural programming was one
of the first paradigms to emerge that supported modularity in its constructs. This was
then followed by object-oriented introducing the support for encapsulation and
polymorphism. functional programming. All of which had a significant impact on modern
programming languages like Java and C\#. The constructs of these programming languages
significantly enhanced the capabilities of modular and evolvable software architectures. 

\section{Research Problem: The plethora of proposed solutions}
\label{sec:research_problem}

There is a wide variety of proposed solutions available for the challenges that occur in
modular and evolvable software architecture. There are plentiful experiences documented
throughout professional and personal blog posts on the internet. Many of those experiences
have mixed outcomes, some are opinionated and results are sometimes based on improper
interpretations of the proposed solutions.

Deciding on the best fit for one of the solutions is a recurring and often difficult task
for software architects. A popular and widely accepted solution from software engineering
literature is Clean Architecture. There is a wide supporting community and many corporate
solutions move toward architectures that have similarities with the Clean Architecture
approach. 

An architecture that derives from science and empirical evidence is Normalized Systems
\parencite*{mannaert_normalized_2009,mannaert_normalized_2016}. Deciding between the two
approaches can be a challenging task on which there is very little documentation and
research. Could it even be the case that combining the two approaches can be an approach
that leads to a highly modular and evolvable software artifact? Let's start with a small
introduction of both approaches.

\subsection{On Normalized Systems Theorems} \label{subsec:intro_to_ns}

The Normalized Systems theorems are a scientific approach to creating software systems
based on the laws for software evolvability. These theorems have resulted in a documented
track record of achieving software stability, in a scientific environment. Effectively, it
prevents the accumulation of combinatorial effects on anticipated change drivers. This
prevents the positive feedback loop and prevents the degradation of the software artifact.
Preventing positive feedback loops has a positive effect on the evolvability of software
artifacts.\parencite[]{mannaert_normalized_2009}. 

\citeauthor[]{mannaert_normalized_2009} have formulated the theorem of Normalized Systems
as prescriptive structures (elements) that lead to a modular architecture with low
coupling and high cohesion. The resulting software architecture will be designed to cope
with future change \parencites[]{mannaert_normalized_2009}.

\subsection{On Clean Architecture} \label{subsec:into_to_ca}

Clean Architecture is the accumulation of more than half a century of coding, designing,
and architecting software systems by \citeauthor*[]{martin_clean_2018}. He published his
experience in his book \citetitle*[]{martin_clean_2018} in \citeyear[]{martin_clean_2018}.
In this book, he states that creating a software artifact does not require that much skill
and knowledge. However, creating stable and evolvable software artifacts is a skill that
requires a lot of knowledge, skill, dedication, and time.

The book's goal is to have a software architecture that minimizes the human resources
required to build and maintain the information system. Just like Normalized Systems, it
has a prescribed design of software classes that will lead to a modular architecture with
low coupling and high cohesion \parencite{martin_clean_2018}.
\section{Introducing Clean Architecture}

Clean Architecture is the accumulation of more than half a century of coding, designing,
and architecting software systems by Martin, Robert C. He published experience in his book
\citetitle*[]{martin_clean_2018} in \citeyear[]{martin_clean_2018}. He states that
creating a programming artifact does not require much skill and knowledge. However, creating
stable and evolvable software artifacts is a different skill that requires
a lot of knowledge, skill, dedication, and time.

The book's goal is to have a software architecture that minimizes the human resources
required to build and maintain the information system. Just like Normalized Systems, it
has a prescribed design of software classes that will lead to a modular architecture with
low coupling and high cohesion \parencite{martin_clean_2018}.

\section{Introducing Normalized Systems Theorems}

The Normalized Systems theorems are a scientific approach to creating software systems
based on the laws for software evolvability. These theorems have resulted in a documented
track record of achieving software stability, in a scientific environment. Effectively, it
prevents the accumulation of combinatorial effects on anticipated change drivers. This
prevents the positive feedback loop and prevents the degradation of the software artifact.
Preventing positive feedback loops has a positive effect on the evolvability of software
artifacts.\parencite[]{mannaert_normalized_2009}. 

\citeauthor[]{mannaert_normalized_2009} have formulated the theorem of Normalized Systems
as prescriptive structures (elements) that lead to a modular architecture with low
coupling and high cohesion. The resulting software architecture will be designed to cope
with future change \parencites[]{mannaert_normalized_2009}.
\section{Hypothesised outcome} \label{hypothesis} 

The proposed hypothesis is that the \gls{ca} approach can be applied in conjunction with
the \gls{ns} theorems. The design principles of \gls{ca} converge with the theorems of
\gls{ns}. Consequently, the artifact that is part of this design research will lead to a
highly modular, stable, and evolvable C\# artifact that does not contradict Normalized
Systems theorems.

Both architectural approaches formulate modular structures independent of any programming
technology \parencite{mannaert_normalized_2009,robert_c_martin_clean_2018}. As such, the
C\# artifact produced as part of this research has similar trademarks of modularity,
evolvability, and stability compared to case studies where Java SE has been used
\parencite{oorts_building_2014, de_bruyn_enabling_2018}. Furthermore, the applicability of
\gls{ca} has no additional or negative effect when used in conjunction with the
\gls{ns} Theorems.

\begin{figure}[H]
    \centering
    \includegraphics[width=0.8\textwidth]{Figures/hypothesis.pdf}
    \caption[The hypothesis]{The hypothesis}
    \label{fig_hypothesis}
\end{figure}
\section{Research Objectives} \label{sec_research_objectives}

In this Design Science research, the focus shifts from a research question toward
research objectives. The following objectives apply to this research.

\subsection*{Objective 1: The expander artifact}
A C\# artifact that consists of a source code expander and some helper classes to
generate the second \emph{expanded} artifact. The artifact supports the harvesting and
rejuvenation of custom code snippets on the expanded artifact so that regenerations do not
have a loss of implementations on the expanded artifact. The expander artifact is entirely
based on the design principles of \gls{ca}.

Chapter \ref{sec_generator_artifact} entirely describes the expander artifact.

\subsection*{Objective 2: The expanded artifact}
This artifact is an entirely working restful API, that is based on ASP.NET and C\#. It has
essential CRUD support. The artifact has been generated by the \emph{expander} and has
been enriched with custom code snippets to comply with the initial requirements.
The expanded artifact is entirely based on the design principles of \gls{ca}.

Chapter \ref{sec_generator_artifact} entirely describes the expander artifact.


\section{Research question} \label{sec:research_questions}
In order to test the hypothesis described in \ref{hypothesis} the following research
questions can be applied to both research objectives.

\subsubsection*{Research Question 1:}
What violations of the Normalized Systems Theorems can be found as an effect of
following the Clean Architecture principles on both the expander- and expanded artifact?

\subsubsection*{Research Question 2:}
Which of the Normalized Systems Theorems are excluded as an effect of following the
Clean Architecture principles on both the expander- and expanded artifact?

When no violations or excludes on the Normalized Systems Principles are found in the
artifacts, it is assumed that there will be no combinatorial ripple effect on changing
both artifacts. Only in that case, it can be concluded that Clean Architecture can be
used in conjunction with the Normalized Systems theorems. 
\section{Research method} \label{sec:research_method}

This research is a Design Science Method and relies on the Engineering Cycles as described
by Wieringa. The engineering cycle provides a structured approach to develop the
required artifacts to analyze the design problem \parencite{wieringa_design_2014}.

\begin{figure}[H]
    \centering
    \includegraphics[width=1\textwidth]{Figures/engineering_cycle.pdf}
    \caption[Engineering cycle]{Wieringa's engineering cycle}
    \label{fig:engineering_cycle}
\end{figure}

In the context of this research, the artifacts described in chapters
\ref{sec:generator_artifact} and \ref{sec:generated_artifact} are considered to be
information systems. \citeauthor{hevner_design_nodate} proposed a framework for research
in information systems by introducing the interacting relevance and rigor cycles.

Figure \ref{fig:dsr} depicts a specialized overview of Hevners Design Science Framework.
The rigor cycle is composed of the theories and knowledge from Normalized Systems
and Clean architecture. This is supplemented by the rigorous knowledge of modularity,
evolvability, and stability of software systems. The Relevance cycle represents the
business needs of the stakeholders. The business needs are described as research
objectives, research questions, and research requirements.

\begin{figure}[H]
    \centering
    \includegraphics[width=1\textwidth]{Figures/rigor_relevance_cycle.pdf}
    \caption[DSF]{The Design Science Framework for IS Research}
    \label{fig:dsr}
\end{figure}