\subsubsection{Action version Transparancy}
\gls{avt} is the property of a system to modify existing processing functions without
affecting the existing ones. It should be possible to upgrade a function without affecting
the callers of those functions. This description leads to the following theorem
\parencite[282]{mannaert_normalized_2016}.

\mycolorbox{A processing function that is called by another processing function, needs to exhibit version transparency in order to achieve stability.}{Theorem III}

Most of the modern technology environments support some form of \gls{avt}. Polymorphism is
a widely used technique in order to support this theorem. Specifically, parametric
polymorphism \footnote{\url{https://en.wikipedia.org/wiki/Parametric_polymorphism}} allows
for a processing function to have multiple input parameters. There are also quite some
design patterns supporting this theorem. Some random examples are the state pattern
\footnote{\url{https://en.wikipedia.org/wiki/State_pattern}}, facade pattern
\footnote{\url{https://en.wikipedia.org/wiki/Facade_pattern}} and observer pattern
\footnote{\url{https://en.wikipedia.org/wiki/Observer_pattern}}.