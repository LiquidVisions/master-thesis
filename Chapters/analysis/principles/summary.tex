\subsection{The Principles Convergence Overview}

By cross-referencing these principles, we aim to uncover the degree of convergence between
\gls{ca} and \gls{ns}. Our observations drawn from this comparative analysis provide a
nuanced understanding of how these principles interact and highlight areas of strong
convergence as well as potential gaps. Table \ref{tab_convergence_principles_summarized}
offers a brief overview of this convergence.

\begin{table}[H]
\renewcommand{\arraystretch}{1.5}
\centering
\begin{tabular}{r|llll}

    \textbf{\acrlong{ca}   } \textbf{   \rotatebox[origin=l]{90}{\acrlong{ns}}} & 
    \rotatebox[origin=l]{90}{\acrlong{soc}} & \rotatebox[origin=l]{90}{\acrlong{dvt}} &
    \rotatebox[origin=l]{90}{\acrlong{avt}} & \rotatebox[origin=l]{90}{\acrlong{sos}} \\
\midrule


\acrlong{srp} & \fullConvergence & \npartialConvergence & \npartialConvergence & \noConvergence \\
\acrlong{ocp} & \fullConvergence & \noConvergence & \fullConvergence & \noConvergence \\
\acrlong{lsp} & \fullConvergence & \noConvergence & \npartialConvergence & \noConvergence \\
\acrlong{isp} & \fullConvergence & \noConvergence & \npartialConvergence & \noConvergence \\
\acrlong{dip} & \fullConvergence & \noConvergence & \npartialConvergence & \noConvergence \\
\bottomrule
\end{tabular}
\caption{An overview of the convergence of all \gls{ca} and \gls{ns} principles}
\label{tab_convergence_principles_summarized}
\end{table}

