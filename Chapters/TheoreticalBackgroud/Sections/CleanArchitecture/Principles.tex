\subsection{The Design principles} \label{subsec:design_principles}

\Citeauthor{robert_c_martin_clean_2018} argues that, without a solid (pun intended) set of
design principles a design and/or architecture can quickly turn into a well-intended mess
of bricks and building blocks. This is where the SOLID design principles come into place.

SOLID is an acronym for \gls{solid}. The \gls{solid} principles guide the developer on how
to arrange the architecture of the software system. It can be considered a set set of
rules on how to arrange data structures and functions into classes
\parencite[78]{robert_c_martin_clean_2018}.

The upcoming sections will provide a brief overview of each of the SOLID principles, as
there is a plethora of literature on this subject. In chapter
\ref{sec:artifact_requirements} we learn that one of the requirements is to design the
artifact solely based on the philosophy of \gls{ca}. As such, each principle's description
will be accompanied by one or more manifestation examples in the artifact, aligning with
the research objectives.