\section{The Theoretical background of Normalized Systems} 
\label{ns_theory}
\begin{itemize}
    \item Inleiding
\end{itemize}


\subsection{Design Theorems of Normalized Systems} \label{subsec:ns_desing_theorems}

The theorems of Normalized Systems described in the following paragraphs are all
considered to be scientifically proven theorems \parencite{mannaert_normalized_2016}. The
authors of the book claim that the theories, when implemented correctly increase
the stability, and therefore the evolvability of software systems. The theorems are proven
mathematically followed by simple examples of describing them by \textit{reductio ad absurdum} 
\footnote{\url{https://en.wikipedia.org/wiki/Reductio_ad_absurdum}}.

\subsubsection{Separation of Concerns}
Since the early years of software engineering, 'Separation of Concerns (SoC)' has been one
of the most fundamental software engineering principles. According to Wikipedia
\footnote{\url{https://en.wikipedia.org/wiki/Separation_of_concerns}} the principle has
first been mentioned by \citeauthor{dijkstra_selected_1982} as the crucial principle to
design modular software architecture \parencite[]{dijkstra_selected_1982}. The concept
itself was introduced by \citeauthor{broy_criteria_1972} in his book
\citetitle*{broy_criteria_1972}.

According to \citeauthor{dijkstra_selected_1982}, this principle promotes the idea that a
program should be divided into distinct sections, each addressing a separate concern or
aspect of the problem. This allows for a more organized and maintainable program, as
changes to one concern do not affect the others. Dijkstra also emphasized that the
separation of concerns should be applied at the level of individual modules or functions,
rather than at the level of the entire program.

The SoC had its effect on later versions of software engineering principles like SOLID.
The principles of 'Single Responsibility' and 'Interface Segregation' are directly derived
from Soc. It also affected the theorems of Normalized Systems, although it has a more
strict definition of this principle. In the book of \citeauthor{mannaert_normalized_2016}
it is described as followed: 

\textit{A processing function can only contain a single task to achieve stability.}

There are a couple of manifestation examples described like the application of
an integration service bus. The manifestation of external workflows is another example.
One example of an external workflow in the prototype of this research project is the
\citetitle*{noauthor_seedingboundary_nodate}. \todo{add hyperlink}This class is responsible for
the provision of the initial data to the model, based on the elements of the meta-model
itself. It 'seeds' the data by orchestrating individual entity seeders to execute in the
correct order. Another example is the segregation of the \citetitle*{noauthor_entity_nodate}
where each CRUD operation is separated in its interface.\todo{add
hyperlink} \todo{Question: should I add examples of the prototype in the chapter on the
theoretical background?}

\subsubsection{Data version transparancy}
Data Version Transparency (DvT) is the act of encapsulation of data entities for specific
tasks at hand. This results in the fact that data structures can have multiple versions
often mentioned as Data Transfer Objects in modern software engineering projects. In other
words, it should be possible to update the data entity without affecting the processing
functions \parencite{mannaert_normalized_2016}. This leads to the following description of
the theorem:

\textit{A data structure that is passed through the interface of a processing
function needs to exhibit version transparency in order to achieve stability.}

DvT is widely used in various technological applications. Every web service currently known
supports some type of versioning. In restful APIs for example, it is common practice to
support versioning over the URI. It is best practice to encapsulate breaking changes in a
new version of the endpoint/service so that the consumers are not (directly) affected by
the change. In modern Object Oriented languages, DvT is also supported by the ability to
determine the scope of visibility of the modifiers of the various programming constructs
like fields, properties, interfaces and classes. Also known as information hiding
\parencite{broy_criteria_1972,mannaert_normalized_2016}.

The prototype also uses information hiding very strictly. In order to seal implementations
to the intended layers, concrete implementations always have internal visibility, making
them impossible to use. The interfaces on the other hand are publically exposed. The
dependent layers are now restricted to the appointed dependency injection container of the
specific layer. Alternatively, it is also possible to implement a custom implementation of
the specific interface. 

Another example from the prototype is the use of ViewModels for the queries, and
CommandModels for the commands. Depending on the context of the operation, the containing
fields of the Model may differ. A CommandModel for deleting a data entity will only
contain the key of the entity that needs to be deleted. The CommandModel for creating the
same type of entity will probably contain all the required fields of the data entities,
except the key field as this key is often auto-generated by the database or
domain layer of the application. \todo{Question: should I add examples of the prototype in the chapter on the
theoretical background?}

\subsubsection{Action version transparancy}
\begin{itemize}
    \item Toelichting op action version transparency
\end{itemize}

\subsubsection{Separation of state}
\begin{itemize}
    \item toelichting op Separation of state.
\end{itemize}

\subsection{Construction elements}
\begin{itemize}
    \item Data elements
    \item task elements
    \item connector elements
    \item flow elements
    \item etc\dots
    \item In het hoofdstuk evaluation wordt
    besproken in hoeverre de CA construction elements convergeren met die van NS
\end{itemize}
