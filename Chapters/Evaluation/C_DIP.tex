\subsection{The convergence of the Interface Segregation Principle}

\begin{table}[H]
    \begin{tabular}{ l | c | p{0.78\linewidth}}
        \toprule
        \gls{soc} & \converges & This principle converges with the \gls{soc} theorem.
        \gls{dip} states that high-level modules should not depend on low-level modules.
        When unavoidable, high-level modules should depend on abstractions of low-level
        modules and abstractions should not depend on details. By adhering to \gls{dip}
        correctly, developers can create modular and decoupled software systems, which
        aligns to break down a system into evolvable components. This convergence enables
        developers to create maintainable, scalable, and adaptable software systems that
        effectively manage complexity. Dependency Injection is a valuable (but not the
        only or mandatory) aspect of \gls{dip}. We have observed that the claim that the
        technique of Dependency Injection solves coupling between classes in an
        application is dangerous and in some cases wrong
        \parencite[215]{mannaert_normalized_2016}. Nevertheless, this technique, when
        applied correctly (see ref{subsubsec:dip}), the artifact have pointed out that it
        has been a great asset in creating evolvable software, especially in the aspect of
        \gls{soc}.\\
        \midrule
        \gls{dvt} & \supports &  Adhering to the \gls{dip} can indirectly support the
        \gls{dvt} theorem. By adhering to the \gls{dip}, developers can promote
        implementations that encourage modules to interact with each other only through
        well-defined interfaces or abstractions. This approach can help minimize the
        impact of data structure changes on the system but does not guarantee full
        compliance with \gls{dvt}. \\
        \midrule
        \gls{avt} & \supports & The \gls{dip} can support the \gls{avt} theorem. Both
        principles emphasize the importance of isolating actions or operations within a
        system. By adhering to \gls{dip}, developers can create modular components that
        interact through abstractions, which may contribute to achieving AvT. However, the
        alignment between \gls{dip} and \gls{avt} less strong than with \gls{soc}, and
        adhering to \gls{dip} alone will not guarantee a system that entirely complies to
        \gls{avt}. \\ 
        \midrule
        \gls{sos} & \diverges & There is a very weak alignment between \gls{dip} and
        \gls{sos}. Although Developers can create components that are less prone to side
        effects caused by a shared state, this can hardly be contributed to the \gls{dip}.
        Therefore it is stated that there is no convergence between the two principles.\\
        \bottomrule
    \end{tabular}
    \caption{Convergence \gls{dip} with \gls{ns}}
    \label{tab:convergence_dip}
\end{table}