\subsection{The Design Elements} \label{subsec_design_elements}

In the context of \gls{ns} approach, the goal is to design a software system that is highly
modular, maintainable and testable. The accumulation of the Desing principles discussed
in chapter \ref{subsec_design_principles} leads to the following generalization of the
architecture. Each of the following elements has a crucial role to achieve the
design goals.

\begin{table}[H]
    \begin{tabular}{ p{0.15\linewidth} p{0.75\linewidth}}
        \hline
        \textbf{Element} & \textbf{Description} \\ 
        \hline
        Entity & Entities are the application's core business objects, representing the
        domain's fundamental concepts and rules. They encapsulate the data and behavior
        essential to the application's functionality.\\ \midrule

        Interactor & Interactors, also known as Use cases, encapsulate the application's
        business logic and represent specific actions that the system can perform. They
        are responsible for coordinating the work of other components and ensuring that
        the system behaves correctly.\\ \midrule

        RequestModel & RequestModels are used to represent the data required by a specific
        interactor. They provide a clear and concise representation of the data required
        by the Use Case, making it easier to manage and modify the application.\\ \midrule

        ViewModel & ViewModels are part of the presentation layer and are responsible for
        managing the state of the user interface. They receive data from the Presenters
        and update the user interface accordingly. They are also responsible for handling
        user input and sending it to the Controllers for processing.\\ \midrule

        Controller & Controllers are responsible for handling requests from the user
        interface and routing them to the appropriate Interactor. They are typically part
        of the user interface layer and are responsible for coordinating the work of other
        components.\\ \midrule

        Presenter & Presenters are responsible for formatting and presenting data to the
        user interface. They receive data from the Interactor and convert it into a format
        readily displayed to the user. They are also responsible for handling user input
        and sending it back to the Interactor for processing.\\ \midrule

        Gateway & A Gateway provides an abstraction layer between the application and its
        external dependencies, such as databases, web services, or other systems. They
        allow the system to be decoupled from its external dependencies and can be easily
        replaced or adapted.\\ \midrule

        Boundary & A Boundary refers to an interface or abstraction that separates
        different layers or components of a system. The purpose of these boundaries is to
        promote modularity, evolvability, and testability by enforcing the separation of
        concerns, allowing each layer to evolve independently.\\ \midrule

        \bottomrule
    \end{tabular}
    \caption{The Elements proposed by Clean Architecture}
    \label{ca_element}
\end{table}