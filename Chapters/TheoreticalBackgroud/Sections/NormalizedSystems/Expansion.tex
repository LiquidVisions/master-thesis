\subsection{Expansion and code generation} \label{subsec:expansion}

In the context of \gls{ns}, code expansion refers to the process of adapting and extending
software components to accommodate new requirements, features, or modifications while
maintaining the software's overall stability and evolvability. This concept aims to
address the inherent complexity of software systems and mitigate the adverse effects of
change propagation and unintended consequences that often arise when modifying software
components.

Creating and maintaining a stable and evolvable system is particularly difficult and
tedious engineering job. The engineer in question needs to be aware of all the \gls{ns}
Theorems \parencite*{mannaert_normalized_2016}[273-290], Normalized Elements
\parencite*{mannaert_normalized_2016}[363-398] and recurring implementations
\parencite*{mannaert_normalized_2016}[232].

Therefore, Code expansion also embraces the concept of code generation, which involves the
automatic creation of software components based on predefined templates, patterns, and
rules. This approach streamlines the process of adapting and extending software components
to accommodate new requirements, features, or modifications while maintaining the
software's overall stability and evolvability.