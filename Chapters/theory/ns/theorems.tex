\subsection{The Design Theorems} \label{subsec_ns_desing_theorems}

In the following table we will describe the Design Theorems of \gls{ns}, firstly presented
by \textcite[111-119]{mannaert_normalized_2009}. They are known as \gls{soc}, \gls{dvt},
\gls{avt} and \gls{sos}.

\begin{table}[H]
    \begin{tabular}{ p{0.15\linewidth} p{0.75\linewidth}}
        \hline
        \textbf{Principle} & \textbf{Definition} \\ 
        \hline
        \acrshort*{soc} & The latest definition of \gls{soc} has been defined by
        \textcite[274]{mannaert_normalized_2016} as: A processing function can only contain a single task to achieve
        stability. \\
        
        \acrshort{dvt} &  A data structure that is passed through the interface of a processing function needs to
        exhibit version transparency to achieve stability.\\
        
        \acrshort{avt} & A processing function that is called by another processing function, needs to exhibit version
        transparency to achieve stability.\\
        
        \acrshort{sos} & Calling a processing function within another processing function, needs to exhibit state
        keeping to achieve stability.\\
        
        \bottomrule
    \end{tabular}
    \caption{The Design Theorems of Normalized Systems.}
    \label{ns_principles}
\end{table}