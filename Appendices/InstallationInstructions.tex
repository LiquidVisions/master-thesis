\chapter{Installing \& using Pantha Rhei} \label{appendix_installation_instructions} 

\section{Installation instructions} \label{appendix_installation_prerequisits}
\subsection*{Step 1: Create output folder}
Create an output folder so that the applications...
\begin{itemize}
    \item ...has a location where to find the required expanders.
    \item ...has a location where the log files are stored.
    \item ...has a location where the result of the generation processes can be stored.
\end{itemize}

The location of the output folder is irrelevant.

\subsection*{Step 2: Create the Nuget configuration file}
Add a configuration file named \emph{nuget.config} file to the output folder with the
the following content: 

\lstinputlisting[ caption={The content of the Nuget configuration file},
label={list_nugetconfig}] {Snippets/nuget.config}

This config file is needed for the following step where the Pantha Rhei application is
installed. The file contains the information to the private feed where the Pantha Rhei
application can be downloaded and installed.

\subsection*{Step 3: Installing Pantha Rhei} \label{appendix_installing_pantha_rhei} 
Open a console in on the location where the Nuget configuration file is stored. The
following command will download the package, and start the installation process which is
executed in the background. 

\lstinputlisting[
    caption={The install command},
    label={list_install-command}]
    {Snippets/dotnet-install-command.txt}

\subsection*{Step 4: Download \& Install the Expanders}
By clicking on the following link, an archived folder will be presented as a download by
your browser. Download the archived folder and extract it on completion. Store the
extracted folder, in a subfolder called \emph{Expanders} in the root of the output
folder. By doing so, the following folder structure should be available:

\begin{forest}
    for tree={
      font=\ttfamily,
      grow'=0,
      child anchor=west,
      parent anchor=south,
      anchor=west,
      calign=first,
      inner xsep=7pt,
      edge path={
        \noexpand\path [draw, \forestoption{edge}]
        (!u.south west) +(7.5pt,0) |- (.child anchor) pic {folder} \forestoption{edge label};
      },
      % style for your file node 
      file/.style={edge path={\noexpand\path [draw, \forestoption{edge}]
        (!u.south west) +(7.5pt,0) |- (.child anchor) \forestoption{edge label};},
        inner xsep=2pt,font=\small\ttfamily
                   },
      before typesetting nodes={
        if n=1
          {insert before={[,phantom]}}
          {}
      },
      fit=band,
      before computing xy={l=15pt},
    }  
    [PanthaRhei.Output
    [Expanders
        [.Templates]
    ]
    [nuget.config, file]
  ]
\end{forest}

The Pantha Rhei application is now ready for use.

\subsection*{Step 5: Setup a SQL database}
Currently, Pantha Rhei is working with an MS SQL database for storing the model of the
applications. Set up a SQL Database. This can either be a licensed version of SQL Server,
the free-to-use SQL Express or an Azure SQL instance. See
\url{https://www.microsoft.com/en/sql-server/sql-server-downloads} for more information.
Make sure to have a valid connection string to the SQL server instance that is needed in
step \fullref{appendix_run_flux}.

\subsection*{Step 6: Execute the command} \label{appendix_run_flux}

Pantha Rhei is used by executing the \emph{flux} command with the parameters as described
in table \ref{tab_commandline_parameters}

\lstinputlisting[
    caption={Example command executing Pantha Rhei},
    label={list_flux}]
    {Snippets/flux.txt}

\begin{table}[H]
    \begin{tabular}{ l | p{0.78\linewidth}}
        \toprule
        --root & A mandatory parameter that should contain the full path to the output
        directory \fullref{appendix_installation_instructions}. \\
        --db & A mandatory parameter that contains the connectionstring to the database. \\
        --app & A mandatory parameter indicating the unique identifier of the application that should be generated. \\
        --mode & An optional parameter that determines if a handler should be executed.
        \emph{Default} is the default fallback mode (see \ref{tab_generation_modes}). \\
        --reseed & An optional parameter that bypasses the expanding process. The model will
        be thoroughly cleaned, and reseeded based on the entities of the expander
        artifact. This enables to a certain extent the meta-circularity and enables the
        expander artifact to generate itself. \\
        \bottomrule
    \end{tabular}
    \caption{The \emph{flux} command line parameters}
    \label{tab_commandline_parameters}
\end{table}



RunModes are available to isolate the execution of the ExpanderHandler. It
requires a current implementation shown in listing \ref{list_runmode_example}. The
following RunModes are available.

\begin{table}[H]
    \begin{tabular}{ l | p{0.78\linewidth}}
        \toprule
        Default & This is the default generation mode that executes all configured
        handlers of the CleanArchitectureExpander. This will also install the required
        Visual Studio templates which are needed for scaffolding the Solution and C\#
        Project files. Furthermore, it also executes the Harvest and Rejuvenation
        handlers. This mode will clean up the entire output folder prior after the
        Harvesting process is finished prior to the execution of the handlers. \\
        
        Extend & This mode will skip the installation of the Visual Studio templates and
        the project scaffolding. It will not clean up the output folder but will
        overwrite any files handled. This mode is often less time-consuming and can be
        used in scenarios to quickly check the result of a part of the generation process. \\

       Deploy & An optional mode that allows for expander handlers to run deployments in
       isolation. For example, when a developer wants to deploy the output to an Azure App
       Service. \\
       
       Migrate & An optional mode that allows for expander handlers to run migrations in
       isolation. For example, this  currently updates the database schema by running the
       Entity Framework Commandline Interface (see \url{https://learn.microsoft.com/en-us/ef/core/cli/dotnet}).\\
        \bottomrule
    \end{tabular}
    \caption{The available \emph{Generation modes}}
    \label{tab_generation_modes}
\end{table}

\lstinputlisting[
    caption={Example on how an expander handler can adhere to the RunMode parameters},
    label={list_runmode_example}]
    {Snippets/RunModeExample.cs}


