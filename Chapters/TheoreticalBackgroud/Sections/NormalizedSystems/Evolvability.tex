\subsection{Towards evolvability} \label{sec_on_evolvability}

In \gls{ns}, evolvability is a crucial property to achieve stable software
systems. An evolvable system can adapt over time in response to changing requirements.
\gls{ns} attempts to achieve evolvability by providing guidelines, principles, and theorems
n order to achieve a modular (and scalable) architecture that allows for easy
adaptability, extensibility, and the replacement of components with a minimum impact on the
quality of the functionality and the overall structures of the architecture. This is
achieved through the use of formalized models that define the system's components,
interfaces, and behavior, as well as through the separation of concerns between different
parts of the system.

There are several aspects concerning the evolvability of software systems. One of which is
the modularity of the architecture. There is also a broad consensus about two fundamental
rules when thinking of- and designing modularity: \emph{high cohesion} and \emph{low
coupling} \autocite[22]{mannaert_normalized_2016}.