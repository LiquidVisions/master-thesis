\subsection{The Theoretical Framework} \label{subsec_ns_desing_theorems}

\gls{ns} consists of a theoretical framework describing a set of design principles. These
principles are the basis for achieving the concepts of stability, evolvability, and
modularity. \gls{ns} provides a rigorous and mathematical foundation for these theorems
and they offer guidelines for designing and developing software systems. In the following
sections, we will focus on the principles of \gls{ns} very briefly as they have been
extensively described in various scientific papers.

We know from Chapter \ref{sec_artifact_requirements} that the design artifacts, as a part
of this research, are implemented based on the \gls{ca} principles. Therefore, contrary to
Chapter \ref{sec_ca_theory}, there will be no references to the manifestations of the NS
design theorems in the design.

\subsubsection{Separation of Concerns} \label{subsubsec_soc}

\gls{soc} as a principle has first been mentioned by
\citeauthor{dijkstra_selected_1982}\footnote{\url{https://en.wikipedia.org/wiki/Separation_of_concerns}}
as the crucial principle to design modular software architecture
\parencite[]{dijkstra_selected_1982}. \gls{soc} promotes the idea that a program should be
divided into distinct sections, each addressing a particular concern or aspect of a design
problem. This allows for a more organized and maintainable source code. When implemented
correctly, a change to one concern does not affect the others. \gls{soc} should be applied
at the level of individual modules, rather than the level of an entire program.

\gls{soc} has been adopted as one of the design theorems of \gls{ns}, although it has a
stricter definition of this principle\parencite{mannaert_normalized_2016}.

\mycolorbox{A processing function can only contain a single task to achieve stability.}{Theorem I}
\subsubsection{Data version Transparency}

\gls{dvt} is the act of encapsulation of data entities for specific tasks at hand. This
results in the fact that data structures can have multiple versions often mentioned as
Data Transfer Objects in modern software engineering projects. In other words, it should
be possible to update the data entity without affecting the processing functions. This
leads to the following description of the theorem \parencite[280]{mannaert_normalized_2016}.

\mycolorbox{A data structure that is passed through the interface of a processing function
needs to exhibit version transparency to achieve stability.}{Theorem II}

\gls{dvt} is widely used in various technological applications. practically every web
service currently known supports some type of versioning. In restful APIs, for example, it
is common practice to support versioning over the URI. It is considered a best practice to
encapsulate breaking changes in a new version of the endpoint/service so that the
consumers are not (directly) affected by the change. In modern Object Oriented languages,
gls{dtv} is also supported by the ability to determine the scope of visibility of the
modifiers of the various programming constructs like fields, properties, interfaces, and
classes, also known as information hiding
\parencites{parnas_criteria_1972}[278]{mannaert_normalized_2016}.
\subsubsection{Action version Transparency}
\gls{avt} is the property of a system to modify existing processing functions without
affecting the existing ones. It should be possible to upgrade a function without affecting
the callers of those functions. This description leads to the following theorem
\parencite[282]{mannaert_normalized_2016}.

\mycolorbox{A processing function that is called by another processing function, needs to exhibit version transparency to achieve stability.}{Theorem III}

Most of the modern technology environments support some form of \gls{avt}. Polymorphism is
a widely used technique to support this theorem. Specifically, parametric
polymorphism \footnote{\url{https://en.wikipedia.org/wiki/Parametric_polymorphism}} allows
for a processing function to have multiple input parameters. There are also quite some
design patterns supporting this theorem. Some random examples are the state pattern
\footnote{\url{https://en.wikipedia.org/wiki/State_pattern}}, facade pattern
\footnote{\url{https://en.wikipedia.org/wiki/Facade_pattern}} and observer pattern
\footnote{\url{https://en.wikipedia.org/wiki/Observer_pattern}}.
\subsubsection{Separation of State}

\gls{sos} is a theorem that is based on the idea that processing functions should not
contain any state information but instead should rely on external data structures to store
state information. By separating state information from processing functions, Normalized
Systems can achieve a higher level of flexibility and adaptability. External data
structures can be updated or replaced without affecting the processing functions
themselves, which significantly reduces the change of unwanted ripple effects. This theorem is
described as followed: \parencite[258]{mannaert_normalized_2016}.

\mycolorbox{Calling a processing function within another processing function, needs to exhibit state keeping to achieve stability.}{Theorem IV}