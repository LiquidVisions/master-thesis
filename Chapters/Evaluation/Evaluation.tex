\chapter{Evaluation results} \label{chap_evaluation}

This Chapter focuses on the comparison of \gls{ca} and \gls{ns} and explores the
convergence between the two architectural design approaches. We will start by comparing
the principles of \gls{ca} with the theories of \gls{ns} and analyze how they converge in
their approach and the effect on the software design. In section
\ref{sec_converging_elements}, we will compare the design elements of the two
architectural approaches. We will highlight the similarities and differences in their
implementation, in order to determine the further convergence of the two discussed
architectural design approaches.

To conduct this comparison, we used a combination of artifacts and a literature study. We
analyzed real-world examples of software artifacts that were designed using \gls{ca}, and
reviewed the relevant literature to gain a comprehensive understanding of both approaches.

Throughout this evaluation chapter, we will explore the strengths and weaknesses of each
approach, as well as the potential benefits of their convergence. By examining the
principles and design elements of \gls{ca} and \gls{ns}, we hope to provide a clear
understanding of the effects of using the different approaches in conjunction with each
other.

\section{Converging principles} \label{sec_converging_principles}

In this section, we will apply a systematic cross-referencing approach to assess the level
of convergence between each of the \gls{solid} principles of \gls{ca} and the \gls{ns}
theories. Along with a brief explanation, the level of convergence is denoted as followed:

\begin{table}[H]
    \begin{tabular}{ l l p{0.57\linewidth}} Fully converges & \conv & This indicates
        a high degree of alignment between the respective \gls{solid} principle and
        \gls{ns} theorem. The application of either principle or theorem results in a
        similar impact on the software design. \\
        Supports convergence & \partconv & In this case, the \gls{solid} principle
        assists in implementing the \gls{ns} theorem through specific design choices.
        However, it is essential to note that applying the principle does
        not inherently ensure adherence to the corresponding theorem. \\
        No convergence & \noconv & This denotation signifies a lack of alignment between
        the \gls{solid} principle and the corresponding theorem. \\
    \end{tabular}
\end{table}
\subsection{Converging the Single Responsibility Principle}

\begin{table}[H]
    \begin{tabular}{ l | c | p{0.78\linewidth}}
        \toprule
        \gls{soc} & \converges & \gls{srp} and \gls{soc} share a common objective:
        facilitating evolvable software systems through the promotion of modularity, low
        coupling, and high cohesion. While there may be some differences in granularity
        when applying both principles according to the original definition of SOC, the
        more stringent definition of Separation of Concerns, offered by the \ns
        Theorems \ref{subsubsec_soc} minimizes these differences. As a result, the two
        principles can be regarded as practically interchangeable. In conclusion, SRP and
        SOC exhibit full convergence, as they both emphasize encapsulating
        'responsibilities' or 'concerns' within modular components of a software system.
        \\
        \midrule
        \gls{dvt} & \supports & While not immediately apparent, \gls{srp} offers supports
        for the \gls{dvt} theorem. While \gls{srp} emphasizes limiting the responsibility of
        each module, it does not explicitly require handling changes in data structures.
        However, following gls{srp} can still indirectly contribute to achieving \gls{dvt}
        by promoting the Law of Demeter
        \footnote{\url{https://en.wikipedia.org/wiki/Law_of_Demeter}}, which encourages
        modules to interact with each other only through well-defined interfaces. This
        approach can minimize the impact of data structure changes, although it does not
        guarantee full convergence with \gls{dvt}. \\
        \midrule
        \gls{avt} & \supports & Although not that apparent, \gls{srp} supports the \gls{dvt}
        theorem. While \gls{srp} emphasizes limiting the responsibility of each module, it
        does not explicitly require handling changes in data structures. However, following
        gls{srp} can still indirectly contribute to achieving \gls{dvt} by promoting the Law
        of Demeter \footnote{\url{https://en.wikipedia.org/wiki/Law_of_Demeter}}, which
        encourages modules to interact with each other only through well-defined interfaces.
        This approach can minimize the impact of data structure changes, although it does not
        guarantee full convergence with \gls{dvt}. \\
        \midrule
        \gls{sos} & \diverges & The convergence between \gls{srp} and the \gls{sos} theorem is
        not as direct as with other theorems. \gls{srp} focuses on assigning a single
        responsibility to each module but does not explicitly address state management.
        Nevertheless, by following \gls{srp}, developers can create modules that manage their
        state, which indirectly contributes to \gls{sos}. \\
        \bottomrule
    \end{tabular}
    \caption{Converge \gls{srp} with \ns}
    \label{tab_convergence_srp}
\end{table}

\subsection{Converging the Open/Closed Principle}

\begin{table}[H]
    \begin{tabular}{ l | c | p{0.78\linewidth}}
        \toprule
        \gls{soc} & \converges & The \gls{ocp} converges with the \gls{soc} theorem.
        \gls{ocp} states that software implementations should be open for extension but
        closed for modification. When applying \gls{ocp} correctly, modifications are
        separated from the original implementations. For example by creating a new
        implementation of an interface or a base class. Conversely, adhering to \gls{soc}
        does not guarantee the fulfillment of \gls{ocp}, as \gls{soc} focuses on
        modularization and encapsulation, rather than the extensibility of modules. \\
        \midrule
        \gls{dvt} & \supports & The \gls{ocp} supports the \gls{dvt} theorem. While
        \gls{dvt} aims to handle changes in data structures without impacting the system,
        \gls{ocp} focuses on the extensibility of software entities. \gls{ocp} does not
        explicitly address data versioning, and thus does not guarantee full convergence
        with \gls{dvt}. However, by designing modules that follow \gls{ocp}, developers
        can create components that are more adaptable to changes in data structures. \\
        \midrule
        \gls{avt} & \converges & The \gls{ocp} converges with the \gls{avt} theorem. Both
        principles emphasize the importance of allowing changes or extensions to actions
        or operations without modifying existing implementations. By adhering to
        \gls{ocp}, developers can create modules that can be extended to accommodate new
        actions or changes in existing ones, effectively achieving \gls{avt}. \\
        \midrule
        \gls{sos} & \diverges & The \gls{ocp} has an indirect relationship with the
        Separation of States (SoSt) theorem. However, not on a level where we can speak of
        convergence. \gls{sos} emphasizes isolating different states within a system.
        Adhering to \gls{ocp} alone does not guarantee full separation of states. \\
        \bottomrule
    \end{tabular}
    \caption{Converge \gls{ocp} with \gls{ns}}
    \label{tab_convergence_ocp}
\end{table}
\subsection{Converging the Liskov Substitution Principle}

\begin{table}[H]
    \begin{tabular}{ l | c | p{0.80\linewidth}}
        \toprule
        \acrshort*{soc} & \conv & Adhering to \gls{lsp} in the software design leads to a more
        modular design and separation of specific concerns. Therefore we state that \gls{lsp}
        converges with \gls{soc}. \gls{lsp} states that objects of a derived class should be
        able to replace objects of the base class without affecting the correctness of the
        program. This can only be achieved by a strict separation of concerns in combination
        with Action version Transparent implementations of the signature.
        \\
        \midrule
        \acrshort*{dvt} & \noconv & \gls{lsp} has a limited alignment with the \gls{dvt}
        theorem. While \gls{lsp} focuses on the substitutability of objects in class
        hierarchies, \gls{dvt} aims to handle changes in data structures without impacting
        the system. By following \gls{lsp}, developers can ensure that derived classes can
        be substituted for their base classes, which may help reduce the impact of data
        structure changes on the system. However, \gls{lsp} does not explicitly address
        data versioning and thus does not guarantee full convergence with \gls{dvt}. \\
        \midrule
        \acrshort*{avt} & \partconv & The \gls{lsp} supports the \gls{avt} theorem. Both principles
        emphasize the importance of allowing the extensibility of the system, without
        negatively impacting the desired requirements. By adhering to \gls{lsp}, developers
        can create class hierarchies that can be easily extended to accommodate new actions or
        changes in existing ones, which may contribute to achieving \gls{avt}. However,
        adhering to \gls{lsp} alone may not guarantee full convergence with \gls{avt}. 
        \\
        \midrule
        \acrshort*{sos} & \noconv & By designing class hierarchies according to \gls{lsp},
        developers can create components that are less prone to side effects caused by shared
        states. However, the alignment between \gls{lsp} and \gls{sos} is very weak, and
        adhering to \gls{lsp} alone may not guarantee full separation of states. 
        \\
        \bottomrule
    \end{tabular}
    \caption{Converge \gls{lsp} with \gls{ns}}
    \label{tab_convergence_lsp}
\end{table}
\subsection{Converging the Interface Segregation Principle}

\begin{table}[H]
    \begin{tabular}{ l | c | p{0.80\linewidth}}
        \toprule
        \gls{soc} & \converges & The \gls{isp} converges with the \gls{soc} theorem, as
        both principles emphasize the importance of modularity and the separation of
        concerns. \gls{isp} states that clients should not be forced to depend on
        implementation they do not use, promoting the creation of smaller, focused
        interfaces. By adhering to \gls{soc} and designing target interfaces, inherently
        support \gls{soc}, leading to more evolvable software systems. \\
        \midrule
        \gls{dvt} & \supports & The \gls{isp} has an indirect relationship with the
        \gls{dvt} theorem. When adhering to the \gls{isp}, a developer can create a
        specific interface supporting a specific version of data entities. Although this
        approach can help minimize the impact of data structure changes on the system, it
        does not guarantee full \gls{dvt} information system. \\
        \midrule
        \gls{avt} & \supports & The \gls{isp} supports the \gls{avt} theorem. Both
        principles emphasize the importance of separating actions within a system.
        \gls{isp} promotes the creation of interfaces for each (version of an) action,
        which aligns with the core concept of \gls{avt}. This alignment allows for a more
        manageable system, where changes in one action do not lead to ripple effects
        throughout the entire system. Although the adherence to \gls{isp} does not
        guarantee full compliance with the \gls{avt} \\
        \midrule
        \gls{sos} & \diverges &  There is no alignment between \gls{isp} and \gls{sos}.
        Mostly because \gls{isp} focuses on the separation of interfaces and abstract
        classes, while \gls{sos} emphasizes isolating different states within a system.
        This is an implementation concern. By no means the application of \gls{isp} could
        lead to a separation of state. \\
        \bottomrule
    \end{tabular}
    \caption{Converge \gls{isp} with \gls{ns}}
    \label{tab_convergence_isp}
\end{table}
\subsection{Converging the Interface Segregation Principle}

\begin{table}[H]
    \begin{tabular}{ l | c | p{0.79\linewidth}}
        \toprule
        \gls{soc} & \converges & This principle converges with the \gls{soc} theorem.
        \gls{dip} states that high-level modules should not depend on low-level modules.
        When unavoidable, high-level modules should depend on abstractions of low-level
        modules and abstractions should not depend on details. By adhering to \gls{dip}
        correctly, developers can create modular and decoupled software systems, which
        aligns to break down a system into evolvable components. This convergence enables
        developers to create maintainable, scalable, and adaptable software systems that
        effectively manage complexity. Dependency Injection is a valuable (but not the
        only or mandatory) aspect of \gls{dip}. We have observed that the claim that the
        technique of Dependency Injection solves coupling between classes in an
        application is dangerous and in some cases wrong
        \parencite[215]{mannaert_normalized_2016}. Nevertheless, this technique, when
        applied correctly (see \ref{subsubsec_dip}), the artifact have pointed out that it
        has been a great asset in creating evolvable software, especially in the aspect of
        \gls{soc}.\\
        \midrule
        \gls{dvt} & \supports &  Adhering to the \gls{dip} can indirectly support the
        \gls{dvt} theorem. By adhering to the \gls{dip}, developers can promote
        implementations that encourage modules to interact with each other only through
        well-defined interfaces or abstractions. This approach can help minimize the
        impact of data structure changes on the system but does not guarantee full
        compliance with \gls{dvt}. \\
        \midrule
        \gls{avt} & \supports & The \gls{dip} can support the \gls{avt} theorem. Both
        principles emphasize the importance of isolating actions or operations within a
        system. By adhering to \gls{dip}, developers can create modular components that
        interact through abstractions, which may contribute to achieving AvT. However, the
        alignment between \gls{dip} and \gls{avt} less strong than with \gls{soc}, and
        adhering to \gls{dip} alone will not guarantee a system that entirely complies to
        \gls{avt}. \\ 
        \midrule
        \gls{sos} & \diverges & There is a very weak alignment between \gls{dip} and
        \gls{sos}. Although Developers can create components that are less prone to side
        effects caused by a shared state, this can hardly be contributed to the \gls{dip}.
        Therefore it is stated that there is no convergence between the two principles.\\
        \bottomrule
    \end{tabular}
    \caption{Converge \gls{dip} with \ns}
    \label{tab_convergence_dip}
\end{table}


\section{Converging elements} \label{sec_converging_elements}

In this section, we will apply a systematic cross-referencing approach to assess the level
of convergence between each of the elements of both \gls{ca} and \gls{ns}. Along with a
brief explanation, the level of convergence is denoted as followed:

\begin{table}[H]
    \begin{tabular}{ l l p{0.57\linewidth}} Strong convergence & \conv & Both
        elements have a high level of similarity or are closely related in terms of their
        purpose, structure, or functionality.\\
        Some convergence & \partconv &  Both elements have some similarities or
        share certain aspects in their purpose, structure, or functionality, but they are
        not identical or directly interchangeable.\\
        No convergence & \noconv &  The elements are not related or have no
        significant similarities in terms of their purpose, structure, or functionality.\\
    \end{tabular}
\end{table}
\subsection{Converging the Entity element}

\begin{table}[H]
    \begin{tabular}{ l | c | p{0.70\linewidth}}
        \toprule
        Data & \partconv & Both elements represent data objects that are part of the
        ontology or data schema of the application, and typically include attributes and
        relationship information. While both can contain a full set of attributes and
        relationships, the Data entity of \gls{ns} may also include a specific set of
        information that is required for a single task or use case. \\ \midrule

        Task & \noconv & The Entity is not convergent with the Task
        element of \gls{ns}. However, the Tasks element might operate on entities to
        perform business logic. \\ \midrule

        Flow & \noconv & The Entity and Flow are not convergent, as the Flow element
        represents the control between Tasks in \gls{ns}, while the Entity in \gls{ca}
        represents domain objects.\\ \midrule
        
        Connector & \noconv & The Entity element and Connector element are not
        convergent, as the Connector element in \gls{ns} is involved in between components,
        while the Entity in \gls{ca} represents domain objects.\\ \midrule
        
        Trigger & \noconv & The Entity element and Trigger element are not convergent,
        as the Trigger element in \gls{ns} is about event-based execution of Tasks, while
        the Entity in \gls{ca} represents domain objects.\\ \bottomrule

    \end{tabular}
    \caption{The convergence of the Entity element}
    \label{tab_convergence_entity}
\end{table}
\subsection{Converging the Interactor element}

\begin{table}[H]
    \begin{tabular}{ l | c | p{0.71\linewidth}}
        \toprule
        Data & \noconv &  The Interactor and Data elements are not convergent. However,
        Interactors might use Data elements as input and output during the execution of
        business logic.\\ \midrule

        Task & \conv &  The Task element in \gls{ns} is very closely related to
        the Interactor element of \gls{ca}, as both encapsulate the execution of business
        logic.\\ \midrule
        
        Flow & \partconv & The Interactor and Flow elements are partly convergent, as the
        Interactor orchestrates the flow of execution for a use case, which can involve
        multiple Tasks in \gls{ns}.\\ \midrule
        
        Connector & \noconv & The Interactor and Connector elements are not convergent.
        However, the Interactor might rely on connectors to communicate with other
        components in the system.\\ \midrule
        
        Trigger & \noconv & The Interactor and Trigger elements are not convergent.
        However, the Interactors can be triggered by events or external requests, similar to
        \gls{ns} Trigger elements. \\ \bottomrule
    \end{tabular}
    \caption{The convergence of the Interactor element}
    \label{tab_convergence_interactor}
\end{table}

\subsection{Converging the RequestModel element} \label{converging_requestmodel_element}

\begin{table}[H]
    \begin{tabular}{ l | c | p{0.70\linewidth}}
        \toprule
        Data & \partconv & Both elements represent data objects that are part of the
        ontology or data schema of the application, and typically include attributes and
        relationship information. While they may contain a specific set of information as
        input for a Task or use case, both elements can also contain a full set of
        attributes and relationships. However, unlike the Data entity in \gls{ns}, which
        may include only a subset of information necessary for a specific Task or use
        case, it may also include the full set of information required for Tasks other
        purposes. \\
        \midrule

        Task & \noconv & The RequestModel is not convergent with the Task element of
        \gls{ns}. However, the Tasks element might operate on RequestModels as input
        parameters to perform business logic. \\ \midrule
        
        Flow & \noconv & The RequestModel and Flow are not convergent, as the Flow element
        represents the control between Tasks in \gls{ns}, while the RequestModel in \gls{ca}
        represents (parts of) domain objects.\\ \midrule
        
        Connector & \noconv & The RequestModel element and Connector element are not
        convergent, as the Connector element in \gls{ns} is involved in the communication
        between components, whilst the RequestModel in \gls{ca} represents (parts of)
        domain objects.\\ \midrule
        
        Trigger & \noconv & The RequestModel element and Trigger element are not convergent,
        as the Trigger element in \gls{ns} is about event-based execution of Tasks, while
        the RequestModel in \gls{ca} represents (parts of) domain objects.\\
        
        \bottomrule
    \end{tabular}
    \caption{The convergence of the RequestModel element}
    \label{tab_convergence_requestmodel}
\end{table}
\subsection{The ResponseModel Element}

\evaluateElementTable{ResponseModel}{tab_convergence_responsemodel}{
    \addEvalRow{Data & \partconv & Both elements represent data objects that are part of the
        ontology or data schema of the application, and typically include attributes and
        relationship information. While they may contain a specific set of information as
        output for a Task or use case, both elements can also contain a full set of
        attributes and relationships. However, unlike the Data entity in \gls{ns}, which
        may include only a subset of information necessary for a specific Task or use
        case, it may also include the full set of information required for Tasks other
        purposes.}

    \addEvalRow{Task & \noconv & The ResponseModel is not convergent with the Task element of
        \gls{ns}. However, the Tasks element might operate on RequestModels as input
        parameters to perform business logic.}

    \addEvalRow{Flow & \noconv & The ResponseModel and Flow are not convergent, as the Flow element
        represents the control between Tasks in \gls{ns}, while the ResponseModel in \gls{ca}
        represents (parts of) domain objects.}

    \addEvalRow{Connector & \noconv & The ResponseModel element and Connector element are not
        convergent, as the Connector element in \gls{ns} is involved in communication
        between components, whilst the ResponseModel in \gls{ca} represents (parts of)
        domain objects.}
    
    \addEvalRow{Trigger & \noconv & The ResponseModel element and Trigger element are not convergent,
        as the Trigger element in \gls{ns} is about event-based execution of Tasks, while
        the ResponseModel in \gls{ca} represents (parts of) domain objects.}
}
\subsection{Converging the ViewModel element} \label{converging_viewmodel_element}

\begin{table}[H]
    \begin{tabular}{ l | c | p{0.70\linewidth}}
        \toprule
        Data & \partconv & The ViewModel and Data element of \gls{ns} is convergent to
        some degree. Both are involved in defining the structure of data used in the
        system. This could include required information about attributes and
        relationships. Additionally, the ViewModel could also represent information that
        is specifically intended for the representation of behavior for a user interface.
        \\ \midrule

        Task & \noconv & The ViewModel is not convergent with the Task element of
        \gls{ns}. The ViewModel is focused on presenting information, whilst the Task element
        is concerned with executing business logic. \\ \midrule
        
        Flow & \noconv & The ViewModel and Flow are not convergent, as the Flow element
        represents the control between Tasks in \gls{ns} and is not directly involved in
        the presentation of information. \\ \midrule
        
        Connector & \noconv & The ViewModel element and Connector element are not
        convergent, as the Connector element in \gls{ns} is involved in the communication
        between components, whilst the ViewModel in \gls{ca} is involved in the
        presentation of information.\\ \midrule
        
        Trigger & \noconv & The ViewModel element and Trigger element are not convergent,
        as the Trigger element in \gls{ns} is responsible for the event-based execution
        of Tasks, whilst the ViewModel in \gls{ca} is involved in the presentation of
        information.\\
        
        \bottomrule
    \end{tabular}
    \caption{The convergence of the ResponseModel element}
    \label{tab_convergence_viewemodel}
\end{table}
\subsection{The Controller Element} \label{converging_controller_element}

\evaluateElementTable{Controller}{tab_convergence_controller}{
    \addEvalRow{Data & \noconv & The Controllers and Data elements are not directly convergent, as
        the Controller element is focused on handling input/output from external systems, while Data
        elements represent domain objects.}

    \addEvalRow{Task & \noconv &  The Controllers and Task elements are not convergent. However,
        controllers might initiate Task elements when handling incoming requests.}

    \addEvalRow{Flow & \noconv &  The Controllers and Flow elements are not convergent, as
        the Controller element is focused on handling input/output from external systems,
        whilst the Flow element is concerned with the orchestration of Tasks.}

    \addEvalRow{Connector & \partconv & The Controller and Connector element are convergent to some
        degree. Both elements are involved in communication between components. The use of
        the Controller is a bit more strict it strictly defines communication from
        external parts of the systems, involving specific Interactor.}
    
    \addEvalRow{Trigger & \partconv & The Controller and the Trigger element of \gls{ns} are
        convergent to some degree as they both can initiate actions based on external
        events or requests. A Controller is primarily involved in receiving events or
        reqeusts from external sources, followed by the invocation of the appropriate
        interactor.}
}
\subsection{The Gateway Element}

\evaluateElementTable{Gateway}{tab_convergence_gateway}{
    \addEvalRow{ Data & \noconv &  The Gateway and Data element are not convergent. Nevertheless,
        the Gateway element might interact with data entities when providing access to
        external resources or systems.}

    \addEvalRow{Task & \noconv &  The Gateway and Task element are not convergent. Nevertheless,
        the Task elements might use gateways when interacting with external resources or systems
        during the execution of business logic.}

    \addEvalRow{Flow & \noconv & The Gateway and Flow element are not directly related, as the Flow
        element represents the orchestration between Tasks in \gls{ns}, whilst the Gateway
        element in \gls{ca} provide access to external resources or systems.}

    \addEvalRow{Connector & \conv & The Gateway and Connector element have a strong convergence, as both
        are involved in communication between components and provide interfaces for
        accessing external resources or systems.}
    
    \addEvalRow{Trigger & \noconv & The Gateway and Trigger element are not convergent, as the
        Trigger in \gls{ns} is about event-based execution of Tasks, whilst the Gateway in
        \gls{ca} provide access to external resources or systems.}
}
\subsection{The Presenter Element}

\evaluateElementTable{Presenter}{tab_convergence_presenter}{
    \addEvalRow{Data & \noconv &  The Presenter and Data element are not convergent. However, Data
        elements might be transformed into a suitable format for the user inter interface
        by \gls{ca}'s Presenter element.}

    \addEvalRow{Task & \noconv &  The Presenter and Task elements are not convergent. The
    Presenter element is focused on transforming output data to the user interface,
    while the Task element of \gls{ns} executes business logic.}

    \addEvalRow{Flow & \noconv & The Presenter and Flow element are not convergent. The Flow
        element of \gls{ns} represents the orchestration between Tasks, while the
        Presenter element in \gls{ca} is responsible for transforming output data to the
        user interface.}

    \addEvalRow{Connector & \noconv & The Presenter and Connector elements are not convergent.
        Although the Presenter element of \gls{ca} might rely on Connector elements of
        \gls{ns} to communicate with other components in the system.}
    
    \addEvalRow{Trigger & \noconv & The Presenter and Trigger elements are not convergent, as the
        Triggers element in \gls{ns} is about event-based execution of Tasks, while
        presenters in \gls{ca} are responsible for transforming output data on behalf of
        the user interface.}
}
\subsection{The Boundary Element}

\evaluateElementTable{Boundary}{tab_convergence_boundary}{
    \addEvalRow{Data & \noconv &  The Boundary and Data elements are not convergent, as the
        Boundary element is focused on separating concerns between components, while the
        Data element of \gls{ns} represents a domain object.}

    \addEvalRow{ Task & \noconv &  The Boundary and Flow Task element are not convergent, However,
        the Boundary element can be used in a Task element to ensure a clear separation
        of concerns between different modules.}

    \addEvalRow{ Flow & \partconv & The Boundary and Flow element are not convergent, However, the
        Boundary element can be used in a Flow element to ensure a clear separation of
        concerns between different modules.}

    \addEvalRow{Connector & \conv & The Boundary and Connector elements have a strong convergence,
        as both are involved in communication between components and help ensure loose
        coupling between these components.}
    
    \addEvalRow{ Trigger & \noconv & The Boundary and Trigger element are not convergent, as the
        Trigger in \gls{ns} is about event-based execution of Tasks, whilst the Boundary in
        \gls{ca} ensures the separation of concerns between components of the system.}
}


