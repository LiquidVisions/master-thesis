\chapter{Introduction} \label{introduction}

Although considerable time has passed, \citeauthor{lehman_programs_1980} laws of software
evolution \cite{lehman_programs_1980} are still very relevant these days for contemporary
corporations delivering software IT solutions. The laws describe the balance between
forces driving new developments on one hand, and forces that slow down progress on the
other hand. 

%benoemen van de uitdagingen
%1. change is constant
%2. lehmans law
%3. evolvable systems

%normalized systems als oplossing

%wat is normalized systems

\section{Problem statement} \label{problem_statement}
Companies that apply the Normalized Systems Theory research into their products are
primarily using Java EE as a programming language. The company NSX for example has
implemented their generation tools, modelling suite (Prime Radiant) and expander using
this programming language. Java EE is still a very popular programming language for
enterprise-, and IT organizations. Many software solutions are created and maintained
using this programming language. The Normalized Systems Theorem is not only applicable to
Java EE. The principles and design patterns that derive from the Normalized Systems
Theorem are in fact applicable for any object-oriented programming languages. 

Another example of a popular programming language in enterprise-, and IT organizations is
C\#. There is however no documented research, or proof of experiences on C\# software
projects using Normalized Systems Theory with the aspects of integration, expansion and
rejuvenation.





\section{Research questions} \label{ResearchQuestions}

Nunc posuere quam at lectus tristique eu ultrices augue venenatis. Vestibulum ante ipsum
primis in faucibus orci luctus et ultrices posuere cubilia Curae; Aliquam erat volutpat.
Vivamus sodales tortor eget quam adipiscing in vulputate ante ullamcorper. Sed eros ante,
lacinia et sollicitudin et, aliquam sit amet augue. In hac habitasse platea dictumst.