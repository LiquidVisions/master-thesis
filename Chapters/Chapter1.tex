\chapter{Introduction} \label{introduction}

\enquote{\emph{Pantha Rhei}} is, according to \emph{Plato}, one of the famous
philosophical statements first described by the Greek philosopher
\emph{Heraclitus}.\footnote[1]{\url{https://plato.stanford.edu/entries/process-philosophy/}}
Translated as \enquote{everything flows} this statement is an unambiguous commitment to
ubiquitous dynamics of everything that exists. \enquote{Life is flux}, one of the
constants in life is change and its best we act accordingly.

In the realms of Software Engineering the \enquote{laws of software evolution}
\parencite[]{lehman_programs_1980} refers to a series of laws described by
\citeauthor{lehman_programs_1980} starting from 1974. With these Laws, he describes the
balance between the forces driving new developments on the one hand (a change), and the
forces that slow down progress on the other hand. Based on \emph{Heraclitus} philosophical
statement we assume a software engineering project frequently will be subjected to change,
possibly due to changing functional requirements and technological progress. As these
changes emerges, the complexity of these software projects will gradually increase over
time. If the system is not adapted appropriately the combinatorial effects of these
changes will result in ever-increasing complexity and render the software system
eventually obsolete, according to \citeauthor{lehman_programs_1980}
\parencite[]{lehman_programs_1980}.

As the competitive environments of contemporary organizations are changing continuously,
the speed at which changes follow each other is also increasing. IT organization are
attempting to cope with this trend by adopting agility and maturing its agile practices
\parencite[]{2024_SIM_key_issues_and_trends}. Agility is defined as a measure for
contemporary organizations to adept to new environments and to cope with rapid change
\parencite[]{neumann_strategic_1994}.

The subjects discussed in previous paragraphs depict the current challenges of software
evolvability

\section{Problem statement} \label{problem_statement}

\subsection{Normalized Systems Theorems}
\lipsum[1-1]

\subsection{Clean Architecture}
\lipsum[1-1]
%3. evolvable systems
%normalized systems als oplossing
%wat is normalized systems

\subsection{Normalized Systems}
Companies that apply the Normalized Systems Theory research into their products are
primarily using Java EE as a programming language. The company NSX for example has
implemented their generation tools, modelling suite (Prime Radiant) and expander using
this programming language. Java EE is still a very popular programming language for
enterprise-, and IT organizations. Many software solutions are created and maintained
using this programming language. The Normalized Systems Theorem is not only applicable to
Java EE. The principles and design patterns that derive from the Normalized Systems
Theorem are in fact applicable for any object-oriented programming languages. 

Another example of a popular programming language in enterprise-, and IT organizations is
C\#. There is however no documented research, or proof of experiences on C\# software
projects using Normalized Systems Theory with the aspects of integration, expansion and
rejuvenation.

\section{Research questions} \label{research_questions}
The goal of this research is to determine if the design of 