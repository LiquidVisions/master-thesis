\section{Normalized Systems: Impacting software stability} \label{sec_ns_theory}

\ns is a software development approach that prioritizes achieving software stability
through the use of standardized, modular components and interfaces. This theory is
informed by several scientific disciplines, including systems theory, mathematics, and
computer science, as well as some other software development approaches, such as agile
development and domain-driven design.

\ns originated in the field of software engineering. However, the underlying theory
of \ns can be applied to various other domains, such as Enterprise Engineering,
Business Process Modeling, and document management. This research acknowledges the
software engineering background of \ns. It consistently refers to software and
Information Systems when referring to \enquote*{artifacts.} However, the reader should
realize that the concepts and artifacts are not restricted to software artifacts alone.

\subsection{Towards stability} \label{subsec_on_stability}

In several disciplines, stability has been defined as \emph{Bounded Input Bounded Output}
(BIBO). It is the fundamental property of a system when subjected to bounded input
disturbances. BIBO stability ensures that the output of a system will also be bounded,
preventing uncontrolled or unexpected behavior \parencite[270]{mannaert_normalized_2016}. 

A real-world example of the importance of stability is the Tacoma Narrows Bridge in
Washington State, USA. This bridge, depicted in Figure \ref*{fig_bridge}, collapsed on
November the 7th, 1940. This was caused due to wind-induced oscillations called
aeroelastic flutter. The wind (Input) induced oscillations in the bridge, causing it to
start swaying back and forth (Output). These oscillations were initially small, but as
they continued, they began to increase in amplitude and magnitude. Eventually, this 
caused the bridge to collapse.

\begin{figure}[H]
    \centering
    \includegraphics[width=0.6\textwidth]{Figures/bridge.pdf}
    \caption[Tacoma Narrows Bridge]{Tacoma Narrows Bridge (Galloping Gertie)}
    \label{fig_bridge}
\end{figure}

Stability can also be used in the context of software engineering. In the context of
\gls{ns}, it is considered a critical property that ensures that the software is not
excessively sensitive to small changes \parencite[270]{mannaert_normalized_2016}. New
functional requirements should only lead to a fixed and expected amount of changes in
the source code. 

Conversely, instabilities occur when the total number of modifications relies on the size
of the software artifact. When there are software instabilities, the number of changes
will grow over time in parallel with the growth of the system. These instabilities are
referred to as combinatorial effects \parencite[270]{mannaert_normalized_2016}. 

When combinatorial effects are absent, the software artifact can be considered evolvable.
\section{Software evolvability} \label{software_evolvability}

An important aspect of this thesis is to determine the evolvability of software artifacts
with a Clean Architecture design. An evolvable software artifact should not have
instabilities: a bounded amount of additional functional requirements cannot lead to an
unbounded amount of additional (versions of) software primitives \parencite[273]{mannaert_normalized_2009}. 

\subsection{Stability}
<<verwijzing naar positive feedback loops uit de theorie van NS en de effect op
evolueerbaarheid van software.>>

\subsection{Combinatorial effects \& anticipated change drivers.}
<<heeft relatie tot hoofdstuk evaluation en toelichtingen dat software aanpassingen niet
zouden moeten leiden tot een toename in combinatorial effects>>

\subsection{Modularity}
<<toelichten waarop modularity invloed heeft op de evolueerbaarheid van de software>>


\subsection{Modularity}

A Software module can be defined as self-contained units of code that perform specific
tasks or sets of tasks within a more extensive system. A software module is designed to operate
independently of other modules, with well-defined interfaces that allow it to communicate
and exchange data with other modules if necessary \autocite[22]{mannaert_normalized_2016}.

A module can be considered a hierarchical and recursive concept. They are independent of
their size (lines of code) or computational magnitude. They can be as small as a function
as part of a class. The class itself can also be considered a module. A group of classes
contained in a Dynamic Link Library (DLL) or Application Programming Interface (API) can
also be considered a module of an even more extensive system. 

An essential part of the design of a software system is to identify the possible different
modules and their interaction interfaces. Figure \ref{fig:modulair_components} presents a
high-level depiction of modular manifestations in the artifact, with additional examples
of modularity found in more granular implementations. Further discussion of this
architecture is provided in Chapter \ref{sec:ca_theory}.
\subsubsection{Cohesion} \label{subsubsec:on_cohesion}

The term cohesion denotes the extent to which the various structural components of a
software system operate cohesively towards a singular and well-defined objective or goal.
Empirical studies in software engineering have extensively demonstrated the significance
of cohesion, linking higher levels of cohesion with reduced defects, enhanced
maintainability, and greater openness to change. Consequently, achieving high cohesion has
been associated with an overall improvement in software quality attributes such as
reliability, maintainability, reusability, and evolvability.

Cohesion facilitates the reduction of complexity and interdependence among the components
of a system, thereby contributing to a more efficient, maintainable, and reliable system.
By organizing components around a shared purpose or function, or by standardizing their
interfaces, data structures, and protocols, cohesion can offer the following benefits:

\begin{itemize}
    \item \textbf{Reduce redundancy and duplication of effort}: \\
    Cohesion ensures that components are arranged around a common purpose or function,
    reducing duplicates or redundant code. This simplifies system comprehension,
    maintenance, and modification.
    \item \textbf{Promoting code reuse:}\\
    Cohesion facilitates code reuse by making it easier to extract and reuse components
    designed for specific functions. This saves time and effort during development and
    enhances overall system quality.
    \item \textbf{Enhance maintainability:}\\
    Cohesion decreases the complexity and interdependence of system components, making it
    easier to identify and rectify bugs or errors in the code. This improves system
    maintainability and reduces the risk of introducing new errors during maintenance.
    \item \textbf{Increase scalability:}\\
    Cohesion improves a system's scalability by enabling it to be extended or modified
    effortlessly to accommodate changing requirements or conditions. By designing
    well-organized and well-defined components, developers can easily add or modify
    functionality as needed without disrupting the rest of the system.  
\end{itemize}


\subsection{Coupling: The bad} \label{subsec:on_coupling}

Coupling is a central concept in software engineering that pertains to the degree of
interdependence among software modules or components. The level of coupling between
modules denotes the strength of their relationship, whereby a high level of coupling
implies a significant degree of interdependence. Conversely, low coupling signifies a
weaker relationship between modules, where modifications in one module are less likely to
impact others.

The negative impacts of excessive coupling on software systems are considerable. High
coupling can render software systems difficult to maintain, modify, or evolve. It can
impede the identification and resolution of errors within a system, leading to prolonged
debugging periods. Additionally, it can cause fragility in the system, where slight
modifications in one module can trigger cascading failures throughout the entire system.
Therefore, it is crucial for software engineers to minimize coupling between modules while
maintaining a cohesive design. By developing systems with low coupling, software engineers
can construct more maintainable, scalable, and adaptable systems that are easier to evolve.

Coupling, in software engineering, can take several forms, including content, common,
control, stamp, and data coupling. Content coupling occurs when one module accesses or
modifies the internal data or logic of another module, leading to high interdependence and
difficulty in isolating errors. Common coupling occurs when several modules access and use
the same global data, increasing their interdependence and reducing modularity. Control
coupling occurs when one module controls the execution flow of another module, making it
difficult to modify or reuse the controlled module. Stamp coupling arises when two modules
share a common data structure, leading to tight coupling and high interdependence.
Finally, data coupling exists when two modules share data, which can lead to coupling
between them.

To avoid the negative impacts of coupling on software systems, software engineers should
aim to minimize the degree of coupling between modules. This can be achieved by designing
cohesive, loosely coupled modules with well-defined interfaces. Loose coupling enables
each module to operate independently, reducing the impact of modifications made to other
modules. By implementing modular design principles, such as high cohesion and low
coupling, software engineers can develop systems that are easier to maintain, test, and
evolve.

In conclusion, coupling is a critical concept in software engineering that can have a
considerable impact on the maintainability, flexibility, and scalability of software
systems. By minimizing coupling between modules, software engineers can develop more
robust, adaptable systems that are easier to modify and evolve. Adopting modular design
principles can also facilitate the development of cohesive, loosely coupled modules that
enable the independent operation and reduce the impact of modifications made to other
modules.
\subsection{Expansion and code generation} \label{subsec_expansion}

Creating and maintaining a stable and evolvable system is a particularly challenging and
meticulous engineering job. Developers are required to have a sound knowledge of \ns,
whilst implementing new requirements in an always consistent manner. Given the recurring
structures, processing new requirements is a very precise, strict and meticulous job.
\parencite[403]{mannaert_normalized_2016} Manual labor could be error-prone given modern
time-to-market requirements. 

Therefore, it seems logical to automate the instantiation process of software structures
and use code generation for recurring tasks \parencite[403]{mannaert_normalized_2016}.
This is where the concepts of code generation and expansion come into place. This does not
only refer to the automatic process of adapting and maintaining software to new
requirements, architectural enablers and technological alterations. It also embraces
manually added craftings to the software, the so-called plugin code. These craftings are
preserved after each expansion by a method that is called harvesting and rejuvenation
\parencite[405-406]{mannaert_normalized_2016}.


\subsection{The Theoretical Framework} \label{subsec:ns_desing_theorems}

\gls{ns} consists of a theoretical framework describing a set of design principles. These
principles are the basis for achieving the concepts of stability, evolvability and
modularity. \gls{ns} provides a rigorous and mathematical foundation for these theorems
and they offer guidelines for designing and developing software systems. In the following
sections, we will focus on the principles of \gls{ns} very briefly as they have been
extensively described in various scientific papers.

We know from Chapter \ref{sec:artifact_requirements} that the design artifacts, as a part
of this research, are implemented based on the \gls{ca} principles. Therefore, contrary to
Chapter \ref{sec:ca_theory}, there will be no references to the manifestations of the NS
design theorems in the design.
\subsubsection*{Separation of Concerns}

\gls{soc} as a principle has first been mentioned by
\citeauthor{dijkstra_selected_1982}\footnote{\url{https://en.wikipedia.org/wiki/Separation_of_concerns}}
as the crucial principle to design modular software architecture
\parencite[]{dijkstra_selected_1982}. \gls{soc} promotes the idea that a program should be
divided into distinct sections, each addressing a separate concern or aspect of a design
problem. This allows for a more organized and maintainable source code. When implemented
correctly, a change to one concern does not affect the others. \gls{soc} should be applied
at the level of individual modules, rather than the level of an entire program.

\gls{soc} has been adopted as one of the design theorems of \gls{ns}, although it has a
stricter definition of this principle\parencite{mannaert_normalized_2016}.

\begin{tcolorbox}[boxrule=0.1pt, colback=mygray, title=Theorem I,colbacktitle=gray]
    \textit{A processing function can only contain a single task to achieve stability.}
\end{tcolorbox}
\subsubsection*{Data version Transparancy}

\gls{dvt} is the act of encapsulation of data entities for specific tasks at hand. This
results in the fact that data structures can have multiple versions often mentioned as
Data Transfer Objects in modern software engineering projects. In other words, it should
be possible to update the data entity without affecting the processing functions. This
leads to the following description of the theorem \parencite[280]{mannaert_normalized_2016}.

\begin{tcolorbox}[boxrule=0.1pt, colback=mygray, title=Theorem II,colbacktitle=gray]
    \textit{A data structure that is passed through the interface of a processing function
    needs to exhibit version transparency in order to achieve stability.}
\end{tcolorbox}

\gls{dvt} is widely used in various technological applications. practically every web
service currently known supports some type of versioning. In restful APIs for example, it
is common practice to support versioning over the URI. It is considered a best practice
to encapsulate breaking changes in a new version of the endpoint/service so that the
consumers are not (directly) affected by the change. In modern Object Oriented languages,
gls{dtv} is also supported by the ability to determine the scope of visibility of the
modifiers of the various programming constructs like fields, properties, interfaces and
classes. Also known as information hiding
\parencites{parnas_criteria_1972}[278]{mannaert_normalized_2016}.
\subsubsection{Action version Transparancy}
\gls{avt} is the property of a system to modify existing processing functions without
affecting the existing ones. It should be possible to upgrade a function without affecting
the callers of those functions. This description leads to the following theorem
\parencite[282]{mannaert_normalized_2016}.

\mycolorbox{A processing function that is called by another processing function, needs to exhibit version transparency in order to achieve stability.}{Theorem III}

Most of the modern technology environments support some form of \gls{avt}. Polymorphism is
a widely used technique in order to support this theorem. Specifically, parametric
polymorphism \footnote{\url{https://en.wikipedia.org/wiki/Parametric_polymorphism}} allows
for a processing function to have multiple input parameters. There are also quite some
design patterns supporting this theorem. Some random examples are the state pattern
\footnote{\url{https://en.wikipedia.org/wiki/State_pattern}}, facade pattern
\footnote{\url{https://en.wikipedia.org/wiki/Facade_pattern}} and observer pattern
\footnote{\url{https://en.wikipedia.org/wiki/Observer_pattern}}.
\subsubsection*{Separation of State}

\gls{sos} is a theorem that is based on the idea that processing functions should not
contain any state information but instead should rely on external data structures to store
state information. By separating state information from processing functions, Normalized
Systems can achieve a higher level of flexibility and adaptability. External data
structures can be updated or replaced without affecting the processing functions
themselves, which greatly reduces the change of unwanted ripple effects. This theorem is
described as followed: \parencite[258]{mannaert_normalized_2016}.

\mycolorbox{Calling a processing function within another processing function, needs to exhibit state keeping in order to achieve stability.}{Theorem IV}
\subsection{Normalized Elements} \label{subsec:ns_elements} 

In the context of the \gls{ns} Theory approach, the goal is to design evolvable software,
independent of the underlying technology. Nevertheless, when implementing the software and
its components, a particular technology must be chosen. For Object Oriented Programming
Languages like Java, the following Normalized Elements have been proposed
\parencite{mannaert_normalized_2016}[363-398].

This research's artifacts utilized C\# as the primary programming language. It is essential
to recognize that different programming languages may necessitate alternative constructs
\parencite{mannaert_normalized_2016}[364]. Given the strong similarities between C\# .NET
and Java, it is assumed that the same Normalized Elements are applicable for C\# .NET
implementation of this research's artifacts.

\subsubsection*{title}{The Data Element}
This is an object that represents a piece of data in the system. Data elements are used to
pass information between processing functions and other objects. In Normalized Systems,
data elements are typically standardized to ensure consistency across the system.

\subsubsection*{The Task Element}
This is an object that represents a specific task or action in the system. Tasks can be
composed of one or more processing functions and can be used to represent complex
operations within the system.

\subsubsection*{The Connector Element}
This object is used to connect different parts of the system together. Connectors can be
used to link processing functions, data elements, and other objects, allowing them to work
together seamlessly.

\subsubsection*{The Flow Element}
This object represents the flow of control through the system. It determines the order in
which processing functions are executed and can be used to handle error conditions or
other exceptional cases.

\subsubsection*{The Trigger Element}
a trigger element is an object that reacts to specific events or changes in the system
by executing predefined actions.