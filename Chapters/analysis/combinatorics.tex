\section{Analysis on Combinatorial Effects} \label{combinatorics}

Besides the theoretical analysis by comparing the principles in section
\ref{sec_converging_principles} and elements in section \ref{sec_converging_elements}, we
have also analyzed the combinatorial effects on the artifacts. To ensure clarity, we have
divided the analysis into the following change dimension, which we will describe in
successive sections.

\subsection{The Mirror World}
\textcite[137]{mannaert_normalized_2016} uses the term \enquote*{Mirror world} as an
analogy that refers to the activation of the technology of the information system. 

The analysis of the both artifacts did not show immediate combinatorial effects, aside
from the analysis described in the next section of Combinatorics in the templates.
However, table \ref{tab_convergence_principles_summarized} clearly shows that \gls{sos} is
not represented by any of the design principles of \gls{ca}. Therefore, artifacts solely
based on \gls{ca} principle will potentially lack stability and evolvability when
implementing stateful solutions. Nevertheless, we could not detect any combinatorial
effects due to the underrepresentation of \acrlong{sos} in \gls{ca}, which might have been
influenced due to the absence of complex stateful implementations, aside from Interactors
handling multiple actions similarly as how this is prescribed by the Workflow element of
\gls{ns}.

As indicated in table \ref{tab_convergence_elements_summarized}, there also seems to be a
lack of a strong foundation for receiving external triggers in the design philosophy of
\gls{ca}. The Controller element partially represents this. This feature is typically
utilized for web-based platforms like websites and Restful APIs. However, this approach
may not be as thorough regarding receiving external triggers from different technologies
or systems. Also here we could not detect any combined effect which might have been
influenced due to the absence handling external triggers.

\subsection{The Templates}
Using templates in the Clean Architecture Expander has led to some notable Combinatorial
effects when changing the names of Entities, Attributes, and Namespaces or naming
conventions of certain pre- and postfixes. These combinatorial effects are attributed to
the lack of support for the Data Version Transparency principle in the \gls{ca}
principles.

\subsection{Technologies and frameworks}
We did not observe or find any combinatorial effects using frameworks and technologies
that are part of the functionality. The Artifacts uses several frameworks for data
persistence (EntityFramework with Microsoft Azure SQL), Logging (NLog), and Template
rendering engine (Scriban). Each of these technologies is implemented adhering to the
\gls{lsp} principle. We have found that replacing them is an anticipated change and a
relatively simple task when adhering to the contracts that separate the implementation of
the technology from its use.

However, we observe a combinatorial effect when requirements dictate that the programming
language is replaced, for example, using Java instead of C\#. When the requirement only
applies to the generated artifact, a new expander should be created, impacting the uses of
frameworks and templates. In this case, the impact of combinatorial effects is moved from
the generated artifact to the expander.


\subsection{The Craftings}
Implementing the Harvesting and Injection process has led to some minor instabilities.
Currently, there is a lack of support for re-injecting craftings on elements moved to a
different target folder. In addition, changing the names of placeholders in the templates
also leads to failures when re-injecting craftings. These combinatorial effects are
attributed to the lack of support for the Data Version Transparency principle in the
\gls{ca} principles.