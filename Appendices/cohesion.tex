\chapter{Component Cohesion Principles} \label{appendix_cohesion_principles}

\begin{table}[H]
    \small
    \begin{tabular}{ P{0.25\linewidth} | p{0.69\linewidth}} 
        \hline
        \textbf{Name} & \textbf{Description} \\ \hline
        \acrlong{rep} & \acrshort{rep} is a concept related to software development that
        refers to the balance between reusing existing software components and releasing
        new ones to ensure the efficient use of resources and time
        \parencite[104]{robert_c_martin_clean_2018}.\\ \midrule 
        
        \acrlong{ccp} & In the context of Clean Architecture, the \acrshort{ccp} states
        that classes or components that change together should be packaged together. In
        other words, if a group of classes is likely to be affected by the same kind of
        change, they should be grouped into the same package or module. This approach
        enhances the maintainability and modularity of the software
        \parencite[105]{robert_c_martin_clean_2018}.\\ \midrule 
        
        \acrlong{crp} & \acrshort{crp} states that classes or components that are reused
        together should be packaged together. It means that if a group of classes tends to
        be used together or has a high level of cohesion, they should be grouped into the
        same package or module. This approach aims to make it easier for developers to
        reuse components and understand their relationships
        \parencite[107]{robert_c_martin_clean_2018}.\\

        \bottomrule
    \end{tabular}
    \caption{The component Cohesion Principles}
    \label{appendix_tab_cohesion_principles}
\end{table}

Cohesion facilitates the reduction of complexity and interdependence among the components
of a system, thereby contributing to a more efficient, maintainable, and reliable system.
By organizing components around a shared purpose or function or by standardizing their
interfaces, data structures, and protocols, cohesion can offer the following benefits:

\begin{itemize}
    \item \textbf{Reduce redundancy and duplication of effort}: \\
    Cohesion ensures that components are arranged around a common purpose or function,
    reducing duplicates or redundant code. This simplifies system comprehension,
    maintenance, and modification.
    \item \textbf{Promoting code reuse:}\\
    Cohesion facilitates code reuse by making it easier to extract and reuse components
    designed for specific functions. This saves time and effort during development and
    enhances overall system quality.
    \item \textbf{Enhance maintainability:}\\
    Cohesion decreases the complexity and interdependence of system components, making it
    easier to identify and rectify bugs or errors in the code. This improves system
    maintainability and reduces the risk of introducing new errors during maintenance.
    \item \textbf{Increase scalability:}\\
    Cohesion improves a system's scalability by enabling it to be extended or modified
    effortlessly to accommodate changing requirements or conditions. By designing
    well-organized and well-defined components, developers can easily add or modify
    functionality as needed without disrupting the rest of the system.  
\end{itemize}