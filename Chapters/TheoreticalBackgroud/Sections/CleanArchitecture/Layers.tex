\subsection{Layers and components} \label{subsec_layers}

\gls{ca} organizes software systems into distinct layers or components, each
with its own responsibilities. This structure promotes the separation of concerns,
maintainability, testability, and adaptability. The following section is a short
description of each of the layers \parencite{robert_c_martin_clean_2018}.

\subsubsection{Domain layer}
This layer contains the core business objects, rules, and domain logic of the application.
Entities represent the fundamental concepts and relationships in the problem domain and
are independent of any specific technology or framework. The domain layer focuses on
encapsulating the essential complexity of the system and should be kept as pure as
possible.

\subsubsection{Application layer}
This layer contains the use cases or application-specific
business rules that orchestrate the interaction between entities and external systems. Use
cases define the behavior of the application in terms of the actions users can perform and
the expected outcomes. This layer is responsible for coordinating the flow of data between
the domain layer and the presentation or infrastructure layers, while remaining agnostic
to the specifics of the user interface or external dependencies.

\subsubsection{Presentation layer}
This layer is responsible for translating data and interactions between the use cases and
external actors, such as users or external systems. Interface adapters include components
like controllers, view models, presenters, and data mappers, which handle user input,
format data for display, and convert data between internal and external representations.
The presentation layer should be as thin as possible, focusing on the mechanics of user
interaction and deferring application logic to the use cases.

\subsubsection{Infrastructure layer}
This layer contains the technical implementations of external systems and dependencies,
such as databases, web services, file systems, or third-party libraries. The
infrastructure layer provides concrete implementations of the interfaces and abstractions
defined in the other layers, allowing the core application to remain decoupled from
specific technologies or frameworks. This layer is also responsible for any configuration
or initialization code required to set up the system's runtime environment.

By organizing code into these layers and adhering to the principles of \gls{ca},
developers can create software systems that are more flexible, maintainable, and testable,
with well-defined boundaries and separation of concerns.