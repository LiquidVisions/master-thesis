\section{On modularisation} \label{sec:on_modules}

Software modules can be defined as self-contained units of code that perform specific
tasks or sets of tasks within a larger system. A software module is designed to operate
independently of other modules, with well-defined interfaces that allow it to communicate
and exchange data with other modules if necessary \autocite[22]{mannaert_normalized_2016}.

A module can be considered a hierarchical and recursive concept. They are independent of
their size (lines of code) or computational magnitude. They can be as small as a function
as part of a class. The class itself can also be considered a module. A group of classes
contained in a Dynamic Link Library (DLL) or Application Programming Interface (API) can
also be considered a module of an even bigger system. 

An important part of the design of a software system is to identify the possible different
modules and their interaction interfaces. There is a wide consensus about two fundamental
rules when thinking of-, and designing modules: \emph{high cohesion} and \emph{low
coupling} \autocite[22]{mannaert_normalized_2016}.