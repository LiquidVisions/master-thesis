\chapter{Theoretical background} \label{chap_theoreticalbackground} 

The goal of this thesis is to investigate whether \gls{ca} approach aligns with the goals
of \gls{ns}. Therefore, it is essential to have a comprehensive understanding of software
stability and the key concepts, principles, and architectures that impact software
stability.

This chapter begins by examining the concepts of software stability, evolvability, and
modularity, highlighting their significance in achieving software stability in \gls{ns}.
This is followed by a brief overview of the design theorems and proposed architecture of
\gls{ns}.

The subsequent sections of the thesis explore the fundamental principles that underlie
\gls{ca}, as well as its proposed architectural designs. Finally, the thesis
concludes by discussing which aspects of \gls{ca} align with the principles of
\gls{ns} and contribute to achieving software stability in this approach.

\input{chapters/theoreticalbackgroud/sections/normalizedsystems/normalizedsystems}
\input{chapters/theoreticalbackgroud/sections/cleanarchitecture/cleanarchitecture}
