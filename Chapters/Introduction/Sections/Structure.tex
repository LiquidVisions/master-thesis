\section{Thesis outline} \label{sec_structure}

The structure of this thesis reflects the research methodology described in the previous
section \ref{sec_research_method}. Chapter \ref{chap_theoreticalbackground} presents the
theoretical backgrounds of both \gls{ns} and \gls{ca}, discussing important
characteristics and requirements of software stability, as well as the principles and
architectures proposed by both development approaches.

Chapter \ref{chap_requirements} focuses on the requirements relevant to this research. It
is divided into two sections: section \ref{sec_research_requirements} outlines the
research requirements, describing the requirements necessary for conducting the research,
while section \ref{sec_artifact_requirements} details the artifact requirements, laying
out the requirements relevant to the artifacts contributing to this research.

Chapter \ref{chap_evaluation} evaluates the research results, discussing the impact of
using CA on NS. Notable experiences and findings in Chapter \fullref{chap_discussion}. The
conclusion of this research is presented in the final chapter, Chapter \ref{chap_conclusions}.

Lastly, it is worth mentioning that this thesis follows the guidelines of the American
Psychological Association (APA) style, including the use of US spelling.