\chapter{Introduction} \label{introduction}

\enquote{\emph{Pantha Rhei}} is, according to \emph{Plato}, one of the famous
philosophical statements first described by the Greek philosopher
\emph{Heraclitus}\footnote{\url{https://plato.stanford.edu/entries/process-philosophy/}}.
His statement unambiguously describes the dynamics of everything that exists. The
\enquote{flux of life} is one of the constants in life and can be applied to contemporary
corporate environments where change is continuously introduced at an ever-increasing pace.
These changes lead to an evolution of requirements impacting the evolvability,
maintainability and quality of Software artifacts.

The \enquote{laws of software evolution} \parencite[]{lehman_programs_1980} refers to a
series of laws that have a deteriorating effect on the evolvability of software.
\citeauthor{lehman_programs_1980} describes the balance between the forces driving new
requirements on the one hand, and the forces that slow down progress on the other hand.
Changing software leads to deterioration of the maintainability, impacting the
evolvability and possibly also the quality of these software systems. More than a half of
century of software engineering-, and architecture practices show that the complexity of
these software artifacts gradually increases over time. Eventually, this will render most
of the software artifacts obsolete, according to \citeauthor{lehman_programs_1980}
\parencite[]{lehman_programs_1980}.

Over time there have been many attempts to solve the deterioration of Software Artifact,
some of which with scientific backgrounds. Even before the publication of
\citeauthor{lehman_programs_1980} laws of evolution, McIlroy proposed a vision where
the systematic reuse of software building blocks leads to negative programming practices
where software changes eventually lead to a reduction of complexity. Parnas continued with
the principle of information hiding that is the foundation of modular software architectures 

\section{Research Problem: The plethora of proposed solutions}
\label{sec:research_problem}

There is a wide variety of proposed solutions available for the challenges that occur in
modular and evolvable software architecture. There are plentiful experiences documented
throughout professional and personal blog posts on the internet. Many of those experiences
have mixed outcomes, some are opinionated and results are sometimes based on improper
interpretations of the proposed solutions.

Deciding on the best fit for one of the solutions is a recurring and often difficult task
for software architects. A popular and widely accepted solution from software engineering
literature is Clean Architecture. There is a wide supporting community and many corporate
solutions move toward architectures that have similarities with the Clean Architecture
approach. 

An architecture that derives from science and empirical evidence is Normalized Systems
\parencite*{mannaert_normalized_2009,mannaert_normalized_2016}. Deciding between the two
approaches can be a challenging task on which there is very little documentation and
research. Could it even be the case that combining the two approaches can be an approach
that leads to a highly modular and evolvable software artifact? Let's start with a small
introduction of both approaches.

\subsection{On Normalized Systems Theorems} \label{subsec:intro_to_ns}

The Normalized Systems theorems are a scientific approach to creating software systems
based on the laws for software evolvability. These theorems have resulted in a documented
track record of achieving software stability, in a scientific environment. Effectively, it
prevents the accumulation of combinatorial effects on anticipated change drivers. This
prevents the positive feedback loop and prevents the degradation of the software artifact.
Preventing positive feedback loops has a positive effect on the evolvability of software
artifacts.\parencite[]{mannaert_normalized_2009}. 

\citeauthor[]{mannaert_normalized_2009} have formulated the theorem of Normalized Systems
as prescriptive structures (elements) that lead to a modular architecture with low
coupling and high cohesion. The resulting software architecture will be designed to cope
with future change \parencites[]{mannaert_normalized_2009}.

\subsection{On Clean Architecture} \label{subsec:into_to_ca}

Clean Architecture is the accumulation of more than half a century of coding, designing,
and architecting software systems by \citeauthor*[]{martin_clean_2018}. He published his
experience in his book \citetitle*[]{martin_clean_2018} in \citeyear[]{martin_clean_2018}.
In this book, he states that creating a software artifact does not require that much skill
and knowledge. However, creating stable and evolvable software artifacts is a skill that
requires a lot of knowledge, skill, dedication, and time.

The book's goal is to have a software architecture that minimizes the human resources
required to build and maintain the information system. Just like Normalized Systems, it
has a prescribed design of software classes that will lead to a modular architecture with
low coupling and high cohesion \parencite{martin_clean_2018}.
\section{Hypothesised outcome} \label{hypothesis} 

The proposed hypothesis is that the \gls{ca} approach can be applied in conjunction with
the \gls{ns} theorems. The design principles of \gls{ca} converge with the theorems of
\gls{ns}. Consequently, the artifact that is part of this design research will lead to a
highly modular, stable, and evolvable C\# artifact that does not contradict Normalized
Systems theorems.

Both architectural approaches formulate modular structures independent of any programming
technology \parencite{mannaert_normalized_2009,robert_c_martin_clean_2018}. As such, the
C\# artifact produced as part of this research has similar trademarks of modularity,
evolvability, and stability compared to case studies where Java SE has been used
\parencite{oorts_building_2014, de_bruyn_enabling_2018}. Furthermore, the applicability of
\gls{ca} has no additional or negative effect when used in conjunction with the
\gls{ns} Theorems.

\begin{figure}[H]
    \centering
    \includegraphics[width=0.8\textwidth]{Figures/hypothesis.pdf}
    \caption[The hypothesis]{The hypothesis}
    \label{fig_hypothesis}
\end{figure}
\section{Research question} \label{sec:research_questions}
In order to test the hypothesis described in \ref{hypothesis} the following research
questions can be applied to both research objectives.

\subsubsection*{Research Question 1:}
What violations of the Normalized Systems Theorems can be found as an effect of
following the Clean Architecture principles on both the expander- and expanded artifact?

\subsubsection*{Research Question 2:}
Which of the Normalized Systems Theorems are excluded as an effect of following the
Clean Architecture principles on both the expander- and expanded artifact?

When no violations or excludes on the Normalized Systems Principles are found in the
artifacts, it is assumed that there will be no combinatorial ripple effect on changing
both artifacts. Only in that case, it can be concluded that Clean Architecture can be
used in conjunction with the Normalized Systems theorems. 
\section{Research model} \label{research_model}

When decomposing the statements made in section \ref{sec:research_relevance}
\nameref{sec:research_relevance} and the \nameref{sec:research_questions} the cause and
effect relationship between the independent-, and dependent variable can be determined. It
is clear that the independent variable 'Clean Architecture' has an effect on the outcome
of the design article that has been described in \ref{chap:artifact_design}
\nameref{chap:artifact_design}. 

Since there is plentiful scientific proof, the Normalized Systems theorems are used as the
baseline to measure the results. In the overall conceptual research framework, Normalized
Systems Theorems are considered to be the Moderator variable. In the context of this
research, Clean Architecture does not have a causal effect on Normalized Systems.

Since this research intends to demonstrate the level of convergence of Clean Architecture
with Normalized Systems, modularity, stability and evolvability are positioned as the
Mediator variable in this conceptual framework.

Figure \ref{fig:conceptual_framework} \nameref{fig:conceptual_framework} depicts a
graphical representation of the overall conceptual framework described above.

\begin{figure}[H]
    \centering
    \includegraphics[width=1\textwidth]{Figures/conceptual_framework}
    \caption[Overall conceptual framework]{Overall conceptual framework}
    \label{fig:conceptual_framework}
\end{figure}
