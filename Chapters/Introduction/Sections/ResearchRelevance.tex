\section{Research relevance} \label{sec:research_relevance} 


A lot of research has been done about the modularity, stability and evolvability of
software artifacts. Specifically in the \todo{finish this part}

In chapter~\ref{ns_theory}~\nameref{ns_theory} the Normalized Systems Theorems are
described to be grounded in science. Normalized Systems prescribe modular software
architectures by applying a strict set of design principles. Throughout extensive research
and practical experience, there is both scientific and practical proof that Normalized
Systems contribute to the stability and evolvability of a software artifact.

Clean Architecture can be described as a software architecture philosophy that has emerged
from the experience of its author. According to the author, a highly maintainable and
modular software architecture can be achieved by following the design principles and
applying architectural rules to the software artifact.

Since the introduction of Normalized Systems Theorems, Java EE (Java Enterprise Edition)
has been the prevalent programming language used in scientific research settings. It has
been used to create the evolvability of software architectures based on the stability
concepts of systems theory \parencite[]{mannaert_towards_2012}.

\citeauthor{mannaert_towards_2012} stated in the paper that the design theorems were
formulated as modular structures that are independent of any software language or
development paradigm. The applicability of Normalized Systems Theorems with

Java EE is still a prevalent programming language for enterprise- and IT organizations.
Many software solutions are created and maintained using this programming language.