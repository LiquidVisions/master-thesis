\textbf{Single Responsibility Principle}

There are various manifestations of \gls{srp} implemented in the artifacts. One of which is
already mentioned in Figure \ref{fig_modulair_components}, where \gls{srp} is applied to
separate the domain logic from the application, infrastructure and presentation logic. One
could argue that this manifestation is more related to \gls{soc}, considering the high
granularity of the components.

A better example is the separation of handlers that are part of the \gls{ca}
Expander. Each of those handlers executes an isolated part of the expanding process.
Consider the Listing \ref{list_entityexpander} \nameref{list_entityexpander}
\parencite{koks_expandentitieshandlerinteractor_2023} for example. This Handler is solely
responsible for the generation of data entities. 

\begin{figure}[H]
    \centering
    \includegraphics[width=0.6\textwidth]{figures/expander_handlers.pdf}
    \caption[handlers]{Each of the handlers handles an isolated part of the expanding process.}
    \label{fig_handlers}
\end{figure}

\lstinputlisting[
    caption={The \citetitle{koks_expandentitieshandlerinteractor_2023}},
    label={list_entityexpander}]
    {Snippets/ExpandEntitiesHandlerInteractor.cs}

    \textbf{Open/Closed Principle}
    
A relevant manifestation of \gls{ocp} are all the different implementations of expander
handlers in figure \ref{fig_handlers}. The availability of the
\code{koks_iexpanderhandlerinteractor_2023} interface makes it possible to add more
functionality to the CleanArchtictureExpander without modifying any existing
implementation. New handlers are added by extension, and when implemented correctly, the
handler is automatically executed in the desired order and the required conditions.

\lstinputlisting[
    caption={The \citetitle{koks_iexpanderhandlerinteractor_2023}},
    label={list_iexpanderhandlerinteractor} ]
    {Snippets/IExpanderHandlerInteractor.cs}

\lstinputlisting[
    caption={The \citetitle{koks_iexecutioninteractor_2023}},
    label={list_iexecutioninteractor} ]
    {Snippets/IExecutionInteractor.cs}

The fact that \citecode{koks_iexpanderhandlerinteractor_2023} derives from
\citecode{koks_iexecutioninteractor_2023} is another manifestation of \gls{ocp}. This
design decision allows for object types that need to be treated as executables by the
\code{koks_codegeneratorinteractor_2023}. Examples are
\citecode{koks_regionharvesterinteractor_2023},
\citecode{koks_regionrejuvenatorinteractor_2023},
\citecode{koks_preprocessorinteractor_2023} and
\citecode{koks_postprocessorinteractor_2023}. 

Listing \ref{list_CodeGeneratorInteractor} shows the
\code{koks_codegeneratorinteractor_2023} that cohesively executes all of the
\code{koks_iexecutioninteractor_2023} in order. The software engineer only has to focus on
implementing the specific type of \code{koks_iexecutioninteractor_2023} without having to
affect the implementation. This is by definition an example of \enquote{open for
extension} and \enquote{closed for modifications}.