\section{Discussion}

In this research, the convergence between \gls{ca} and \gls{ns} has been thoroughly
investigated. While it has been demonstrated that the convergence between these two
approaches is incomplete, combining both methodologies is highly beneficial for both
\gls{ns} and \gls{ca} for various reasons. The primary advantage of this convergence
lies in the complementary nature of \gls{ca} with \gls{ns}, where each approach provides
strengths that can be leveraged to address a strong architectural design. 

Clean Architecture offers a well-defined, practical, and modular structure for software
development. Its principles, such as SOLID, guide developers in creating maintainable,
testable, and scalable systems. This architectural design approach is highly suitable for
various applications and can be easily integrated with the theoretical foundations
provided by \gls{ns}. 

Conversely, the NS approach offers a more comprehensive theoretical understanding of
achieving stable and evolvable systems. Furthermore, the popularity and widespread
adoption of Clean Architecture in the software development community can benefit
Normalized Systems. As more developers already adopting \acrlong{ca} become more familiar
with \acrlong{ns} and recognize their value to software design. Combining both approaches
will likely lead to increased adoption of \acrlong{ns}.