\section{Research Objectives} \label{sec_research_objectives}

In this Design Science Research, we will shift the focus from Research Questions to
Research Objects. The primary goal of this research is to determine the degree of
convergence of \gls{ca} with the \gls{ns} Theory. In order to achieve this goal, the
research is divided into the following objectives:

\begin{enumerate}
    \item \textbf{Literature Analysis} \\
    Conduct a literature review of \gls{ca} and \gls{ns}, focusing on their
    fundamental elements, principles, and real-world case studies. This review will
    provide a solid foundation for understanding the underlying concepts and their
    practical implications.
    
    \item \textbf{Architectural Desing} \\
    Create an Architectural Design fully and solely based on \gls{ca}. Implement the
    findings of the Literature Review in the Design. This design will be the basis for
    the Artifact Development.

    \item \textbf{Artifact Development} \\
    Construct two artifacts that facilitate the research of the convergence
    between \gls{ca} and \gls{ns} Theories.
    \begin{enumerate}[label*={\arabic*.}]
        
        \item \textbf{The Code Generator and Clean Architecture Expander} \\
        Inspired by \gls{ns}, create a code generator based on the \gls{ca} design. The
        generator will enable the rapid creation of software systems adhering to the
        principles and design of \gls{ca} and allows for efficient examination of their
        characteristics. The Clean Architecture expander expands a RESTful API based on
        the same \gls{ca} design as the code generator. 
        
        \item \textbf{Expanded Clean Architecture artifact} \\
        The expanded artifact will facilitate the analysis of a RESTful API implementation
        and its alignment with the \gls{ca} principles and design.
        
    \end{enumerate}
    
    \item \textbf{Analysis of combinatorics} \\
    Analyze the artifacts to determine if any combinatorial effects occur due to following
    the principles and architectural approach of \gls{ca}.
\end{enumerate}