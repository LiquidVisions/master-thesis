\documentclass[
11pt, % The default document font size, options: 10pt, 11pt, 12pt
%oneside, % Two side (alternating margins) for binding by default, uncomment to switch to one side
english, % ngerman for German
singlespacing, % Single line spacing, alternatives: onehalfspacing or doublespacing
%draft, % Uncomment to enable draft mode (no pictures, no links, overfull hboxes indicated)
%nolistspacing, % If the document is onehalfspacing or doublespacing, uncomment this to set spacing in lists to single
%liststotoc, % Uncomment to add the list of figures/tables/etc to the table of contents
%toctotoc, % Uncomment to add the main table of contents to the table of contents
parskip, % Uncomment to add space between paragraphs
%nohyperref, % Uncomment to not load the hyperref package
headsepline, % Uncomment to get a line under the header
chapterinoneline, % Uncomment to place the chapter title next to the number on one line
%consistentlayout, % Uncomment to change the layout of the declaration, abstract and acknowledgements pages to match the default layout
]{MastersDoctoralThesis} % The class file specifying the document structure
\usepackage{url}
\usepackage[utf8]{inputenc} % Required for inputting international characters
\usepackage[T1]{fontenc} % Output font encoding for international characters
\usepackage{lipsum}
\usepackage{mathpazo} % Use the Palatino font by default 
\usepackage{fltrace}
\usepackage{float}
\usepackage{hyperref}
\usepackage{tcolorbox}
\usepackage{forest}
\usepackage{rotating}
\usepackage[font=scriptsize]{caption}
\usepackage[colorinlistoftodos,textwidth=30mm,textsize=tiny]{todonotes}
\usepackage[style=authoryear,maxbibnames=9,maxcitenames=2,uniquelist=false,backend=biber]{biblatex} % Use the bibtex backend with the authoryear citation style (which resembles APA)

\usepackage[autostyle=true]{csquotes} % Required to generate language-dependent quotes in the bibliography
\usepackage[acronym, toc]{glossaries}
\usepackage{fontawesome}
\makeglossaries
\loadglsentries{Pages/Abbreviations.tex}
\let\cleardoublepage=\clearpage

%----------------------------------------------------------------------------------------
%  code listings
%----------------------------------------------------------------------------------------
\usepackage{listings}
\usepackage{etoolbox}
\usepackage{color}

\definecolor{base0}{RGB}{131,148,150}
\definecolor{base01}{RGB}{88,110,117}
\definecolor{base2}{RGB}{238,232,213}
\definecolor{sgreen}{RGB}{133,153,0}
\definecolor{sblue}{RGB}{38,138,210}
\definecolor{scyan}{RGB}{42,161,151}
\definecolor{smagenta}{RGB}{211,54,130}
\definecolor{mygray}{gray}{0.98}


\newcommand\digitstyle{\color{smagenta}}
\newcommand\symbolstyle{\color{base01}}
\makeatletter
\newcommand{\ProcessDigit}[1]
{%
  \ifnum\lst@mode=\lst@Pmode\relax%
   {\digitstyle #1}%
  \else
    #1%
  \fi
}
\makeatother


\lstdefinestyle{solarizedcsharp} {
  language=[Sharp]C,
  frame=lr,
  breaklines=true,
  tabsize=2,
  numbers=left,
  numbersep=5pt,
  firstnumber=auto,
  numberstyle=\tiny\ttfamily\color{base0},
  rulecolor=\color{base2},
  backgroundcolor=\color{mygray},
  basicstyle=\scriptsize\ttfamily,
  commentstyle=\color{base01},
  morecomment=[s][\color{base01}]{/*+}{*/},
  morecomment=[s][\color{base01}]{/*-}{*/},
  morekeywords={  abstract, event, new, struct,
                as, explicit, null, switch,
                base, extern, object, this,
                bool, false, operator, throw,
                break, finally, out, true,
                byte, fixed, override, try,
                case, float, params, typeof,
                catch, for, private, uint,
                char, foreach, protected, ulong,
                checked, goto, public, unchecked,
                class, if, readonly, unsafe,
                const, implicit, ref, ushort,
                continue, in, return, using,
                decimal, int, sbyte, virtual,
                default, interface, sealed, volatile,
                delegate, internal, short, void,
                do, is, sizeof, while,
                double, lock, stackalloc,
                else, long, static,
                enum, namespace, string, var},
  keywordstyle=\bfseries\color{sgreen},
  showstringspaces=false,
  stringstyle=\color{scyan},
  identifierstyle=\color{sblue},
  extendedchars=true,
  captionpos=b,
}

\lstset{style=solarizedcsharp}


%----------------------------------------------------------------------------------------
%	RANDOM COMMANDS USED IN THE THESIS
%----------------------------------------------------------------------------------------
\newcolumntype{P}[1]{>{\centering\arraybackslash}p{#1}}
\newcolumntype{M}[1]{>{\centering\arraybackslash}m{#1}}
\newcommand{\ns}{\gls{ns}}
\newcommand{\ca}{\gls{ca}}
\newcommand{\fullref}[1]{\ref{#1} \nameref{#1}}
\newcommand{\code}[1]{\textcolor{sblue}{\textit{\citetitle{#1}}}}
\newcommand{\citecode}[1]{\textcolor{sblue}{\citetitle{#1} \parencite{#1}}}
\newcommand{\mycolorbox}[2]{\begin{tcolorbox}[boxrule=0.1pt, colback=mygray, title={#2} ,colbacktitle=gray]
  \textit{#1}
\end{tcolorbox}}

%----------------------------------------------------------------------------------------
%	FONT ICON COMMANDS
%----------------------------------------------------------------------------------------
\newcommand{\converges}{\faPlus\faPlus}
\newcommand{\diverges}{\faMinus}
\newcommand{\supports}{\faPlus}


\newcommand{\strongConvergence}{\faPlus\faPlus}
\newcommand{\someConvergence}{\faPlus}
\newcommand{\noConvergence}{\faMinus}
%----------------------------------------------------------------------------------------
%	MARGIN SETTINGS
%----------------------------------------------------------------------------------------
\geometry{
	paper=a4paper, % Change to letterpaper for US letter
	inner=2.5cm, % Inner margin
	outer=3.8cm, % Outer margin
	bindingoffset=.5cm, % Binding offset
	top=1.5cm, % Top margin
	bottom=1.5cm, % Bottom margin
	%showframe, % Uncomment to show how the type block is set on the page
}

%----------------------------------------------------------------------------------------
%	THESIS INFORMATION
%----------------------------------------------------------------------------------------

\NewDocumentCommand{\promo}{m}{\newcommand{\promotor}{#1}}
\NewDocumentCommand{\coproone}{m}{\newcommand{\firstco}{#1}}
\NewDocumentCommand{\coprotwo}{m}{\newcommand{\secondco}{#1}}

\thesistitle{On the convergence of Clean Architecture with the Normalized Systems Theorems} % \ttitle
\subject{A Design Science approach of modularity,\\ stability and evolvability of a C\#
software artifact.} % \subjectname

\promo{Prof. Dr. Ing. Hans Mulder}
\coproone{Dr. ir. Geert Haerens}
\coprotwo{Frans Verstreken, MSc}
\degree{Master of Enterprise IT Architecture}
\author{Gerco Koks}

\keywords{} % Keywords for your thesis, this is not currently used anywhere in the template, print it elsewhere with \keywordnames
\university{Antwerp Management School}
%\department{\href{http://department.university.com}{Department or School Name}}
%\group{\href{http://researchgroup.university.com}{Research Group Name}}
\faculty{Master of Enterprise IT Architecture}

\AtBeginDocument{
\hypersetup{pdftitle=\ttitle} % Set the PDF's title to your title
\hypersetup{pdfauthor=\authorname} % Set the PDF's author to your name
\hypersetup{pdfkeywords=\keywordnames} % Set the PDF's keywords to your keywords
\hypersetup{citecolor=., linkcolor=., urlcolor=.}
}
\pdfinfo{
	/title=\ttitle
}

%----------------------------------------------------------------------------------------
%	Bibliography
%----------------------------------------------------------------------------------------
\setcounter{biburllcpenalty}{7000}
\setcounter{biburlucpenalty}{8000}
\addbibresource{bibliography.bib} % The filename of the bibliography

%----------------------------------------------------------------------------------------
%	Enabling a treeview used in the appendix
%----------------------------------------------------------------------------------------
\definecolor{folderbg}{RGB}{124,166,198}
\definecolor{folderborder}{RGB}{110,144,169}

\def\Size{4pt}
\tikzset{
      folder/.pic={
        \filldraw[draw=folderborder,top color=folderbg!50,bottom color=folderbg]
          (-1.05*\Size,0.2\Size+5pt) rectangle ++(.75*\Size,-0.2\Size-5pt);  
        \filldraw[draw=folderborder,top color=folderbg!50,bottom color=folderbg]
          (-1.15*\Size,-\Size) rectangle (1.15*\Size,\Size);
      }
    }

\begin{document}
\frontmatter % Use roman page numbering style (i, ii, iii, iv) for the pre-content pages

\pagestyle{plain}

%----------------------------------------------------------------------------------------
%	PAGES
%----------------------------------------------------------------------------------------
\begin{titlepage}
    \begin{center}
    
    \vspace*{.06\textheight}
    {\scshape\LARGE \univname\par \small Master's Curriculum Program 2021 - 2023}\vspace{1.5cm} % University name
    
    {\huge \bfseries \ttitle\par}\vspace{0.4cm} % Thesis title
    {\emph{\large \subjectname}}\vspace{2.4cm}
     
    \begin{minipage}[t]{0.4\textwidth}
    \begin{flushleft} \large
    \textbf{Author:}\\
    \authorname % Author name - remove the \href bracket to remove the link
    \end{flushleft}
    \end{minipage}
    \begin{minipage}[t]{0.4\textwidth}
    \begin{flushright} \large
    \textbf{Supervisor:} \\
    \supervisorname\vspace{0.2cm} \\
    \textbf{Promotor:} \\
    \promotor\vspace{0.2cm} \\
    \end{flushright}
    \end{minipage}\\[3cm]
     
    \vfill
    
    \large \textit{A thesis submitted in fulfillment of the requirements\\ for the degree
    of \degreename  }\\[0.3cm] % University requirement text
    
     
    
    \vfill
    
    {\large \today}\\[4cm] % Date
    %\includegraphics{Logo} % University/department logo - uncomment to place it
     
    \vfill
    \end{center}
\end{titlepage}
\include{Pages/Declaration.tex}
\vspace*{0.2\textheight}

\noindent\enquote{\itshape I have not failed. Instead, I have found 10.000 ways that won't work.}\bigbreak

\hfill Thomas Edison

\noindent\enquote{\itshape Success is not final, failure is not fatal: it is the courage to continue that count.}\bigbreak

\hfill Winston Churchill

\noindent\enquote{\itshape In all chaos there is a cosmos, in all disorder a secret order.}\bigbreak

\hfill Carl Jung
\thispagestyle{plain}
\begin{center}
	\Large
	\textbf{\ttitle}
	
	\vspace{0.4cm}
	\large
	\subjectname
	
	\vspace{0.4cm}
	\textbf{\authorname}
	
	\vspace{0.9cm}
	\textbf{Abstract}
\end{center}

\small \noindent \lipsum[2-4]
\addcontentsline{toc}{chapter}{Acknowledgements} % Add Table of Contents to the Table of Confents
\chapter*{Acknowledgements}
\small

\begin{center}
\enquote{\textit{With unwavering commitment, heartfelt encouragement, \\and steadfast support, the
impossible becomes possible.}}
\end{center}

Looking back, I never imagined I would embark on this academic journey. As a youngling, I
struggled with a focusing disorder, which led me to attend a school specializing in
helping kids with similar challenges. Despite this, I discovered I could excel by
dedicating myself to enjoyable activities. Early in my life, this manifested primarily in
sports.

I knew I had to provide for myself as an adult, so I pursued an education in what I loved
most: Sports. Consequently, the first decade of my professional career revolved around
being a sports instructor. Eventually, my interests shifted toward the technical
challenges of software programming. I decided to take the plunge and pursue it
professionally. So, I embarked on my first educational endeavor, my bachelor's degree in
Computer Science, completed in 2012. In hindsight, it was the best decision I have made.
It caused a giant leap forward from a career perspective but also for my personal growth.
My decision has brought me to this point where I am close to completing my Master's
degree, which makes me incredibly proud. 

It would have been a much more difficult journey without the encouragement and support of
many people. Rene Bliekendaal encouraged and convinced me to embark on this journey. Thank
you! I also greatly respect my employer for facilitating this pursuit while I continued to work
full-time. Your flexibility has played an important role in making it a reality.

I sincerely thank my supervisor, Hans Mulder, and promoter, Geert Haerens, for their
invaluable guidance, feedback, and mentorship throughout my research. Your expertise and
insights have been instrumental in shaping my work and ensuring this success. Also, a
special mention goes to Frans Vertreken, who really stepped up in guiding me in the early
stages of research when no one else was available. Your knowledge and encouragement have
been inspiring and laid the basis for my result.

I extend my heartfelt gratitude to Antwerp Management School, which provided me with an
excellent platform to pursue my personal and scholarly growth. To everyone else who has
played a role in this journey, no matter how big or small, I offer my heartfelt thanks.
This achievement would not have been possible without each one of you.

So, what is next for me? First, I fully intend to repay my employer's continued support
and encouragement by sharing my knowledge and planning to bring additional value. Then, in
my free time, I'll spend some well-deserved and needed time focusing on my health and my
closest family members. After that? I'm not sure yet, but I want to explore more on the
subject of Software stability and Evolvability, as this is where my true passion and
potential lie. 

\noindent --- Gerco
\tableofcontents % Prints the main table of contents
\listoffigures % Prints the list of figures
\listoftables % Prints the list of tables
\printglossary[title=List of Abbreviations, toctitle=List of Abbreviations, type=\acronymtype, nonumberlist]
%\clearpage
\printglossary[title=Abbreviations, toctitle=Abbreviations, type=\acronymtype, nonumberlist]
\include{Pages/Dedications.tex}
 
%----------------------------------------------------------------------------------------
%	THESIS CONTENT - CHAPTERS
%----------------------------------------------------------------------------------------
\mainmatter % Begin numeric (1,2,3) page numbering
\pagestyle{thesis} % Return the page headers back to the "thesis" style

\chapter{Introduction} \label{introduction}

\enquote{\emph{Pantha Rhei}} is, according to \emph{Plato}, one of the famous
philosophical statements first described by the Greek philosopher
\emph{Heraclitus}\footnote{\url{https://plato.stanford.edu/entries/process-philosophy/}}.
His statement unambiguously describes the dynamics of everything that exists. The
\enquote{flux of life} is one of the constants in life and can be applied to contemporary
corporate environments where change is continuously introduced at an ever-increasing pace.
These changes lead to an evolution of requirements impacting the evolvability,
maintainability and quality of Software artifacts.

The \enquote{laws of software evolution} \parencite[]{lehman_programs_1980} refers to a
series of laws that have a deteriorating effect on the evolvability of software.
\citeauthor{lehman_programs_1980} describes the balance between the forces driving new
requirements on the one hand, and the forces that slow down progress on the other hand.
Changing software leads to deterioration of the maintainability, impacting the
evolvability and possibly also the quality of these software systems. More than a half of
century of software engineering-, and architecture practices show that the complexity of
these software artifacts gradually increases over time. Eventually, this will render most
of the software artifacts obsolete, according to \citeauthor{lehman_programs_1980}
\parencite[]{lehman_programs_1980}.

Over time there have been many attempts to solve the deterioration of Software Artifact,
some of which with scientific backgrounds. Even before the publication of
\citeauthor{lehman_programs_1980} laws of evolution, McIlroy proposed a vision where
the systematic reuse of software building blocks leads to negative programming practices
where software changes eventually lead to a reduction of complexity. Parnas continued with
the principle of information hiding that is the foundation of modular software architectures 

\section{Research Problem: The plethora of proposed solutions}
\label{sec:research_problem}

There is a wide variety of proposed solutions available for the challenges that occur in
modular and evolvable software architecture. There are plentiful experiences documented
throughout professional and personal blog posts on the internet. Many of those experiences
have mixed outcomes, some are opinionated and results are sometimes based on improper
interpretations of the proposed solutions.

Deciding on the best fit for one of the solutions is a recurring and often difficult task
for software architects. A popular and widely accepted solution from software engineering
literature is Clean Architecture. There is a wide supporting community and many corporate
solutions move toward architectures that have similarities with the Clean Architecture
approach. 

An architecture that derives from science and empirical evidence is Normalized Systems
\parencite*{mannaert_normalized_2009,mannaert_normalized_2016}. Deciding between the two
approaches can be a challenging task on which there is very little documentation and
research. Could it even be the case that combining the two approaches can be an approach
that leads to a highly modular and evolvable software artifact? Let's start with a small
introduction of both approaches.

\subsection{On Normalized Systems Theorems} \label{subsec:intro_to_ns}

The Normalized Systems theorems are a scientific approach to creating software systems
based on the laws for software evolvability. These theorems have resulted in a documented
track record of achieving software stability, in a scientific environment. Effectively, it
prevents the accumulation of combinatorial effects on anticipated change drivers. This
prevents the positive feedback loop and prevents the degradation of the software artifact.
Preventing positive feedback loops has a positive effect on the evolvability of software
artifacts.\parencite[]{mannaert_normalized_2009}. 

\citeauthor[]{mannaert_normalized_2009} have formulated the theorem of Normalized Systems
as prescriptive structures (elements) that lead to a modular architecture with low
coupling and high cohesion. The resulting software architecture will be designed to cope
with future change \parencites[]{mannaert_normalized_2009}.

\subsection{On Clean Architecture} \label{subsec:into_to_ca}

Clean Architecture is the accumulation of more than half a century of coding, designing,
and architecting software systems by \citeauthor*[]{martin_clean_2018}. He published his
experience in his book \citetitle*[]{martin_clean_2018} in \citeyear[]{martin_clean_2018}.
In this book, he states that creating a software artifact does not require that much skill
and knowledge. However, creating stable and evolvable software artifacts is a skill that
requires a lot of knowledge, skill, dedication, and time.

The book's goal is to have a software architecture that minimizes the human resources
required to build and maintain the information system. Just like Normalized Systems, it
has a prescribed design of software classes that will lead to a modular architecture with
low coupling and high cohesion \parencite{martin_clean_2018}.
\section{Hypothesised outcome} \label{hypothesis} 

The proposed hypothesis is that the \gls{ca} approach can be applied in conjunction with
the \gls{ns} theorems. The design principles of \gls{ca} converge with the theorems of
\gls{ns}. Consequently, the artifact that is part of this design research will lead to a
highly modular, stable, and evolvable C\# artifact that does not contradict Normalized
Systems theorems.

Both architectural approaches formulate modular structures independent of any programming
technology \parencite{mannaert_normalized_2009,robert_c_martin_clean_2018}. As such, the
C\# artifact produced as part of this research has similar trademarks of modularity,
evolvability, and stability compared to case studies where Java SE has been used
\parencite{oorts_building_2014, de_bruyn_enabling_2018}. Furthermore, the applicability of
\gls{ca} has no additional or negative effect when used in conjunction with the
\gls{ns} Theorems.

\begin{figure}[H]
    \centering
    \includegraphics[width=0.8\textwidth]{Figures/hypothesis.pdf}
    \caption[The hypothesis]{The hypothesis}
    \label{fig_hypothesis}
\end{figure}
\section{Research question} \label{sec:research_questions}
In order to test the hypothesis described in \ref{hypothesis} the following research
questions can be applied to both research objectives.

\subsubsection*{Research Question 1:}
What violations of the Normalized Systems Theorems can be found as an effect of
following the Clean Architecture principles on both the expander- and expanded artifact?

\subsubsection*{Research Question 2:}
Which of the Normalized Systems Theorems are excluded as an effect of following the
Clean Architecture principles on both the expander- and expanded artifact?

When no violations or excludes on the Normalized Systems Principles are found in the
artifacts, it is assumed that there will be no combinatorial ripple effect on changing
both artifacts. Only in that case, it can be concluded that Clean Architecture can be
used in conjunction with the Normalized Systems theorems. 
\section{Research model} \label{research_model}

When decomposing the statements made in section \ref{sec:research_relevance}
\nameref{sec:research_relevance} and the \nameref{sec:research_questions} the cause and
effect relationship between the independent-, and dependent variable can be determined. It
is clear that the independent variable 'Clean Architecture' has an effect on the outcome
of the design article that has been described in \ref{chap:artifact_design}
\nameref{chap:artifact_design}. 

Since there is plentiful scientific proof, the Normalized Systems theorems are used as the
baseline to measure the results. In the overall conceptual research framework, Normalized
Systems Theorems are considered to be the Moderator variable. In the context of this
research, Clean Architecture does not have a causal effect on Normalized Systems.

Since this research intends to demonstrate the level of convergence of Clean Architecture
with Normalized Systems, modularity, stability and evolvability are positioned as the
Mediator variable in this conceptual framework.

Figure \ref{fig:conceptual_framework} \nameref{fig:conceptual_framework} depicts a
graphical representation of the overall conceptual framework described above.

\begin{figure}[H]
    \centering
    \includegraphics[width=1\textwidth]{Figures/conceptual_framework}
    \caption[Overall conceptual framework]{Overall conceptual framework}
    \label{fig:conceptual_framework}
\end{figure}

\chapter{Theoretical background} \label{chap:theoreticalbackground} 

\gls{ns} is a software development approach that prioritizes achieving software stability
through the use of standardized, modular components and interfaces. This theory is
informed by several scientific disciplines, including systems theory, mathematics, and
computer science, as well as some other software development approaches such as agile
development and domain-driven design.

The goal of this thesis is to investigate whether the philosophy of \gls{ca}
aligns with the goals of \gls{ns}. To do so, it is essential to have a
comprehensive understanding of software stability and the key concepts, principles, and
architectures that impact software stability.

This chapter begins by examining the concepts of software stability, evolvability, and
modularity, highlighting their significance in achieving software stability in \gls{ns}.
This is followed by a brief overview of the design theorems and proposed architecture of
\gls{ns}.

The subsequent sections of the thesis explore the fundamental principles that underlie
\gls{ca}, as well as its proposed architecture designs. Finally, the thesis
concludes by discussing which aspects of \gls{ca} align with the principles of
\gls{ns} and contribute to achieving software stability in this approach.

\section{Towards stable software architectures} \label{sec:on_stability}


\gls{ns} originated in the field of software engineering, aiming to achieve modular and
stable software artifacts. However, the underlying theory of \gls{ns} can be applied to
various other domains, such as Enterprise Engineering, Business Process Modeling, and
document management. This research acknowledges the software engineering background of
gls{ns}. It consistently refers to software and Information Systems when referring to
\enquote*{artifacts}. However, the reader should realize that the concepts and artifacts
are not restricted to software artifacts alone.

In several disciplines stability has been defined as \emph{Bounded Input Bounded Output}
(BIBO). It is the fundamental property of a system when subjected to bounded input
disturbances. BIBO stability ensures that the output of a system will also be bounded,
preventing uncontrolled or unexpected behavior \parencite[270]{mannaert_normalized_2016}. 

A real-world example of the importance of stability is the Tacoma Narrows Bridge in
Washington State, USA. The bridge, depicted in figure \ref*{fig:bridge}, collapsed on
November the 7th, 1940. This was caused due to wind-induced oscillations called
aeroelastic flutter. The wind (Input) induced oscillations in the bridge, causing it to
start swaying back and forth (Output). These oscillations were initially small, but as
they continued, they began to increase in amplitude or magnitude, causing the bridge to
collapse.

\begin{figure}[H]
    \centering
    \includegraphics[width=0.6\textwidth]{Figures/bridge.pdf}
    \caption[TNB]{Tacoma Narrows Bridge (Galloping Gertie)}
    \label{fig:bridge}
\end{figure}

Stability can also be used in the context of software engineering. In the context of
\gls{ns}, it is considered a critical property that ensures that the software is not
excessively sensitive to small changes \parencite[270]{mannaert_normalized_2016}. New
functional requirements should only lead to fixed, and an expected amount of changes in
the source code. Conversely, instabilities occur when the total number of modifications
relies on the size of the software artifact. The number of changes will grow over time in
parallel with the growth of the software artifact. These instabilities are referred to as
combinatorial effects \parencite[270]{mannaert_normalized_2016}. when combinatorial effects
are absent, the software artifact can be considered evolvable.
\section{On modularisation} \label{sec:on_modules}

Software modules can be defined as self-contained units of code that perform specific
tasks or sets of tasks within a larger system. A software module is designed to operate
independently of other modules, with well-defined interfaces that allow it to communicate
and exchange data with other modules if necessary \autocite[22]{mannaert_normalized_2016}.

A module can be considered a hierarchical and recursive concept. They are independent of
their size (lines of code) or computational magnitude. They can be as small as a function
as part of a class. The class itself can also be considered a module. A group of classes
contained in a Dynamic Link Library (DLL) or Application Programming Interface (API) can
also be considered a module of an even bigger system. 

An important part of the design of a software system is to identify the possible different
modules and their interaction interfaces. There is a wide consensus about two fundamental
rules when thinking of-, and designing modules: \emph{high cohesion} and \emph{low
coupling} \autocite[22]{mannaert_normalized_2016}.
\section{Software evolvability} \label{software_evolvability}

An important aspect of this thesis is to determine the evolvability of software artifacts
with a Clean Architecture design. An evolvable software artifact should not have
instabilities: a bounded amount of additional functional requirements cannot lead to an
unbounded amount of additional (versions of) software primitives \parencite[273]{mannaert_normalized_2009}. 

\subsection{Stability}
<<verwijzing naar positive feedback loops uit de theorie van NS en de effect op
evolueerbaarheid van software.>>

\subsection{Combinatorial effects \& anticipated change drivers.}
<<heeft relatie tot hoofdstuk evaluation en toelichtingen dat software aanpassingen niet
zouden moeten leiden tot een toename in combinatorial effects>>

\subsection{Modularity}
<<toelichten waarop modularity invloed heeft op de evolueerbaarheid van de software>>


\input{Chapters/TheoreticalBackgroud/Sections/Combinatorics.tex}
\section{Cohesion: The Beauty of Software Design}
<<toelichten waarop Cohesion invloed heeft op de evolueerbaarheid van de software>>

Cohesion in software engineering refers to the degree to which the different structural
parts of a software system work together to achieve a single and well-defined purpose, a
common goal. 

There is a considerable body of scientific evidence supporting the importance of software
cohesion. Several studies correlate high levels of cohesion with fewer defects and are
likely to be more maintainable. It has shown that software engineering artifacts are more
open to change. Having a high cohesion is often referred to have a positive effect on
software quality attributes like reliability, maintainability, reusability and thus the
evolvability of the software artifact. 

\subsection{Coupling}
<<toelichten waarop Coupling invloed heeft op de evolueerbaarheid van de software>>
\section{Normalized Systems: Impacting software stability} 
\label{ns_theory}
\begin{itemize}
    \item Inleiding
\end{itemize}


\subsection{The Design Theorems} \label{subsec:ns_desing_theorems}

\gls{ns} Theorems is a theoretical framework of principles that aims to enhance the
stability of a software artifact. \gls{ns} provides a rigorous mathematical foundation
that offers guidelines for designing and developing software systems. The principles of NS
have gained significant attention in both academic and industrial circles due to their
potential to increase the evolvability and stability of software artifacts while reducing
maintenance costs. The theorems have been scientifically established and proven. In the
following sections, we will focus on the theorems of \gls{ns}. When applicable, we
emphasize some manifestations in one of the artifacts to demonstrate the level of
convergence of \gls{ca}.

\subsubsection{Separation of Concerns}
Since the early years of software engineering, \gls{soc} has been one of the most
fundamental software engineering principles. The principle has first been mentioned by
\citeauthor{dijkstra_selected_1982}\footnote{\url{https://en.wikipedia.org/wiki/Separation_of_concerns}}
as the crucial principle to design modular software architecture
\parencite[]{dijkstra_selected_1982}. The concept itself was introduced by
\citeauthor{parnas_criteria_1972} in his book \citetitle*{parnas_criteria_1972}.

This principle promotes the idea that a program should be divided into distinct sections,
each addressing a separate concern or aspect of the problem. This allows for a more
organized and maintainable source code, as changes to one concern do not affect the
others. \gls{soc} should be applied at the level of individual modules, rather that the level
of an entire program.

The \gls{soc} had its effect on later versions of software engineering principles like SOLID.
The principles of 'Single Responsibility' and 'Interface Segregation' are directly derived
from \gls{soc}. It also affected the theorems of Normalized Systems, although it has a more
strict definition of this principle. In the book of \citeauthor{mannaert_normalized_2016}
it is described as followed: 

\begin{center}
    \textbf{Theorem I}\\
    \textit{A processing function can only contain a single task to achieve stability.}    
\end{center}


There are a couple of manifestation examples described like the application of an
integration service bus. The manifestation of external workflows is another example. One
example of an external workflow in the prototype of this research project is the
\ref{SnipSeedingBoundary}. This class is responsible for the provision of the initial data
to the model, based on the elements of the meta-model itself. It 'seeds' the data by
orchestrating individual entity seeders to execute in the correct order.

\lstinputlisting[
    caption=\citeurl{koks_seedingboundary_2023},
    label={SnipSeedingBoundary}]
    {Snippets/Seedingboundary.cs}

\subsubsection{Data version Transparancy}
\gls{dvt} is the act of encapsulation of data entities for specific tasks at hand. This
results in the fact that data structures can have multiple versions often mentioned as
Data Transfer Objects in modern software engineering projects. In other words, it should
be possible to update the data entity without affecting the processing functions. This
leads to the following description of the theorem \parencite[280]{mannaert_normalized_2016}.

\begin{center}
    \textbf{Theorem II}\\
    \textit{A data structure that is passed through the interface of a processing function 
    needs to exhibit version transparency in order to achieve stability.}
\end{center}

\gls{dvt} is widely used in various technological applications. Every web service
currently known supports some type of versioning. In restful APIs for example, it is
common practice to support versioning over the URI. It is best practice to encapsulate
breaking changes in a new version of the endpoint/service so that the consumers are not
(directly) affected by the change. In modern Object Oriented languages, gls{dtv} is also
supported by the ability to determine the scope of visibility of the modifiers of the
various programming constructs like fields, properties, interfaces and classes. Also known
as information hiding \parencites{parnas_criteria_1972}[278]{mannaert_normalized_2016}.

The prototype also uses information hiding very strictly. In order to seal implementations
to the intended layers, concrete implementations always have internal visibility, making
them impossible to use. The interfaces on the other hand are publically exposed. The
dependent layers are now restricted to the appointed dependency injection container of the
specific layer. Alternatively, it is also possible to implement a custom implementation of
the specific interface. 

Another example from the prototype is the use of ViewModels for the queries, and
CommandModels for the commands. Depending on the context of the operation, the containing
fields of the Model may differ. A CommandModel for deleting a data entity will only
contain the key of the entity that needs to be deleted. The CommandModel for creating the
same type of entity will probably contain all the required fields of the data entities,
except the key field as this key is often auto-generated by the database or
domain layer of the application.

\subsubsection{Action version Transparancy}
\gls{avt} is the property of a system to modify existing processing functions without
affecting the existing ones. It should be possible to upgrade a function without affecting
the callers of those functions. This description leads to the following theorem
\parencite[282]{mannaert_normalized_2016}.

\begin{center}
    \textbf{Theorem III}\\
    \textit{A processing function that is called by another processing function, needs to exhibit version transparency in order to achieve stability.}
\end{center}

Most of the modern technology environments support some form of \gls{avt}. Polymorphism is
a widely used technique in order to support this theorem. Specifically, parametric
polymorphism \footnote{\url{https://en.wikipedia.org/wiki/Parametric_polymorphism}} allows
for a processing function to have multiple input parameters. There are also quite some
design patterns supporting this theorem. Some random examples are the state pattern
\footnote{\url{https://en.wikipedia.org/wiki/State_pattern}}, facade pattern
\footnote{\url{https://en.wikipedia.org/wiki/Facade_pattern}} and observer pattern
\footnote{\url{https://en.wikipedia.org/wiki/Observer_pattern}}.

Manifestations in the artifacts are abundant. One example is the
\citetitle{koks_icommandlineinteractor_2023} \parencite{koks_icommandlineinteractor_2023}.
There are multiple signatures of the processing function Start, each of them allowing a
slightly different behavior when executed.

In specific situations \enquote{Dependency Injection} can be viewed as an application of
\gls{avt}. The \citetitle{koks_logger_2023} \parencite{koks_logger_2023} is currently an
implementation of the NLOG framework \footnote{\url{https://nlog-project.org/}}. It is
fairly simple to implement a new logging framework as a replacement for NLOG when the
consuming implementations only use the ILogger interface, instead of the concrete NLOG
wrapper implementation. A warning is in order though. Dependency Injection does not remove
the dependency. It moves the dependency to a different class. When implemented incorrectly
this could even violate the \gls{avt} principle \parencite[213]{mannaert_normalized_2016}.
Therefore it is recommended only to use Dependency Injection in combination with an
\enquote{Inversion Of Control Container} where the dependencies from the viewpoint of a
module are managed centrally.

The use of Generics is another use of Polyphormismadhering to the \gls{avt} theorem.
Consider code snippet \ref{SnipICreateGateway} \nameref{SnipICreateGateway} which is a
generic implementation of the Create repository. This allows consumer code to create
entities without having dependencies on the entity itself, or without having to know the
concrete class.

\lstinputlisting[
    caption={\citetitle{koks_icreategateway_2023}},
    label={SnipICreateGateway}]
    {Snippets/ICreateGateway.cs}

\subsubsection{Separation of State}

\gls{sos} is a theorem that is based on the idea that processing functions should not
contain any state information but instead should rely on external data structures to store
state information. By separating state information from processing functions, Normalized
Systems can achieve a higher level of flexibility and adaptability. External data
structures can be updated or replaced without affecting the processing functions
themselves, which greatly reduces the change of unwanted ripple effects. This theorem is
described as followed: \parencite[258]{mannaert_normalized_2016}.

\begin{center}
    \textbf{Theorem IV}\\
    \textit{Calling a processing function within another processing function, needs to exhibit state keeping in order to achieve stability.}
\end{center}

\gls{sos} fits very well in distributed information systems with asynchronous calls. The
expanded artifact is designed in a manner that all external process functions are executed
asynchronously. \todo{snippet toevoegen.}.

A simpler manifestation of the \gls{sos} theorem involves the use of Resources as an
integral part solution
\footnote{url{https://learn.microsoft.com/en-us/dotnet/core/extensions/resources}}. In
addition to enabling the localization of strings, this approach offers the advantage of
centralized management, thereby exhibiting \gls{sos}. For instance, when the functional
requirements evolve, the name of a template in the expander artifact is likely to change.
As the name of the template is used in multiple class instances, a centralized approach to
managing the template name can mitigate the risk of excessive modifications when a name
change is mandated.

Another example. The state of the model (see chapter \ref{sec:artifact_model}) is
currently persisted in an Azure SQL Database.

\lstinputlisting[
    caption={The \citetitle{koks_genericrepository_2023} \parencite{koks_genericrepository_2023}}]
    {Snippets/GenericRepository.cs}

The state of of the expander artifact model as described in \ref{sec:artifact_model} for the expander in the 
voorbeeld met seerders...
voorbeeld met repositories....



\subsection{Architecture elements} \label{subsec:ns_elements} 

\textbf{Data Element}\\
This is an object that represents a piece of data in the system. Data elements are used to
pass information between processing functions and other objects. In Normalized Systems,
data elements are typically standardized to ensure consistency across the system.

\textbf{Flow Element}\\
This object represents the flow of control through the system. It determines the order in
which processing functions are executed and can be used to handle error conditions or
other exceptional cases.

\textbf{Connector Element}\\
This object is used to connect different parts of the system together. Connectors can be
used to link processing functions, data elements, and other objects, allowing them to work
together seamlessly.

\textbf{Task Element}\\
This is an object that represents a specific task or action in the system. Tasks can be
composed of one or more processing functions and can be used to represent complex
operations within the system.


\chapter{Requirements} \label{chap_requirements} 

This Chapter outlines the requirements for this Design Science Research study, where we
focus on the stability a\& evolvability of Software Artifacts. Section
\ref{sec_requirements_transformation} begins by discussing Software Transformation
Requirements proposed by \textcite{mannaert_normalized_2016}, which serve as a foundation
for assessing the stability \& evolvability of the Artifacts. Next, Section
\ref{sec_artifact_requirements} details the specific requirements of the Artifacts used in
this study. These requirements will help ensure that the Artifacts are suitable for
evaluating the stability \& evolvability of Software Artifacts designed based on Clean
Architecture and SOLID Principles.


\section{software Transformation Requirements} \label{sec_requirements_transformation}

We study stability and evolvability by investigating potential combinatorial effects in
\gls{ca} artifacts. Therefore, during the implementation, we will apply parts of the
Functional-Construction software Transformation from
\textcite[251]{mannaert_normalized_2016} by using the following five proposed Functional
Requirements Specifications. \textcite[254-261]{mannaert_normalized_2016} have defined
them as follows.

\begin{enumerate}[leftmargin=*]
    \item An information system needs to be able to represent instances of
    data entities. A data entity consists of several data fields. Such a field may be a basic
    data field representing a value of a reference to another data entity.
    
    \item An information system needs to be able to execute processing actions on
    instances of data entities. A processing action consists of several consecutive processing
    tasks. Such a task may be a basic task, i.e., a unit of processing that can change
    independently or an invocation of another processing action.
    
    \item An information system needs to be able to input or output values
    of instances of data entities through connectors.
    
    \item An existing information system representing a set of data entities needs to be
    able to represent a new version of a data entity that corresponds to including an
    additional data field and an additional data entity.
    
    \item An existing information system providing a set of processing actions needs to
    be able to provide a new version of a processing task, whose use may be mandatory, a
    new version of a processing action, whose use may be mandatory, an additional
    processing task, and an additional processing action
    
\end{enumerate}
\section{Artifact requirements} \label{sec_artifact_requirements}

Chapter \fullref{sec_research_objectives} outlines the construction of two artifacts. Both
of these artifacts will be meticulously designed and developed in accordance with the
design philosophy and principles of \gls{ca}, ensuring strict adherence to the following
requirements:

\subsection{Component Architecture Requirements}
 
The following requirements are applied to the Component Architecture of both the Generator
Artifact and the Generated Artifact.

\begin{table}[H]
    \begin{tabular}{@{\makebox[2em][c]{\rownumber\space}}  p{0.87\linewidth}}
        \multicolumn{1}{@{\makebox[2em][c]{Nr.}}  p{0.87\linewidth}}{Requirement}\\ 
    \hline
    The solution is organized into separate Visual Studio projects for the Domain,
    Application, Infrastructure, and Presentation layers of the component. A detailed
    description of these layers can be found in Section \fullref{subsec_layers}.
    \\
    The Visual Studio projects representing the component layers comply with the naming
    conventions outlined in the appendix \fullref{appendix_component_naming_convention} \\
       
    The dependencies between the component layers must follow an inward direction towards
    the higher-level components as illustrated in Figure \ref{fig_modulair_components}
    schematically, and cannot skip layers. \\
       \hline
    \end{tabular}
\caption{The Component Architecture Requirements}
\label{table_component_requirements}
\end{table}

\begin{table}[H]
    \begin{tabular}{@{\makebox[2em][c]{\rownumber\space}}  p{0.87\linewidth}}
        \multicolumn{1}{@{\makebox[2em][c]{Nr.}}  p{0.87\linewidth}}{Technology Requirement}\\ 
    \hline
       The Domain and Application layers have no dependencies on any infrastructure technologies, like web- or database
       technologies. 
       \\
       The Presentation Layer relies on various infrastructure technologies for
       facilitating end-user interaction. Examples of such technologies include Command
       Line Interfaces (CLIs), RESTful APIs, and web-based solutions. Each dependency is
       isolated and managed in separate Visual Studio Projects to ensure the stability and
       evolvability of the system. 
       \\
       The Infrastructure Layer may rely on other infrastructure components, such
       as databases or filesystems. Each infrastructure dependency is isolated and managed
       in separate Visual Studio Projects to promote stability and evolvability. 
       \\
       All Component Layers utilize the C\# programming languages, explicitly targeting
       the .NET 7.0 framework. 
       \\
       Reusing existing functionality or technology (packages) is permitted only when
       adhering to the \gls{lsp} and utilizing the open-source package manager, \gls{nuget}.
        \\
       \hline
    \end{tabular}
\caption{Application Layer Technology Requirements}
\label{table_requirements_application_layer_technology}
\end{table}
\subsection{Software Architecture Requirements} \label{software_requirements}

Figure \ref{fig_design} illustrates the generic Software Architecture of the Artifacts.
Each instantiated element adheres to the Element Naming Convention outlined in Appendix
\ref{appendix_element_naming_convention}. In addition, the following tables detail the
requirements specific to each element.

\begin{figure}[H]
    \centering
    \includegraphics[width=1\textwidth]{figures/generic_design.pdf}
    \caption[Generic architecture]{The Generic architecture of the artifacts}
    \label{fig_design}
\end{figure}


\subsubsection*{Presentation Layer}
\begin{table}[H]
    \begin{tabular}{@{\makebox[2em][c]{\rownumber\space}}  p{0.87\linewidth}}
        \multicolumn{1}{@{\makebox[2em][c]{Nr.}}  p{0.87\linewidth}}{ViewModel Requirement}\\ 
        \hline
        The ViewModel consists of data attributes representing fields from the
        corresponding Entity. In addition, it may contain information specific to the user
        interface. \\

        The ViewModel has no external dependencies on other objects within the
        architecture. \\
       
       \hline
    \end{tabular}
\caption{ViewModel Requirements}
\label{table_requirements_viewmodel}
\end{table}

\begin{table}[H]
    \begin{tabular}{@{\makebox[2em][c]{\rownumber\space}}  p{0.87\linewidth}}
        \multicolumn{1}{@{\makebox[2em][c]{Nr.}}  p{0.87\linewidth}}{Presenter Requirement}\\ 
    \hline
    The Presenter Implementation is derived from the IPresenter interface and follows the
    specified implementation. The IPresenter interface can be found in the Application
    Layer. \\   
    
    The Presenter is responsible for creating the Controller's Response by instantiating
    the ViewModel, constructing the HTTP Response message, or combining both elements as
    needed. \\
    
    When required, the Presenter utilizes the IMapper interface without depending on
    specific implementations of the IMapper interface. \\
    
    The Presenter has an internal scope and cannot be instantiated outside of the
    Presentation layer. \\
    \hline
    \end{tabular}
\caption{Presenter Requirements}
\label{table_requirements_presenter}
\end{table}

\begin{table}[H]
    \begin{tabular}{@{\makebox[2em][c]{\rownumber\space}}  p{0.87\linewidth}}
        \multicolumn{1}{@{\makebox[2em][c]{Nr.}}  p{0.87\linewidth}}{ViewModelMapper Requirement}\\ 
    \hline
    The ViewModelMapper is derived from the IMapper interface and follows the specified
    implementation. The IMapper interface can be found in the Application Layer. \\

    The ViewModelMapper is responsible for mapping the values of the necessary data
    attributes from the ResponseModel to the ViewModel. \\
    
    The ViewModelMapper has an internal scope and cannot be instantiated outside of the
    Presentation layer. \\
    \hline
    \end{tabular}
\caption{ViewModelMapper Requirements}
\label{table_requirements_viewModelMapper}
\end{table}

\begin{table}[H]
    \begin{tabular}{@{\makebox[2em][c]{\rownumber\space}}  p{0.87\linewidth}}
        \multicolumn{1}{@{\makebox[2em][c]{Nr.}}  p{0.87\linewidth}}{Controller Requirement}\\ 
    \hline
    The Controller is responsible for receiving external requests and forwarding the
    request to the appropriate Boundary within the Application Layer. \\

    The Controller relies on the IBoundary interface without depending on specific
    implementations of the IBoundary interface. \\
    \hline
    \end{tabular}
\caption{Controller Requirements}
\label{table_requirements_controlle}
\end{table}

\subsubsection*{Application Layer}

\begin{table}[H]
    \begin{tabular}{@{\makebox[2em][c]{\rownumber\space}}  p{0.87\linewidth}}
        \multicolumn{1}{@{\makebox[2em][c]{Nr.}}  p{0.87\linewidth}}{IBoundary Requirements}\\ 
    \hline
    The IBoundary interface establishes the contract for its derived Boundary implementations. \\

    The IBoundary interface has public scope within the system. \\
    \hline
    \end{tabular}
\caption{IBoundary Requirements}
\label{table_requirements_iboundary}
\end{table}

\begin{table}[H]
    \begin{tabular}{@{\makebox[2em][c]{\rownumber\space}}  p{0.87\linewidth}}
        \multicolumn{1}{@{\makebox[2em][c]{Nr.}}  p{0.87\linewidth}}{Boundary Implementation Requirements}\\ 
    \hline
    A Boundary implementation is derived from the IBoundary interface and follows the
    specified implementation. \\

    The Boundary implementation serves as a separation between the internal aspects of the
    Application Layer and the other layers within the Component. \\
    
    Each Boundary implementation handles a single task, which is then
    executed using the IInteractor interface. \\
    
    Boundary implementations have an internal scope and cannot be instantiated outside the
    Application Layer. \\
    \hline
    \end{tabular}
\caption{Boundary Implementation Requirements}
\label{table_requirements_boundary}
\end{table}

\begin{table}[H]
    \begin{tabular}{@{\makebox[2em][c]{\rownumber\space}}  p{0.87\linewidth}}
        \multicolumn{1}{@{\makebox[2em][c]{Nr.}}  p{0.87\linewidth}}{IInteractor Requirements}\\ 
    \hline
    The IInteractor interface establishes the contract for its derived Interactor
    implementations. \\

    The IInteractor has an internal scope and cannot be implemented outside the Application
    Layer. \\
    \hline
    \end{tabular}
\caption{IInteractor Requirements}
\label{table_requirements_iinteractor}
\end{table}

\begin{table}[H]
    \begin{tabular}{@{\makebox[2em][c]{\rownumber\space}}  p{0.87\linewidth}}
        \multicolumn{1}{@{\makebox[2em][c]{Nr.}}  p{0.87\linewidth}}{Interactor Implementation Requirements}\\ 
    \hline
    An Interactor implementation is derived from the IInteractor interface and follows the
    specified implementation. \\

    The Interactor implementation executes a single task or orchestrates a series of
    tasks. Each of these tasks is implemented in separate Interactors. Alternatively, a
    Gateway is used for Tasks with Infrastructure dependencies, such as data persistence
    in a database. \\
    
    Depending on the Task, the Interactor implementation orchestrates the mapping from
    RequestModels to Entities, or from Entities to ResponseModels, utilizing the IMapper
    interface. \\
    
    Interactor implementations have an internal scope and cannot be implemented outside
    the Application Layer. \\
    \hline
    \end{tabular}
\caption{Interactor Implementation Requirements}
\label{table_requirements_interactor}
\end{table}

\begin{table}[H]
    \begin{tabular}{@{\makebox[2em][c]{\rownumber\space}}  p{0.87\linewidth}}
        \multicolumn{1}{@{\makebox[2em][c]{Nr.}}  p{0.87\linewidth}}{IMapper Requirements}\\ 
    \hline
    The IMapper interface establishes the contract for its derived Mapper implementations.
    \\

    The IMapper interface has a public scope within the system. \\
    \hline
    \end{tabular}
\caption{IMapper Requirements}
\label{table_requirements_imapper}
\end{table}

\begin{table}[H]
    \begin{tabular}{@{\makebox[2em][c]{\rownumber\space}}  p{0.87\linewidth}}
        \multicolumn{1}{@{\makebox[2em][c]{Nr.}}  p{0.87\linewidth}}{RequestModelMapper Requirement}\\ 
    \hline
    The RequestModelMapper is derived from the IMapper interface and follows the specified
    implementation. \\

    The RequestModelMapper is responsible for mapping the values of the necessary data
    attributes from the RequestModel to an Entity. \\
    
    The RequestModelMapper has an internal scope and cannot be implemented outside
    the Application Layer. \\
    \hline
    \end{tabular}
\caption{RequestModelMapper Requirements}
\label{table_requirements_requestmodelmapper}
\end{table}

\begin{table}[H]
    \begin{tabular}{@{\makebox[2em][c]{\rownumber\space}}  p{0.87\linewidth}}
        \multicolumn{1}{@{\makebox[2em][c]{Nr.}}  p{0.87\linewidth}}{ResponseModelMapper Requirement}\\ 
    \hline
    The RequestModelMapper is derived from the IMapper interface and follows the specified
    implementation. \\

    The RequestModelMapper is responsible for mapping the values of the necessary data
    attributes from the RequestModel to an Entity. \\
    
    The RequestModelMapper has an internal scope and cannot be implemented outside the
    Application Layer. \\
       \hline
    \end{tabular}
\caption{ResponseModelMapper Requirements}
\label{table_requirements_responsemodelmapper}
\end{table}

\begin{table}[H]
    \begin{tabular}{@{\makebox[2em][c]{\rownumber\space}}  p{0.87\linewidth}}
        \multicolumn{1}{@{\makebox[2em][c]{Nr.}}  p{0.87\linewidth}}{IPresenter Requirements}\\ 
    \hline
    The IPresenter interface establishes the contract for its derived Presenter
    implementations, typically implemented as part of the Presentation Layer. \\

    The IPresenter interface has a public scope within the system. \\
    \hline
    \end{tabular}
\caption{IPresenter Requirements}
\label{table_requirements_ipresenter}
\end{table}

\begin{table}[H]
    \begin{tabular}{@{\makebox[2em][c]{\rownumber\space}}  p{0.87\linewidth}}
        \multicolumn{1}{@{\makebox[2em][c]{Nr.}}  p{0.87\linewidth}}{I\textit{[\gls{verb}]}Gateway Requirements}\\ 
    \hline
    The \textit{[\gls{verb}]Gateway} interface establishes the contract for its derived Gateway
    implementations, which are typically implemented in the Infrastructure Layer. \\

    The \textit{[\gls{verb}]}Gateway interface has a public scope within the system. \\
    
    Each task is represented in the naming convention of the interface. As an example, the
    basic \gls{crud} actions result in a total of five IGateway interfaces: ICreateGateway,
    IGetGateway, IGetByIdGateway, IUpdateGateway, and IDeleteGateway. \\
    \hline
    \end{tabular}
\caption{\textit{[\gls{verb}]}Gateway Requirements}
\label{table_requirements_IGateway}
\end{table}

\begin{table}[H]
    \begin{tabular}{@{\makebox[2em][c]{\rownumber\space}}  p{0.87\linewidth}}
        \multicolumn{1}{@{\makebox[2em][c]{Nr.}}  p{0.87\linewidth}}{ResponseModel Requirement}\\ 
    \hline
    The ResponseModel consists primarily of data attributes representing the fields of the
    corresponding Entity. Additionally, the ResponseModel may contain data specific to the
    output of the Interactor. \\

    The ResponseModel does not depend on external objects within the architecture. \\
    \hline
    \end{tabular}
\caption{ResponseModel Requirements}
\label{table_requirements_responsemodel}
\end{table}

\begin{table}[H]
    \begin{tabular}{@{\makebox[2em][c]{\rownumber\space}}  p{0.87\linewidth}}
        \multicolumn{1}{@{\makebox[2em][c]{Nr.}}  p{0.87\linewidth}}{RequestModel Requirement}\\ 
    \hline
    The RequestModel consists primarily of data attributes representing the fields of the
    corresponding Entity. Additionally, the RequestModel may contain data specific to the
    input of the Interactor. \\

    The RequestModel does not depend on external objects within the architecture. \\
       
       \hline
    \end{tabular}
\caption{RequestModel Requirements}
\label{table_requirements_requestmodel}
\end{table}

\subsubsection*{Domain Layer}

\begin{table}[H]
    \begin{tabular}{@{\makebox[2em][c]{\rownumber\space}}  p{0.87\linewidth}}
        \multicolumn{1}{@{\makebox[2em][c]{Nr.}}  p{0.87\linewidth}}{Data Entity Requirement}\\ 
    \hline
    The Data Entity consists solely of attributes representing the corresponding data
    fields. \\

    The Data Entity does not rely on external objects within the architecture. \\
    
    The Application Layer is the only layer that utilizes the Data Entity. \\
    \hline
    \end{tabular}
\caption{Data Entity Requirements}
\label{table_requirements_data_entity}
\end{table}

\subsubsection*{The Infrastructure Layer}

\begin{table}[H]
    \begin{tabular}{@{\makebox[2em][c]{\rownumber\space}}  p{0.87\linewidth}}
        \multicolumn{1}{@{\makebox[2em][c]{Nr.}}
        p{0.87\linewidth}}{\textit{[\gls{verb}]}Gateway Implementation Requirement}\\ 
    \hline
    The [\textit{\gls{verb}}]Gateway Implementation derives from the I[\textit{\gls{verb}}]Gateway interface
    and adheres to the specified implementation. \\

    The [\textit{\gls{verb}}]Gateway Implementation is responsible for the interaction
    associated with the specific task, utilizing the infrastructure technology of the
    specific layer (e.g., a SQL database or a filesystem).
    
    The [\textit{\gls{verb}}]Gateway Implementation has an internal scope and cannot be
    instantiated outside of the Layer. \\
    \hline
    \end{tabular}
\caption{\textit{[\gls{verb}]}Gateway Implementation Requirements}
\label{table_requirements_gatewayimplementation}
\end{table}

\subsubsection*{Design Principles compliancy}
Each architectural pattern adheres to at least one of the SOLID principles to ensure that
none of the implementations violate these principles.





\subsection*{Design Principles compliancy}
Each architectural pattern follows at least one of the SOLID principles, ensuring that
none of the implementations violate these principles. 


    \textcolor{red}{TODO: requirement mbt code expansion toevoegen (pag 403)}
\chapter{Artifact Design} \label{chap_designing_artifacts}

This chapter will discuss specific design decisions made to meet the required
functionality while adhering to the requirements outlined in chapter
\ref{chap_requirements}. Two different artifacts are used to support this study. Figure
\ref{fig_overview_design} is a schematical overview of both these artifacts.

\section{The Research Artifacts}
The first artifact consists of two main components: the Clean Architecture Expander and
the Expander framework. The name of the Expander Framework, Pantha Rhei, was inspired by
the Greek philosopher \emph{Heraclitus}, who famously stated that \enquote{life is flux.}
The name reflects the artifact's perceived ability to cope with constant change in a
stable and evolvable manner. Users can interact with the Expander Framework using the
\gls{cli} command \enquote*{flux} in combination with several parameters. Appendix
\ref{appendix_run_flux}, yprovides a comprehensive guide on using this command, including
all available options and parameters.

As illustrated in Figure \ref{fig_overview_design}, the main task of the first artifact or
\enquote*{expand} the second artifact. By entering the correct command, the Expander
Framework loads the model being instantiated during the expansion process. Then, the
required expanders are prepared based on information available through the model. In the
case of this study, the Clean Architecture Expander. The Clean Architecture Expander
consists of a set of tasks and templates. When the Expander Framework executes the Clean
Architecture Expander, the model is instantiated into the generated artifact with the aid
of the templates.

The model is an instance of the meta-model. Consequently, the model can represent any
application as long as the meta-model is respected. In the case of this study, the model
represents the entities, attributes, relationships, and other characteristics of the
meta-model.

As a result, the second artifact (artifact II) allows a user to modify or maintain the
model used by the Expander Framework by exposing a Restful interface. This method
approaches the meta-circularity process, where an expansion process is used to update the
meta-model. Although not fully compliant with the theory of \gls{ns}, the Expander
Framework consists of the required tasks to update its own meta-model. This is illustrated
in Figure \ref{fig_overview_design} by the \enquote*{updates} arrow.

\begin{figure}[H]
    \centering
    \includegraphics[width=0.8\textwidth]{figures/artifactOverview.pdf}
    \caption[Schematic overview of the Artifacts]{Schematic overview of the Artifacts}
    \label{fig_overview_design}
  \end{figure}

\section{The Artifact name and use} \label{sec_artifact_name}

The name of the Expander Framework, Pantha Rhei, was inspired by the Greek philosopher
\emph{Heraclitus}, who famously stated that \enquote{life is flux}. The name reflects the
artifact's ability to cope with constant change in a stable and evolvable manner. The name
is also reflected in the use of the \enquote{flux} command in the \gls{cli}, which allows
users to interact with the application.

To install Pantha Rhei, interested readers can follow the instructions provided in the
appendix \fullref{appendix_installation_instructions}.





\chapter{The Entity Relationship Diagram of the Meta Mode} \label{appendix_metamodel_description}  

\section{The App entity}

The App entity represents an application and is regarded as the highest entry point for the
Generator artifact. The App Entity and subsequent entities contain all the information
needed to generate the Generated Artifact described in section \ref{sec_generated_artifact}. 

\begin{table}[H]
    \small
    \begin{tabular}{ p{0.21\linewidth} p{0.21\linewidth} p{0.49\linewidth} }
        \hline
        \textbf{Name} & \textbf{DataType} & \textbf{Description} \\
        \hline
        Id & Guid & Unique identifier of the application \\
        Name & string & Name of the application \\
        FullName & string & Full name of the application \\
        Expanders & List of Expanders & The Expanders that will be used during the
        generation process. \\
        Entities & List of Entities & The Entities that are applicable for the Generated artifact. \\
        ConnectionStrings & List of ConnectionStrings & The ConnectionString to the
        database that is used by the Generator Artifact. \\
        \hline
    \end{tabular}
    \caption{The fields of the App entity}
    \label{table:app_entity}
\end{table}

\section{The Component entity}

The Component entity represents a software component that can be part of an application.
Based on this entity the Generator Artifact can make design time on where to place certain
elements  

\begin{table}[H]
\small
\begin{tabular}{ p{0.20\linewidth} p{0.20\linewidth} p{0.50\linewidth} }
\hline
\textbf{Name} & \textbf{DataType} & \textbf{Description} \\
\hline
Id & Guid & Unique identifier of the component \\
Name & string & Name of the component \\
Description & string & Description of the component \\
Packages & List of Package & The Packages that should be applied to the component. \\
Expander & Expander & Navigation property to the Expander entity. \\
\hline
\end{tabular}
\caption{The fields of the Component entity}
\label{table:component_entity}
\end{table}

\section{The ConnectionString entity}

The ConnectionString entity represents a ConnectionString used by an application to
connect to a database or other external system.

\begin{table}[H]
\small
\begin{tabular}{ p{0.20\linewidth} p{0.20\linewidth} p{0.50\linewidth} }
\hline
\textbf{Name} & \textbf{DataType} & \textbf{Description} \\
\hline
Id & Guid & Unique identifier of the ConnectionString \\
Name & string & Name of the ConnectionString \\
Definition & string & Definition of the ConnectionString \\
App & App & Navigation property to the App entity \\
\hline
\end{tabular}
\caption{The fields of the ConnectionString entity}
\label{table:connectionstring_entity}
\end{table}

\section{The Entity entity}

The Entity entity represents an entity in the application's data model. 

\begin{table}[H]
\small
\begin{tabular}{ p{0.20\linewidth} p{0.23\linewidth} p{0.47\linewidth} }
\hline
\textbf{Name} & \textbf{DataType} & \textbf{Description} \\
\hline
Id & Guid & Unique identifier of the entity \\
Name & string & Name of the entity \\
Callsite & string & The source code location where the entity is defined. In the case of a C\#
artifact, this is to determine the name of the namespace.\\
Type & string & Type of the entity \\
Modifier & string & Modifier of the entity (e.g. public, private) \\
Behavior & string & The behavior of the entity (e.g. abstract, virtual) \\
App & App & Navigation property to the App entity. \\
Fields & List of Fields & The Fields property represents a collection of the fields that
make up the entity. \\
ReferencedIn & List of Fields & Represents a navigation property to a Field that uses the
current entity as a return type. \\
Relations & List of Relationships & List of relationships involving this entity \\
IsForeignEntityOf & List of Relationships & List of relationships where this entity is the foreign entity \\
\hline
\end{tabular}
\caption{The fields of the Entity entity}
\label{table:entity_entity}
\end{table}

\section{The Expander entity}

The Expander entity represents an expander, which is responsible for generating code for
an application. The Generator Artifact attempts to execute all expanders that are related
to the selected App.

\begin{table}[H]
\small
\begin{tabular}{ p{0.20\linewidth} p{0.22\linewidth} p{0.48\linewidth} }
\hline
\textbf{Name} & \textbf{DataType} & \textbf{Description} \\
\hline
Id & Guid & Unique identifier of the expander \\
Name & string & Name of the expander \\
TemplateFolder & string & relative path to the templates that are used by the expander. \\
Order & int & The order in which the expander is executed \\
Apps & List of Apps & List of applications associated with the expander. \\
Components & List of Components & List of components associated with the expander \\
\hline
\end{tabular}
\caption{The fields of the Expander entity}
\label{table:expander_entity}
\end{table}

\section{The Field entity}

The Field entity represents a field or property of an entity in an application's data
model. Each field has a unique ID, name, and other properties such as its return type,
modifiers, and behavior. It can be associated with an entity and can have relationships
with other entities. The IsKey and IsIndex properties indicate whether the field is part
of the primary key or an index of the entity, respectively.

\begin{table}[H]
\small
\begin{tabular}{ p{0.23\linewidth} p{0.23\linewidth} p{0.45\linewidth} }
\hline
\textbf{Name} & \textbf{DataType} & \textbf{Description} \\
\hline
Id & Guid & Unique identifier of the field \\
Name & string & Name of the field \\
ReturnType & string & Return type of the field \\
IsCollection & bool & Whether the field is a collection or not \\
Modifier & string & Modifier of the field (e.g. public, private) \\
GetModifier & string & Modifier of the get accessor for the field \\
SetModifier & string & Modifier of the set accessor for the field \\
Behavior & string & The behavior of the field (e.g. abstract, virtual) \\
Order & int & The order of the field within its entity \\
Size & int? & The size of the field \\
Required & bool & Whether the field is required or not \\
Reference & Entity & The entity that this field refers to\\
Entity & Entity & A navigation property to the parent entity \\
IsKey & bool & Indicates whether the field is part of the primary key \\
IsIndex & bool & Indicates whether the field is part of an index \\
RelationshipKeys & List of Relationships & A List of entities that are defined as relations. \\
IsForeignEntityKeyOf & List of Relationships & List of relationships to the field that is the foreign key \\
\hline
\end{tabular}
\caption{The fields of the Field entity}
\label{table:field_entity}
\end{table}

\section{The Package entity}

The Package entity represents a software package that can be used by a component. This
could either be a Nuget package in the case of .NET projects, or for example npm packages
for web projects.

\begin{table}[H]
\small
\begin{tabular}{ p{0.20\linewidth} p{0.20\linewidth} p{0.50\linewidth} }
\hline
\textbf{Name} & \textbf{DataType} & \textbf{Description} \\
\hline
Id & Guid & Unique identifier of the package \\
Name & string & Name of the package \\
Version & string & Version of the package used \\
Component & Component & Component associated with the package \\
\hline
\end{tabular}
\caption{The fields of the Package entity}
\label{table:package_entity}
\end{table}

\section{The Relationship entity}

The Relationship entity represents a relationship between two entities in the App's data
model. The Relationship entity has proper cardinality support. Relationships are
bidirectional and can be navigated from either entity.

\begin{table}[H]
\small
\begin{tabular}{ p{0.24\linewidth} p{0.12\linewidth} p{0.55\linewidth} }
\hline
\textbf{Name} & \textbf{DataType} & \textbf{Description} \\
\hline
Id & Guid & Unique identifier of the relationship \\
Key & Field & The key field of the relationship \\
Entity & Entity & Navigation property to the parent Entity \\
Cardinality & string & The cardinality of the relationship \\
WithForeignEntityKey & Field & The foreign key field of the relationship, pointing to a
Field entity. \\
WithForeignEntity & Entity & The entity associated with the foreign key field \\
WithCardinality & string & The cardinality of the relationship with the foreign entity \\
Required & bool & indicates whether the relationship is required or not \\
\hline
\end{tabular}
\caption{The fields of the Relationship entity}
\label{table:relationship_entity}
\end{table}
\section{Plugin Architecture} \label{subsec_plugin_architecture}

The Expander Framework Artifact is responsible for loading and bootstrapping Expanders and
initiating the generation process. Expanders are dynamically loaded at runtime through a
dotnet capability called assembly binding, allowing the architecture illustrated in the
following image \parencite{koks_expanderpluginloaderinteractor_2023}.

\begin{figure}[H]
  \centering
  \includegraphics[width=0.5\textwidth]{figures/plugin_architecture.pdf}
  \caption[Plugin Archticture]{Expanders are considered plugins}
  \label{fi:plugin_architecture}
\end{figure}

This plugin design adheres to several principles of \gls{solid}. The \gls{srp} principle
is implemented by ensuring that an expander generates one and only one construct. The
\gls{ocp} principle is applied by allowing the creation of new expanders in addition to
the already existing ones. The \gls{lsp} principle is respected by enabling the addition
or replacement of expanders without modifying the internal workings of the Expander
Framework.

More details can be found in the Appendix \fullref{list_expanderpluginloaderinteractor}
\section{Expanders}

The Exander Framework allows for the miscellaneous execution of expanders of any type. The
Expander Framework is independent of any of the details of Expanders, fully adhering to
the principle of \gls{dip}. Conversely, an Expander is required to implement several
interfaces to ensure execution and dependency management are available during runtime. The
Expander Framework also consists of a set of default tasks, such as the execution of the
expansion tasks known as ExpanderHandlerInteractors
\citecode{koks_iexpanderhandlerinteractor_2023}, logging, bootstrapping dependencies, and
tasks to execute harvestings and injections. Except for the use of the
IExpanderInteractor, non of which are required.

Figure \ref{fig_expander_design} illustrates the dependencies between the domain layer of
the Expander Framework. The Clean Architecture Expander is considered an application layer
containing specific tasks bounded to a particular application or process. In this case,
the Expansion process.

\begin{figure}[H]
    \centering
    \includegraphics[width=1\textwidth]{figures/expander.pdf}
    \caption[The design of an Expander]{The design of an Expander}
    \label{fig_expander_design}
  \end{figure}
\section{The IExecutionInteractor command} \label{subsec_IExecutionInteractorObject}

An exciting implementation that facilitates a high degree of cohesion while maintaining
low coupling is the utilization of the \code{koks_iexecutioninteractor_2023} interface
\parencite{koks_iexecutioninteractor_2023}. This interface allows for the execution of
various derived types responsible for specific tasks, such as executing Handlers,
Harvesters, and Rejuvenators \parencites{koks_expandentitieshandlerinteractor_2023,
koks_regionharvesterinteractor_2023, koks_regionrejuvenatorinteractor_2023}. The
implementation promotes decoupling by adhering to both \gls{ocp} and \gls{lsp}.

\begin{figure}[H]
    \centering
    \includegraphics[width=1\textwidth]{figures/command_pattern.pdf}
    \caption[Low coupling with \code{koks_iexecutioninteractor_2023}]{Low coupling with \code{koks_iexecutioninteractor_2023}}
    \label{fig_iexecutioninteractor}
  \end{figure}


Figure \ref{fig_iexecutioninteractor} illustrates that the required interfaces are placed
in the Domain Layer of the Expander Framework. In contrast, the concrete classes also can
be implemented as part of the internal scope of the Clean Architecture Expander
\parencite{koks_migrationharvesterinteractor_2023}. Code listing
\fullref{list_expandentitieshandlerinteractor} illustrates an implementation example of
this interface. Finally, the code listing \fullref{list_CodeGeneratorInteractor}
illustrates the aggregation of the execution, which allows for a graceful cohesion of the
execution Tasks \parencite{koks_codegeneratorinteractor_2023}.

\section{Dependency management} Dependency management is an extremely valuable aspect of
achieving stability and evolvability. Dependency management can be achieved by using
Dependency Injection. This research acknowledges the statement of
\textcite[215]{mannaert_normalized_2016} that Dependency Injection does not solve coupling
between classes. Working on the Artifact has shown that combinatorial effects can occur
when not careful. Nevertheless, Dependency Injection is a widely used pattern in building
the Artifact. In order to achieve stability and evolvability, the Dependency Injection
pattern \underline{must} be combined with various other principles of both \gls{ca} and
\gls{ns}. 

The goal is to centralize the management of dependencies and remove unwanted manual object
instantiations in the code. Al this while respecting the \gls{dip} principle so that each
Component Layer is responsible for managing its dependencies. The Artifact achieves this
by using extension methods as illustrated in Code Listing
\ref{list_DependencyInjectionExtension}
\parencite{koks_dependencyinjectionextension_2023}. Additionally, and quite significantly,
implementations primarily rely on abstractions or contracts (interfaces) instead of the
details of concrete implementations. 

Traditionally, Dependency Injection injects instantiations through constructor parameters
or class properties. Although there are benefits in this approach, doing so will
eventually lead to combinatorial effects, breaking the stability of a Software Artifact.
In order to solve this problem, the Artifact used the Service Locator pattern, a central
registry responsible for resolving dependencies \parencite{wikipedia_service_2023}. Many
frameworks are available from \gls{nuget}, but the Artifact uses the Service Registry that
is part of the .NET framework. This service registry is considered a cross-cutting
concern. The dependency on this technology is reduced by applying the principles of the
\gls{lsp} and \gls{isp}. The Artifact creates and uses separate interfaces to register
\parencite{koks_idependencymanagerinteractor_2023} and resolve
\parencite{koks_idependencyfactoryinteractor_2023} dependencies. As illustrated in Code
Listing \ref{list_DependencyManager}, the framework technology dependency is abstracted
behind implementing those interfaces \parencite{koks_dependencymanagerinteractor_2023}. 

Practically every class gets the \citecode{koks_idependencyfactoryinteractor_2023}
injected, on which further resolving is responsible for that class's inner workings. Code
Listing \ref{list_Injecting} illustrates how this is done in the
\citecode{koks_abstractexpander_2023} class. Finally, all the dependencies are
bootstrapped on application bootup, depicted in Code Listing \ref{list_dip}. 

The approach described here has many advantages in managing the stability and evolvability
of the Software Artifact. However, as for most things, there are also some drawbacks. For
example, a good amount of experience is required for developers to understand code that
incorporates abstractions, contracts, and Dependency Injection. Another drawback is that
dependency errors are detected in runtime rather than compile time. The benefits of the
Artifacts, however, outweigh the drawbacks.
\chapter{Evaluation results} \label{chap_evaluation}

In this chapter, we will compare the two approaches to architectural design: \gls{ca} and
\gls{ns}. We will examine how these approaches converge and affect software design on the
artifacts. First, in Section \ref{sec_converging_principles}, we will compare the
principles of \gls{ca} with the principles of \gls{ns}. Then, we will compare the design
elements of the two approaches in section \ref{sec_converging_elements}. Both sections
will showcase real-world examples from the Artifacts to illustrate theoretical
manifestations.

\section{Converging elements} \label{sec_converging_elements}

In this section, we will apply a systematic cross-referencing approach to assess the level
of convergence between each of the elements of both \gls{ca} and \gls{ns}. Along with a
brief explanation, the level of convergence is denoted as followed:

\begin{table}[H]
    \begin{tabular}{ l l p{0.57\linewidth}} Strong convergence & \conv & Both
        elements have a high level of similarity or are closely related in terms of their
        purpose, structure, or functionality.\\
        Some convergence & \partconv &  Both elements have some similarities or
        share certain aspects in their purpose, structure, or functionality, but they are
        not identical or directly interchangeable.\\
        No convergence & \noconv &  The elements are not related or have no
        significant similarities in terms of their purpose, structure, or functionality.\\
    \end{tabular}
\end{table}
\section{Converging elements} \label{sec_converging_elements}

In this section, we will apply a systematic cross-referencing approach to assess the level
of convergence between each of the elements of both \gls{ca} and \gls{ns}. Along with a
brief explanation, the level of convergence is denoted as followed:

\begin{table}[H]
    \begin{tabular}{ l l p{0.57\linewidth}} Strong convergence & \conv & Both
        elements have a high level of similarity or are closely related in terms of their
        purpose, structure, or functionality.\\
        Some convergence & \partconv &  Both elements have some similarities or
        share certain aspects in their purpose, structure, or functionality, but they are
        not identical or directly interchangeable.\\
        No convergence & \noconv &  The elements are not related or have no
        significant similarities in terms of their purpose, structure, or functionality.\\
    \end{tabular}
\end{table}
\section*{Evaluating the findings}

\begin{table}[!ht]
    \centering
    \begin{tabular}{lcccc}
    \toprule
     & SoC & DVT & AVT & SoS \\
    \midrule
    SRP & \fullAlignment & \partialAlignment & \partialAlignment & \noAlignment \\
    OCP & \fullAlignment & \noAlignment & \fullAlignment & \noAlignment \\
    LSP & \fullAlignment & \noAlignment & \partialAlignment & \noAlignment \\
    ISP & \fullAlignment & \partialAlignment & \partialAlignment & \noAlignment \\
    DIP & \fullAlignment & \partialAlignment & \partialAlignment & \noAlignment \\
    \bottomrule
    \end{tabular}
    \caption{Convergence between the SOLID and \gls{ns} principles}
    \label{tab_convergence_principles_summarized}
    \end{table}

    \begin{table}[!ht]
        \centering
        \begin{tabular}{lccccc}
        \toprule
         & Data & Task & Flow & Connector & Trigger \\
         \midrule
        Entity & \fullAlignment & \noAlignment & \noAlignment & \noAlignment & \noAlignment \\
        Interactor & \noAlignment & \fullAlignment & \partialAlignment & \noAlignment & \noAlignment \\
        RequestModel & \partialAlignment & \noAlignment & \noAlignment & \noAlignment & \noAlignment \\ 
        ResponseModel & \partialAlignment & \noAlignment & \noAlignment & \noAlignment & \noAlignment \\
        ViewModel & \partialAlignment & \noAlignment & \noAlignment & \noAlignment & \noAlignment \\
        Controller & \noAlignment & \noAlignment & \noAlignment & \partialAlignment & \partialAlignment \\
        Gateway & \noAlignment & \noAlignment & \noAlignment & \fullAlignment & \noAlignment \\
        Presenter & \noAlignment & \noAlignment & \noAlignment & \noAlignment & \noAlignment \\
        Boundary & \noAlignment & \noAlignment & \partialAlignment & \fullAlignment & \noAlignment \\ \bottomrule
        
        \end{tabular}
        \caption{Convergence between the SOLID and \gls{ns} elements}
        \end{table}




\chapter{Discussion} \label{chap_discussion}
\chapter{Conclusions and discussion} \label{chap_conclusions}

This thesis culminates a multifaceted exploration into the convergence of \gls{ca} with
\gls{ns}. We have drawn upon the author's firsthand experience in designing Software
Architectures used rigorous theoretical research and created a practical and working
Software Artifact. The primary objective was to investigate the convergence between \gls{ca}
and \gls{ns}, by analyzing their principles and design elements through theory and
practice. This Chapter will summarize the findings into a research conclusion.

\section{Conclusion}

A noteworthy distinction between \gls{ns} and \gls{ca} lies in their foundational roots.
\gls{ns} is a product of computer science research built upon formal theories and
principles derived from rigorous scientific investigation. Although, throughout this
thesis, \gls{ns} is referred to as a development approach, it is a part of Computer
Science.

Stability and evolvability are concepts not directly referenced in the literature of
\gls{ca}, but very much converges with the goal of \textcite[31]{mannaert_normalized_2016}.
Clearly depicted in Table \ref{tab_convergence_principles_summarized} The attentive reader
surely observes the shared emphasis on modularity and the separation of concerns, as all
SOLID principles have a stong convergence with \gls{soc}. Both approaches attempt to
achieve low coupling and high cohesion. In addition, \gls{ca} add the dimensions of
dependency management as usefull measure to improve maintainability and manage modularity.

The Transparency of Data versions appears to be underrepresented in the SOLID principles
of \gls{ca}. \gls{dvt} is primarily supported by the \gls{srp} of \gls{ca}, as evidenced
by the presence of ViewModels, RequestModels, ResponseModels, and Entities in the
Artifact. It is worth noting that these are an integral part of the design elements of
\gls{ca}. While \gls{ca} does address \gls{dvt} through the \gls{srp}, a more
comprehensive representation and integration of \gls{dvt} as a principle within the
principles of \gls{ca} will improve the convergence of \gls{ca} with \gls{ns}, potentially
improving the stability and evolvability of Software Systems based on \gls{ca}.

As indicated in Table \ref{tab_convergence_elements_summarized}, seems to lack a strong
foundation for receiving external triggers in it's desing phylosophy. This is partially
represented by the Controller element, however this element tends to be use for web
enabled environments like websites and Restful APIs. This may result in a less
comprehensive approach to receiving external triggers across various technologies or
systems.

A critical difference between \gls{ca} and \gls{ns} lies in their approach to handling
state. \gls{ca} does not explicitly address state management in its principles or design
elements. At the same time, \gls{ns} provides the principle of Separation of State,
ensuring that state changes within a Software System are stable and evolvable. This
principle can be crucial in developing scalable and high-performance systems, as it
isolates state changes from the rest of the system, reducing the impact of state-related
dependencies and side effects. 

The findings can only leads to the conclusion that the convergence between \gls{ca} and
\gls{ns} is incomplete because \gls{ca} needs specific state management principles. As a
result, \gls{ca} cannot fully ensure stable and evolvable software artifacts as defined by
\gls{ns}.



\section{Discussion}

In this research, the convergence between \gls{ca} and \gls{ns} has been thoroughly
investigated. While it has been demonstrated that the convergence between these two
approaches is incomplete, combining both methodologies is highly beneficial for both
\gls{ns} and \gls{ca} for various reasons. The primary advantage of this convergence
lies in the complementary nature of \gls{ca} with \gls{ns}, where each approach provides
strengths that can be leveraged to address a strong architectural design. 

Clean Architecture offers a well-defined, practical, and modular structure for software
development. Its principles, such as SOLID, guide developers in creating maintainable,
testable, and scalable systems. This architectural design approach is highly suitable for
various applications and can be easily integrated with the theoretical foundations
provided by \gls{ns}. 

Conversely, the NS approach offers a more comprehensive theoretical understanding of
achieving stable and evolvable systems. Furthermore, the popularity and widespread
adoption of Clean Architecture in the software development community can benefit
Normalized Systems. As more developers already adopting \acrlong{ca} become more familiar
with \acrlong{ns} and recognize their value to software design. Combining both approaches
will likely lead to increased adoption of \acrlong{ns}.
\section{Limitations}

In this research, the artifacts created to demonstrate the alignment between \gls{ca} and
\gls{ns} have shown promising results. However, it is essential to recognize these
artifacts' limitations, particularly in implementing \gls{ns} principles such as the
Separation of State Principle and the Trigger Element. These limitations must be
acknowledged to guide future work and refinement of the combined architectural design
approaches.

One of the primary limitations of the artifacts lies in their incomplete representation of
the Separation of State principle. This principle is crucial in \gls{ns} to ensure proper
handling of state changes while achieving stability and evolvability. While the artifacts
incorporate some aspects of state management, they fall short of fully implementing the
Separation of State principle as \gls{ns} prescribes. 

The other limitation of the artifacts is their lack of a comprehensive Trigger Element, an
essential element of \gls{ns}. The Trigger Element manages external triggers while
ensuring that software remains stable and evolvable. In the artifacts, incorporating the
Trigger Element is limited, primarily relying on the Controller element from \gls{ca}.
While this approach may be sufficient for web-enabled environments such as websites and
RESTful APIs, it may not be adequate for a broader range of requirements.
\section{Reflections} \label{chap_reflection}

This section will dive into my enriching experiences and invaluable learnings acquired
while working on this research and thesis and is inspired by one of the master classes
about the \gls{ee} discipline. \gls{ee} encourages using grounded methodologies and
theories, like Five Way Framework, to comprehend the inner workings of an enterprise
\parencite[262]{dietz_enterprise_2020}. I will apply the so-called Five Way Framework to
reflect on this research. By incorporating the Five Way Framework into the section, I
aspire to coherently showcase my learnings and reflections, shedding light on my thought
processes, strategies, modeling techniques, working methodologies, and support mechanisms.

\begin{figure}[H]
    \centering
    \includegraphics[width=0.5\textwidth]{figures/5ways.pdf}
    \caption[The Five Way Framework]{The Five Way Framework, inspired by \textcite{dietz_enterprise_2020}}
    \label{fig_5ways}
\end{figure}

\subsection{Way of Thinking}

Early in my career, I became obsessed with software Quality and Maintainability. The
\gls{ns} theory shifted the obsession toward Stability and evolvability. Gaining a
thorough understanding of the essential principles of this theory has boosted my
confidence in making informed decisions regarding all aspects of architecture, not just
limited to software.

As a Domain Architect, my job involved creating software products using the \gls{mdd}
paradigm. Initially, I was skeptical about this approach based on my early experiences.
The theory of \gls{ns} taught me to understand better the reasoning and characteristics of
code generation, on which I then realized that my skepticism was more about the process
caused as an effect on the implementation of the \gls{mdd}. The knowledge of \gls{ns}
helped me gain a clearer vision and helped me push the roadmap on the \gls{mdd} framework
in the right direction.

\subsection{Way of Modeling}

To explain the implementation concepts of the artifact, I looked into various modeling
languages. Archimate was the first option I considered. However, during one of the \gls{ee}
masterclasses, I learned the hard way that Archimate is not always the best choice for
communicating your models to a broad audience. I thought about using basic "boxes and
arrows" but decided to use the UML2 standard because it is a formal modeling language.

\subsection{Way of Working}

I very much enjoyed designing and creating the C\# artifacts. In hindsight, I enjoyed it
so much that I put in way too much effort than was needed. I was very curious about the
aspects of code generation, the effect of code generation on stable and evolvable
artifacts, and meta-circularity characteristics. I am confident I could have arrived at
the same conclusions presented in this thesis using a manually built Restful C\# artifact.
However, the insights I gained on the subjects of code expansion are of invaluable value
to me. Therefore, I am very pleased and satisfied that I took the effort to build the Code
Expander as a primary artifact.

The use of a testing framework like \citetitle{github_archunitnet_2023}
\parencite*{github_archunitnet_2023} could have benifitted the developement process in
testing the adherence to the artifact requirements described in chapter
\ref{sec_artifact_requirements}

The \gls{ns} theorems are formulated very clearly and abstractly, making them also
applicable outside the software engineering field. During the masterclasses, we learned
about the application of \gls{ns} in the domains of Firewalls, Document management
systems, and Evolvable Business Processes. I also experienced benefits in structuring and
maintaining my thesis document using \gls{vscode} and Latex by applying the principle of
\gls{soc} in managing the various chapters and sections. 

\subsection{Way of Organizing}

I should have been able to finish sooner. I was one of the first with a research topic and
started working on my artifact in the first month when starting this journey. The artifact
was as good as ready before the summer holidays of the first year. Unfortunately, I
postponed writing the thesis until a later moment. I want to think that next time I will
start sooner, but knowing myself, I need some pressure to perform the less fun tasks, like
writing this thesis.

The review process seemed difficult and sometimes even problematic for a couple of
reasons. Next time I will ensure having the proper tools and agree on procedures to
improve reviews from multiple proofreaders. Secondly, I noticed that having multiple
proofreaders sometimes steers in opposite directions. This sometimes affected my ability
to make decisions and negatively affected my confidence. Having a joined review document,
where all proofreaders can leave comments, will significantly improve this experience for
me and my proofreaders. Then there is the personal aspect of sometimes taking things too
personally, grounded in a lack of self-confidence. However, this experience improved my
self-confidence. 

\subsection{Way of Supporting}

At the beginning of my research, I received a thorough introduction to the \gls{ns}
Theories and the Prime Radiant tooling from an employer at NSX. This introduction was
extremely helpful in gaining a better understanding of the fundamentals of \gls{ns}. It
also inspired me to consider the Code Expansion as a primary artifact. 

For the writing of the thesis, I decided to use Latex. I quickly discovered that Overleaf
was one of the most popular editors. Nevertheless, I continued my search since I rejected
the idea of relying on online tooling for writing my thesis. At some point, I decided to
experiment with my favorite code editor \gls{vscode}, and with the help of a latex package
manager and some \gls{vscode} plugins, I was able to create a fully-fledged Latex Editor
in \gls{vscode}, being able to use all the other benefits that come with \gls{vscode}. In
my next project, I will likely use the \gls{vscode} Latex editor again.



%----------------------------------------------------------------------------------------
%	THESIS CONTENT - APPENDICES
%----------------------------------------------------------------------------------------

\appendix % Cue to tell LaTeX that the following "chapters" are Appendices

% Include the appendices of the thesis as separate files from the Appendices folder
% Uncomment the lines as you write the Appendices

%\include{Appendices/Erd.tex}
\chapter{The Entity Relationship Diagram of the Meta Mode} \label{appendix:erd}  
\begin{figure}[H]
    \centering
    \includegraphics[width=1\textwidth]{Figures/erd.pdf}
    \caption[ERD]{Entity Relationship Diagram of the MetaModel}
    \label{fig:erd}
\end{figure}

\chapter{Installation Instructions} \label{appendix:installation_instructions} 

\section{Prerequisits} \label{appendix:installation_prerequisits} 

\subsection*{Create output folder}
Create an output folder so that Pantha Rhei:
\begin{itemize}
    \item has a location where to find the required expanders.
    \item has a location where the log files are stored.
    \item has a location where the result of the generation processes can be stored.
\end{itemize}

The location of the output folder is irrelevant.

\subsection*{Create the nuget.config file}
Add a nuget.config file to the output folder with the following content:
\lstinputlisting[
    caption={The content of the nuget.config file},
    label={list:nugetconfig}]
    {Snippets/nuget.config}

\subsection*{Install the Expanders}
Click the following link and download the required expanders. The expander contains the
required logic and templates in order for Pantha Rhei to execute the code generation process.

\url{https://www.add/a/correct/url/here}

Create a subfolder \emph{Expanders} in the output folder created in the first step. Extract
each of the expanders in their own folder in the \emph{Expanders} folder.

When following the previous steps you should have the following folder structure
available:

\begin{forest}
    for tree={
      font=\ttfamily,
      grow'=0,
      child anchor=west,
      parent anchor=south,
      anchor=west,
      calign=first,
      inner xsep=7pt,
      edge path={
        \noexpand\path [draw, \forestoption{edge}]
        (!u.south west) +(7.5pt,0) |- (.child anchor) pic {folder} \forestoption{edge label};
      },
      % style for your file node 
      file/.style={edge path={\noexpand\path [draw, \forestoption{edge}]
        (!u.south west) +(7.5pt,0) |- (.child anchor) \forestoption{edge label};},
        inner xsep=2pt,font=\small\ttfamily
                   },
      before typesetting nodes={
        if n=1
          {insert before={[,phantom]}}
          {}
      },
      fit=band,
      before computing xy={l=15pt},
    }  
    [PanthaRhei.Output
    [Expanders
        [.Templates]
    ]
    [nuget.config, file]
  ]
\end{forest}

\section{Installing Pantha Rhei} \label{appendix:installing_pantha_rhei} 
Run the following command to install Pantha Rhei.
\lstinputlisting[
    caption={The install command},
    label={list:install-command}]
    {Snippets/dotnet-install-command.txt}

Pantha Rhei is now ready for use.
\chapter{Designs} \label{appendix_designs} 

\section{Legenda} \label{appendix_legenda} 

In order to visualize the designs of the artifact, a standard UML notation is used. The
designs containing relationships adhere to the following definitions.

\begin{figure}[H]
  \centering
  \includegraphics[width=0.8\textwidth]{figures/class_diagram_legenda.pdf}
  \caption[UML Notation used]{UML notation}
  \label{fi:class_diagram_relationship_notation}
\end{figure}

\section{Generic design} \label{appendix_generic_design} 
\begin{figure}[H]
  \centering
  \includegraphics[width=1\textwidth]{figures/classdiagram_generic_architecture.pdf}
  \caption[Generic architecture]{The Generic architecture of the artifacts}
  \label{fi:generic_design}
\end{figure}
%\include{Appendices/AppendixC}

%----------------------------------------------------------------------------------------
%	BIBLIOGRAPHY
%----------------------------------------------------------------------------------------

\printbibliography[heading=bibintoc, notkeyword={artifact}, title=Bibliography]
\printbibliography[heading=bibintoc,  keyword={artifact}, title=Code Examples]


%----------------------------------------------------------------------------------------

\end{document}  
