\subsection{Normalized Elements} \label{subsec_ns_elements} 

In the context of the \gls{ns} Theory approach, the goal is to design evolvable software,
independent of the underlying technology. Nevertheless, when implementing the software and
its components, a particular technology must be chosen. For Object Oriented Programming
Languages like Java, the following Normalized Elements have been proposed
\parencite[363-398]{mannaert_normalized_2016}.It is essential to recognize that different
programming languages may necessitate alternative constructs
\parencite[364]{mannaert_normalized_2016}.

\begin{table}[H]
    \begin{tabular}{ p{0.15\linewidth} p{0.75\linewidth}}
        \hline
        \textbf{Element} & \textbf{Description} \\ 
        \hline
        Data & This object represents a piece of data in the system. Data elements are
        used to pass information between processing functions and other objects. In
        \gls{ns}, data elements are typically standardized to ensure consistency across
        the system.\\ \midrule

        Task & This object represents a specific task or action in the system. Tasks can
        be composed of one or more processing functions and can be used to represent
        complex operations within the system.\\ \midrule

        Connector & This object is used to connect different parts of the system.
        Connectors can link processing functions, data elements, and other objects to work
        together seamlessly.\\ \midrule

        Flow & This object represents the flow of control through the system.
        It determines the order in which processing functions are executed and can be used
        to handle error conditions or other exceptional cases.\\ \midrule

        Trigger & A trigger represents an object that reacts to specific events or changes in the system
        by executing predefined actions.\\

        \bottomrule
    \end{tabular}
    \caption{The Elements proposed by Normalized Systems Theory}
    \label{ns_element}
\end{table}