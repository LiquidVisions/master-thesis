\subsection{The Flux command}

Pantha Rhei is used by executing the \emph{flux} command with the following parameters

\begin{table}[H]
    \begin{tabular}{ l | p{0.78\linewidth}}
        \toprule
        --root & A mandatory parameter that should contain the full path to the output
        directory \fullref{appendix:installation_instructions}. \\
        --app & A mandatory parameter indicating the unique identifier of the application that should be generated. \\
        --mode & An optional parameter that determines if an handler should be executed.
        \emph{Default} is the default fallback mode (see \ref{tab:generation_modes}). \\
        --clean & Determines if the output folder is cleaned prior executing the expanding
        process.\\
        --reseed & An optional parameter. Disregards the expanding process. The model will
        be completely cleaned, and reseeded based on the entities of the expander
        artifact. This enables to a certain extent the meta-circularity and enables the
        expander artifact to generate itself. \\
        \bottomrule
    \end{tabular}
    \caption{The \emph{flux} commandline parameters}
    \label{tab:commandline_parameters}
\end{table}

\code{GenerationModes} are available in order to isolate the execution of the
\code{ExpanderHandler}. It requires a current implementation shown in listing
\ref{list:runmode_example}. The following runmodes are available.

\begin{table}[H]
    \begin{tabular}{ l | p{0.78\linewidth}}
        \toprule
        Default & The default generation mode. This mode will trigger the required dotnet
        templates to be installed. The Harvest and Rejuvenation handlers will also be executed. \\
        Extend & An extended generation mode, which allows for a partial generation when the
       generated artifact has been generated already. \\
       Deploy & An optional mode that allows for expander handlers to run deployments in
       isolation.  \\
       Migrate & An optional mode that allows for expander handlers to run migrations in
       isolation. \\
        \bottomrule
    \end{tabular}
    \caption{The available \emph{Generation modes}}
    \label{tab:generation_modes}
\end{table}

\lstinputlisting[
    caption={Example on how an expander handler can adhere to the RunMode parameters},
    label={list:runmode_example}]
    {Snippets/RunModeExample.cs}


\lstinputlisting[
    caption={Example command executing Pantha Rhei},
    label={list:flux}]
    {Snippets/flux.txt}