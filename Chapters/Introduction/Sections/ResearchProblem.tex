\section{Research Problem: The plethora of proposed design \\ principles}
\label{sec:research_problem}

Design principles, patterns, and theorems are, on top of all, additional measures to
enhance the modularity, stability, and evolvability of software artifacts. This thesis
focuses on two prevalent approaches to research the hypothesized convergence introduced in
\ref{sec:into_ca} and \ref{sec:inro_ns}. 

A wide variety of proposed design principles are available for the challenges that occur
in modular and evolvable software architecture. Many great experiences documented
throughout professional and personal blog posts on the internet. Unfortunately, many
experiences have mixed outcomes, some are opinionated, and results are sometimes based on
improper interpretations of the proposed solutions.

Deciding on the best fit for one of the solutions is a recurring and often challenging task
for software architects. A popular and widely accepted solution from software engineering
literature is Clean Architecture. There is a broad supporting community, and many corporate
solutions move toward architectures similar to the Clean Architecture
approach. 

An architecture that derives from science and empirical evidence is Normalized Systems
\parencite{mannaert_normalized_2009,mannaert_normalized_2016}. Deciding between the two
approaches can be a challenging task with little documentation and research. Is it
possible that combining the two approaches can be an method that leads to a highly modular
and evolvable software artifact? Let us start with a small introduction to both
approaches.