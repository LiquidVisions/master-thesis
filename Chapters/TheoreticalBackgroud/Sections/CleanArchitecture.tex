\section{The Theoretical background of Clean architecture}\label{sec:ca_theory}

\subsection{SOLID Design principles} \label{subsec:solid}

\begin{itemize}
    \item toelichting op solid
\end{itemize}

\subsubsection{Single Responsibility principle}
\begin{itemize}
    \item toelichting op Single Responsibility
\end{itemize}

\subsubsection{Open-Closed principle}
\begin{itemize}
    \item toelichting op Open-Closed principle
\end{itemize}

\subsubsection{Liskov substitution principle}
\begin{itemize}
    \item toelichting op Liskov substitution
\end{itemize}

\subsubsection{Dependency inversion principle}
\begin{itemize}
    \item toelichting op Dependency inversion principle
\end{itemize}

\subsection{Construction elements}
\begin{itemize}
    \item Gateways, Interactors, Entities, Boundaries etc. 
    \item In het hoofdstuk conclusie wordt
    besproken in hoeverre deze construction elements convergeren met die van NS en een
    analyse in hoeverre deze bijdragen aan evolvability zoals NST
\end{itemize}