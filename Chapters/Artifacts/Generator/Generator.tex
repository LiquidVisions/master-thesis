\section{Designing and Creating the generator artifact} \label{sec_generator_artifact}

In the previous chapters, we discussed the importance of software stability and
evolvability in order to cope with the continuously changing business and technological
requirements. Although considered from a much broader perspective, the Greek philosopher
\emph{Heraclitus} described this state of constant \emph{flux} with his famous statement
\emph{Pantha Rhei}. His statement was the main inspiration for the name of the generator
artifact.

The following sections discuss the design, implementations and functionality of the Pantha
Rhei, the generator artifact. The artifact is created in fulfillment of this research.
Whereas the name of the artifact was inspired by Heraclitus, the functional idea behind
the artifact was inspired by the theory behind \gls{ns} and by the Prime Radiant
application of the company NSX
\footnote{\url{https://normalizedsystems.org/prime-radiant/}}. The concept of code
generation in order to achieve stable and evolvable software is one of the important aspects
of \gls{ns}.

For the interested readers of this thesis, it is rather simple to install the Pantha Rhei
application by following the instruction in the appendix
\fullref{appendix_installation_instructions}.

\subsection{The Flux command}

Pantha Rhei is used by executing the \emph{flux} command with the following parameters

\begin{table}[H]
    \begin{tabular}{ l | p{0.78\linewidth}}
        \toprule
        --root & A mandatory parameter that should contain the full path to the output
        directory \fullref{appendix_installation_instructions}. \\
        --app & A mandatory parameter indicating the unique identifier of the application that should be generated. \\
        --mode & An optional parameter that determines if a handler should be executed.
        \emph{Default} is the default fallback mode (see \ref{tab_generation_modes}). \\
        --clean & Determines if the output folder is cleaned prior to executing the expanding
        process.\\
        --reseed & An optional parameter that bypasses the expanding process. The model will
        be thoroughly cleaned, and reseeded based on the entities of the expander
        artifact. This enables to a certain extent the meta-circularity and enables the
        expander artifact to generate itself. \\
        \bottomrule
    \end{tabular}
    \caption{The \emph{flux} command line parameters}
    \label{tab_commandline_parameters}
\end{table}

RunModes are available to isolate the execution of the ExpanderHandler. It
requires a current implementation shown in listing \ref{list_runmode_example}. The
following RunModes are available.

\begin{table}[H]
    \begin{tabular}{ l | p{0.78\linewidth}}
        \toprule
        Default & The default generation mode. This mode will trigger the required dotnet
        templates to be installed. The Harvest and Rejuvenation handlers will also be executed. \\
        Extend & An extended generation mode, which allows for a partial generation when the
       generated artifact has been generated already. \\
       Deploy & An optional mode that allows for expander handlers to run deployments in
       isolation.  \\
       Migrate & An optional mode that allows for expander handlers to run migrations in
       isolation. \\
        \bottomrule
    \end{tabular}
    \caption{The available \emph{Generation modes}}
    \label{tab_generation_modes}
\end{table}

\lstinputlisting[
    caption={Example on how an expander handler can adhere to the RunMode parameters},
    label={list_runmode_example}]
    {Snippets/RunModeExample.cs}


\lstinputlisting[
    caption={Example command executing Pantha Rhei},
    label={list_flux}]
    {Snippets/flux.txt}
\subsection{Expanders} \label{subsec:expanders}

As described in section \ref{fi:plugin_architecture}, and portraited in figure
\ref{fi:plugin_architecture}, expanders are used as plugins by the Pantha Rhei expander
artifact. There are a couple of prerequisites applicable before the expander can be
dynamically loaded as a plugin at runtime.

\subsubsection*{Prerequisite 1: Project dependency}
The expander should have a dependency on the \gls{dll} of the project
\citecode{koks_pantharheigeneratordomain_2023}.

\subsubsection*{Prerequisite 2: Implements \code{koks_iexpanderinteractor_2023}} In order
to behave like an expander, one should behave like an expander. This is done by
implementing the \citecode{koks_iexpanderinteractor_2023} interface. Although not
required, it is strongly recommended to use the abstract
\citecode{koks_abstractexpander_2023} as it contains all the routines required
implementations of \citecode{koks_iexecutioninteractor_2023} like Harvesters,
Rejuvenators, Pre-, and PostProcessors.

\subsection*{Prerequisite 3: Implements
\code{koks_abstractexpanderdependencymanagerinteractor_2023}}
The Dependency Injection pattern is an interesting and beneficiary pattern that fully
adheres to the \gls{dip} principle. By implementing the abstract
\citecode{koks_abstractexpanderdependencymanagerinteractor_2023} all, for the expander
specific implementations of the following interfaces will automatically be registered and
made available for runtime processing.

\begin{itemize}
    \item The expander which should be an implementation of
    \citecode{koks_iexpanderinteractor_2023} or \citecode{koks_abstractexpander_2023}
    \item When applicable, the expander handers, which should be an implementation of \citecode{koks_iexpanderhandlerinteractor_2023}
    \item The default \citecode{koks_regionharvesterinteractor_2023} will automatically be
    executed during the generation process.
    \item Specific implementations of Harversters, which should be an implementation of \citecode{koks_iharvesterinteractor_2023} 
    \item The default \citecode{koks_regionrejuvenatorinteractor_2023} will automatically be
    executed during the generation process.
    \item Specific implementations of Rejuvenators, which should be an implementation of \citecode{koks_irejuvenatorinteractor_2023}
    \item The default PostProcessor \citecode{koks_installdotnettemplateinteractor_2023} will automatically be
    executed during the generation process.
    \item Specific implementation of PostProcessors, which should be an implementation of \citecode{koks_postprocessorinteractor_2023}
    \item The default PreProcessor \citecode{koks_uninstalldotnettemplateinteractor_2023} will automatically be
    executed during the generation process.
    \item Specific implementation of PreProcessors, which should be an implementation of \citecode{koks_preprocessorinteractor_2023}
\end{itemize}


\subsection{Plugin Architecture} \label{subsec_plugin_architecture}

When all preconditions are met and the expander is compiled, the expander consists of a
\gls{dll} and a set of templates. The Generator artifact considers the expanders as
optional plugins, which are dynamically loaded at runtime, through a method called
assembly-binding. See section \fullref{subsec_expanders} for a full explanation of the
required preconditions.

\begin{figure}[H]
  \centering
  \includegraphics[width=1\textwidth]{Figures/plugin_architecture.pdf}
  \caption[Plugin Archticture]{Expanders are considered plugins}
  \label{fi:plugin_architecture}
\end{figure}

By implementing the expanders as plugins, the design adheres to several \gls{solid}
principles. First and foremost, \gls{srp} is respected because an expander should generate
one, and precisely one construct \parencite[403]{mannaert_normalized_2016}. In the case of
this research, this construct is an application following the design and principles of the
design approach of \ca with a Restful API interface. Furthermore, it supports the
\gls{ocp} principles because developers can introduce a new version of the expander as a
separate plugin if this is required. \gls{lsp}. By extension, this also adheres to the
\gls{lsp} principle, as expanders can be replaced with other implementations of expanders,
without affecting the rest of the application.
\subsection{The executer object} \label{subsec:IExecutionInteractorObject}

An important implementation that facilitates a high degree of cohesion, whilst maintaining
low coupling and adhering to the \gls{srp} principle is the use of the
\citecode{koks_iexecutioninteractor_2023} interface. The generation process is designed to
execute \enquote{tasks} in a predefined order. By using the
\code{koks_iexecutioninteractor_2023} it is possible to design each of the tasks as
separate classes, fully complying, or enabling all of the \gls{solid} principles. The 

\begin{figure}[H]
    \centering
    \includegraphics[width=1\textwidth]{Figures/class_diagram_iexecutioninteractor.pdf}
    \caption[Plugin Archticture]{Both high cohesion and low coupling by using the \code{koks_iexecutioninteractor_2023}}
    \label{fig:iexecutioninteractor}
  \end{figure}


As depicted in listing \ref{SipCodeGeneratorInteractor}, this design leads to a cohesive
design where all tasks are gracefully executed from a single point in the application.
\lstinputlisting[ caption={The \citetitle{koks_codegeneratorinteractor_2023}},
label={SipCodeGeneratorInteractor} ] {Snippets/CodeGeneratorInteractor.cs}
\section{The meta-model} \label{sec:artifact_meta_model}


