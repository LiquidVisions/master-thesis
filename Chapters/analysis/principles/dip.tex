\subsection{Dependency Inversion Principle}

\evaluatePrincipleTable{\gls{dip}}{table_dip_convergence}{ 
    
\addEvalRow{\gls{soc} & \npartialConvergence & \gls{dip} states that high-level modules should
not depend on low-level modules.  By adhering to
\gls{dip} correctly, the architecture promotes modular architectures and the use of
component layers, as described in \fullref{subsec_dependency_rule}
\parencite{koks_layers_2023}. Managing Dependencies inheritly promotes \gls{soc},
therefore \gls{dip} converges with \gls{soc} to some extend. However, adhering to \gls{soc}
does not guarantee \gls{soc}.}
    
\addEvalRow{\gls{dvt} & \noConvergence & The \gls{dip} does not relate to the \gls{dvt}
principle. The convergence of both principles is weak, and no manifestations are found in
the artifacts.}
    
\addEvalRow{\gls{avt} & \npartialConvergence & The \gls{dip} can support the \gls{avt}
principle. The \gls{avt} principle emphasizes the importance of isolating actions or
operations within a system. By adhering to \gls{dip}, the architecture simplifies the
dependency management of those isolated versions of actions, which may contribute to
achieving \gls{avt}. The artifact's handling of this is already described in Chapter
\fullref{subsubsec_dip}. However, the convergence between \gls{dip} and \gls{avt} is not so
strong as with \gls{soc}, and adhering to \gls{dip} alone will not guarantee a system that
entirely complies to \gls{avt}. } 
    
\addEvalRow{\gls{sos} & \noConvergence & The \gls{dip} does not relate to the \gls{sos}
principle. The convergence of both principles is weak, and no manifestations are found in
the artifacts.} 

}
