\subsection{Converging the Single Responsibility Principle}

\begin{table}[H]
    \begin{tabular}{ l | c | p{0.80\linewidth}}
        \toprule
        \gls{soc} & \conv & \gls{srp} and \gls{soc} share a common objective:
        facilitating evolvable software systems through the promotion of modularity, low
        coupling, and high cohesion. While there may be some differences in granularity
        when applying both principles according to the original definition of SOC, the
        more stringent definition of Separation of Concerns, offered by the \gls{ns}
        Theorems \ref{subsubsec_soc} minimizes these differences. As a result, the two
        principles can be regarded as practically interchangeable. In conclusion, SRP and
        SOC exhibit full convergence, as they both emphasize encapsulating
        'responsibilities' or 'concerns' within modular components of a software system.
        \\
        \midrule
        \gls{dvt} & \partconv & While not immediately apparent, \gls{srp} offers supports
        for the \gls{dvt} theorem. While \gls{srp} emphasizes limiting the responsibility of
        each module, it does not explicitly require handling changes in data structures.
        However, following gls{srp} can still indirectly contribute to achieving \gls{dvt}
        by promoting the Law of Demeter
        \footnote{\url{https://en.wikipedia.org/wiki/Law_of_Demeter}}, which encourages
        modules to interact with each other only through well-defined interfaces. This
        approach can minimize the impact of data structure changes, although it does not
        guarantee full convergence with \gls{dvt}. \\
        \midrule
        \gls{avt} & \partconv & Although not that apparent, \gls{srp} supports the \gls{dvt}
        theorem. While \gls{srp} emphasizes limiting the responsibility of each module, it
        does not explicitly require handling changes in data structures. However, following
        gls{srp} can still indirectly contribute to achieving \gls{dvt} by promoting the Law
        of Demeter \footnote{\url{https://en.wikipedia.org/wiki/Law_of_Demeter}}, which
        encourages modules to interact with each other only through well-defined interfaces.
        This approach can minimize the impact of data structure changes, although it does not
        guarantee full convergence with \gls{dvt}. \\
        \midrule
        \gls{sos} & \noconv & The convergence between \gls{srp} and the \gls{sos} theorem is
        not as direct as with other theorems. \gls{srp} focuses on assigning a single
        responsibility to each module but does not explicitly address state management.
        Nevertheless, by following \gls{srp}, developers can create modules that manage their
        state, which indirectly contributes to \gls{sos}. \\
        \bottomrule
    \end{tabular}
    \caption{Converge \gls{srp} with \gls{ns}}
    \label{tab_convergence_srp}
\end{table}
