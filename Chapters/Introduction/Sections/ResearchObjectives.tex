\section{Research Objectives} \label{sec_research_objectives}

In this Design Science Research, we will shift the focus from Research Questions to
Research Objects. The Object of this research is to determine the degree of convergence of
Clean Architecture, with the Normalized Systems Theory. 

To summarize, we will start with a generic design that is based on the principles and
design characteristics of Clean Architecture. Given this design, we will conduct the
research on two different artifacts, namely a code-generator artifact and a generated
artifact. The reason for the code-generator artifact is to provide strict and meticulous
adherence to the generic design. Each entity should be implemented in exactly the same way.

AFMAKEN



% In this Design Science research, the focus shifts from a research question toward
% research objectives. The following objectives apply to this research.

% \subsection*{Objective 1: The expander artifact}
% A C\# artifact that consists of a source code expander and some helper classes to
% generate the second \emph{expanded} artifact. The artifact supports the harvesting and
% rejuvenation of custom code snippets on the expanded artifact so that regenerations do not
% have a loss of implementations on the expanded artifact. The expander artifact is entirely
% based on the design principles of \gls{ca}.

% Chapter \ref{sec_generator_artifact} entirely describes the expander artifact.

% \subsection*{Objective 2: The expanded artifact}
% This artifact is an entirely working restful API, that is based on ASP.NET and C\#. It has
% essential CRUD support. The artifact has been generated by the \emph{expander} and has
% been enriched with custom code snippets to comply with the initial requirements.
% The expanded artifact is entirely based on the design principles of \gls{ca}.

% Chapter \ref{sec_generator_artifact} entirely describes the expander artifact.

