\subsubsection{The Dependency Inversion Principle} \label{subsubsec_dip} 

The \gls{dip} prescribes that high-level modules should not depend on low-level modules
and that both should depend on abstractions. The principle emphasizes that the
architecture should be designed so that the flow of control between the different objects,
layers, and components is always from higher-level implementations to lower-level details.
In other words, high-level implementations, like business rules, should not be concerned
about low-level implementations, such as how the data is stored or presented to the end
user. Additionally, both the high-level and low-level implementations should only depend
on abstractions or interfaces that define a contract for how they should interact with
each other \parencite[91]{robert_c_martin_clean_2018}. 

This approach allows for great flexibility and a modular architecture. Modifications in
the low-level implementations will not affect the high-level implementations as long as
they still adhere to the contract defined by the abstractions and interfaces. Similarly,
changes to the high-level modules will not affect the low-level modules as long as they
still fulfill the contract. This reduces coupling and ensures the evolvability system over
time, as changes can be made to specific modules without affecting the rest of the system.