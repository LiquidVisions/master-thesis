\subsubsection{The Interface Segregation Principle} \label{subsubsec:isp}

The \gls{isp} suggests that software components should have narrow, specific interfaces
rather than broad, general-purpose ones. The \gls{isp} states that no client code should
be forced to depend on methods it does not use. In other words, interfaces should be
designed to be as small and focused as possible, containing only the methods that are
relevant to the clients that use them. This allows clients to use only the methods they
need, without being forced to implement or depend on unnecessary methods
\parencite[104]{robert_c_martin_clean_2018}.

Overall, the \gls{isp} is about designing interfaces that are tailored to the specific
needs of the clients that use them, rather than trying to create one-size-fits-all
interfaces that may be bloated or unwieldy.

Take a look at Listing \ref{SipISPExample}. In order to comply with the \gls{isp} the
design decision was made to separate all \gls{crud} operations into separate interfaces.
In the example of the \citecode{koks_appseederinteractor_2023} (see \ref{SipISPExample})
only the delete and create gateways where required. An alternative approach was to create
an IGateway interface containing all of the \gls{crud} operations. Following this approach
would lead to dependencies to all \gls{crud} operations in the
\code{koks_appseederinteractor_2023}.

\lstinputlisting[
    caption={The Gateways for Create, Read, Update, Delete operations},
    label={SipISPExample}]
    {Snippets/ISP_Example.cs}