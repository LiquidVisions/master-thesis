\section{Research Objectives and Questions} \label{sec:research_questions}

In this Design Science research, the focus shifts from a research question towards
research objectives. The following objectives apply to this research.

\subsection{Objective 1: The expander artifact}
A C\# artifact that consists of a source code expander and some helper classes in order to
generate the second \emph{expanded} artifact. The artifact supports the harvesting and
rejuvenation of custom code snippets on the expanded artifact so that regenerations do not
have a loss of implementations on the expanded artifact. The expander artifact is fully
based on the design principles of Clean Architecture.

\subsection{Objective 2: The expanded artifact}
A fully working restful API based on ASP.NET and C\# with basic CRUD support. The artifact
has been generated by the \emph{expander} and has been enriched with custom code snippets
in order to comply with the initial requirements. The expanded artifact is fully based on
the design principles of Clean Architecture.

The objective of this research is to address violations of Normalized Systems principles
and implementation elements on artifacts that are built with C\# programming language
based on the Clean Architecture approach. In order to investigate this the following
research questions will be researched:

\subsection{Research question}
In order to test the hypothesis described in \ref{hypothesis} the following research
questions can be applied to both research objectives.

RQ1: which violations of the Normalized Systems Theorems can be found as an effect of
following the Clean Architecture principles on both the expander- and expanded artifact?

RQ2: Which of the Normalized Systems Theorems are excluded as an effect of following the
Clean Architecture principles on both the expander- and expanded artifact?

When no violations or excludes on the Normalized Systems Principles are found in the
artifacts, it is assumed that there will be no combinatorial ripple effect on changing
both artifacts. Only in that case, it can be concluded that Clean Architecture can be
used in conjunction with the Normalized Systems theorems. 