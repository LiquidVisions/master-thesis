\section{Research method} \label{sec_research_method}

This research is a Design Science Method and relies on the Engineering Cycles as described
by \textcite{wieringa_design_2014}. The engineering cycle provides a structured approach
to developing the required artifacts to analyze the design problem.

\begin{figure}[H]
    \centering
    \includegraphics[width=1\textwidth]{figures/engineering_cycle.pdf}
    \caption[Engineering cycle]{The Engineering Cycle of \textcite{wieringa_design_2014}}
    \label{fig_engineering_cycle}
\end{figure}

In the context of this research, the artifacts described in chapters
\ref{sec_generator_artifact} and \ref{sec_generated_artifact} are considered information
systems. \citeauthor{hevner_design_2004} proposed a framework for research in information
systems by introducing the interacting relevance and rigor cycles.

Figure \ref{fig_dsr} depicts a specialization of the Design Science Framework of
\textcite{hevner_design_2004}. The rigor cycle comprises the theories and knowledge from
\gls{ns} and \gls{ca}, supplemented by the rigorous knowledge of modularity, evolvability,
and stability of software systems. The relevance cycle represents the business needs of
the stakeholders. The research requirements are described as research objectives.

\begin{figure}[H]
    \centering
    \includegraphics[width=1\textwidth]{figures/rigor_relevance_cycle.pdf}
    \caption[Design Science Framework for IS Research]{The Design Science Framework for IS Research}
    \label{fig_dsr}
\end{figure}