\subsection{Expansion and code generation} \label{subsec:expansion}

Creating and maintaining a stable and evolvable system is a particularly challenging and
meticulous engineering job. Developers are required to have a sound knowledge of \gls{ns},
whilst implementing new requirements in an always consistent manner. Given the required
recurring structure, it can be perceived as a repetitive and therefore boring tasks.
\parencite[219]{mannaert_normalized_2016}. On top of that, rejecting the temptation to
take shortcuts to follow the business's time-to-market requirements.

Given the above, it is logical to automate the instantiation process of software
structures and use code generation for recurring
tasks\parencite[403]{mannaert_normalized_2016}. This is where code expansion comes in
place. The concept of code expansion does not only refer to the automatic process of
adapting and maintaining software to new requirements, architectural enablers and
technological alterations. It also embraces manually added craftings to the software, the
so-called plugin code. These craftings are preserved after each expansion by a method that
is called harvesting and rejuvenation \parencite[405-406]{mannaert_normalized_2016}.

