\subsubsection*{The Liskov Substitution Principle} \label{subsubsec:lsp}


The \gls{lsp} is a fundamental concept in object-oriented programming that deals with the
behavior of derived objects (aka sub-types). The principal is named after Barbara Liskov
who first introduced the principle in a paper she co-authored in 1987. Barbara Liskov
wrote the following statement as a way of defining subtypes
\parencite{robert_c_martin_clean_2018}.

\mycolorbox{If for each	object o1 of type S there is an	object o2 of type T such that for
all programs P defined in terms of T, the behavior of P	is unchanged when o1 is
substituted for	o2 then S is a subtype of T.1}{\acrlong{lsp}}

In simpler terms: If you have an object \textit{Volvo} of type \textit{Vehical}, it should be
possible to substitute it for an object \textit{Toyota} of type \textit{Vehical} in any
program that was defined in terms of \textit{Vehical}, without affecting the program's
correctness. This applies to all programs, not just a specific one.

The principle is based on the idea that a subtype should be semantically substitutable for
its base type. This means that the subtype should behave in a way that is consistent with
the expectations of the base type and should not introduce any new behaviors or violate
any of the constraints imposed by the base type.

The practical implications of \gls{lsp} are many. In software design, it means that we
should strive to create subtypes that are as similar as possible to their base types in
terms of their behavior and the constraints they impose. In testing, it means that we
should test subtypes to ensure that they behave correctly when used in place of their base
types.

Consider \citecode{koks_abstractexpander_2023}. This (abstract) object type allows for
multiple implementations of \textit{Expanders}. The main example is the
\citecode{koks_cleanarchitectureexpander_2023} which is responsible for generating the
expanded artifact that is part of this research. Different types of expanders could be
added to the generator, ensuring they all behave in the same way.

The \citecode{koks_icreategateway_2023} in Listing \ref{SipICreateGateway} is another
example. The artifact has two implementations of this interface. The data of the entities
are currently stored in the database, but harvest data is serialized to XML using the same
\code{koks_icreategateway_2023} interface. With this design decision, it is very easy to adapt
to a different type of storage mechanism if future requirement demand such a change.

One might notice the similarities with the \gls{ocp}. The difference is that the \gls{ocp}
focuses on the extensibility of the system, without having to modify existing code.
\gls{lsp} ensures that the behavior of different subtypes is following the required
functionality. \gls{lsp} supports \gls{ocp}, but it is not the only way of doing so.

\lstinputlisting[
    caption={The \citetitle{koks_icreategateway_2023}},
    label={SipICreateGateway} ]
    {Snippets/ICreateGateway.cs}

\lstinputlisting[
    caption={The \citetitle{koks_genericrepository_2023} and \citetitle{koks_harvestrepository_2023}},
    label={SipICreateGatewayImplementations} ]
    {Snippets/ICreateGatewayImplementations.cs}

