\section{Towards evolvable software architectures} \label{sec:on_modules}

There are a couple of aspects concerning evolvable software architectures. One of which is
stability, as described in chapter \ref{sec:on_stability} \nameref{sec:on_stability}.
Another factor that impacts evolvability is the modularity of the architecture. There is a
wide consensus about two fundamental rules when thinking of-, and designing modularity:
\emph{high cohesion} and \emph{low coupling} \autocite[22]{mannaert_normalized_2016}.
These will

Software modules can be defined as self-contained units of code that perform specific
tasks or sets of tasks within a larger system. A software module is designed to operate
independently of other modules, with well-defined interfaces that allow it to communicate
and exchange data with other modules if necessary \autocite[22]{mannaert_normalized_2016}.

A module can be considered a hierarchical and recursive concept. They are independent of
their size (lines of code) or computational magnitude. They can be as small as a function
as part of a class. The class itself can also be considered a module. A group of classes
contained in a Dynamic Link Library (DLL) or Application Programming Interface (API) can
also be considered a module of an even bigger system. 

An important part of the design of a software system is to identify the possible different
modules and their interaction interfaces.

\subsection{Cohesion: The beauty of Software Desing} \label{subsec:on_modularity}

The term cohesion denotes the extent to which the various structural components of a
software system operate cohesively towards a singular and well-defined objective or goal.
Empirical studies in software engineering have extensively demonstrated the significance
of cohesion, linking higher levels of cohesion with reduced defects, enhanced
maintainability, and greater openness to change. Consequently, achieving high cohesion has
been associated with an overall improvement in software quality attributes such as
reliability, maintainability, reusability, and evolvability.

Cohesion facilitates the reduction of complexity and interdependence among the components
of a system, thereby contributing to a more efficient, maintainable, and reliable system.
By organizing components around a shared purpose or function, or by standardizing their
interfaces, data structures, and protocols, cohesion can offer the following benefits:

\begin{itemize}
    \item \textbf{Reduce redundancy and duplication of effort}: \\
    Cohesion ensures that components are arranged around a common purpose or function,
    reducing duplicates or redundant code. This simplifies system comprehension,
    maintenance, and modification.
    \item \textbf{Promoting code reuse:}\\
    Cohesion facilitates code reuse by making it easier to extract and reuse components
    designed for specific functions. This saves time and effort during development and
    enhances overall system quality.
    \item \textbf{Enhance maintainability:}\\
    Cohesion decreases the complexity and interdependence of system components, making it
    easier to identify and rectify bugs or errors in the code. This improves system
    maintainability and reduces the risk of introducing new errors during maintenance.
    \item \textbf{Increase scalability:}\\
    Cohesion improves a system's scalability by enabling it to be extended or modified
    effortlessly to accommodate changing requirements or conditions. By designing
    well-organized and well-defined components, developers can easily add or modify
    functionality as needed without disrupting the rest of the system.  
\end{itemize}

\subsection{Coupling: The beast of Software Desing} \label{subsec:on_coupling}

Coupling is a central concept in software engineering that pertains to the degree of
interdependence among software modules or components. The level of coupling between
modules denotes the strength of their relationship, whereby a high level of coupling
implies a significant degree of interdependence. Conversely, low coupling signifies a
weaker relationship between modules, where modifications in one module are less likely to
impact others.

The negative impacts of excessive coupling on software systems are considerable. High
coupling can render software systems difficult to maintain, modify, or evolve. It can
impede the identification and resolution of errors within a system, leading to prolonged
debugging periods. Additionally, it can cause fragility in the system, where slight
modifications in one module can trigger cascading failures throughout the entire system.
Therefore, it is crucial for software engineers to minimize coupling between modules while
maintaining a cohesive design. By developing systems with low coupling, software engineers
can construct more maintainable, scalable, and adaptable systems that are easier to evolve
over time.

Coupling, in software engineering, can take several forms, including content, common,
control, stamp, and data coupling. Content coupling occurs when one module accesses or
modifies the internal data or logic of another module, leading to high interdependence and
difficulty in isolating errors. Common coupling occurs when several modules access and use
the same global data, increasing their interdependence and reducing modularity. Control
coupling occurs when one module controls the execution flow of another module, making it
difficult to modify or reuse the controlled module. Stamp coupling arises when two modules
share a common data structure, leading to tight coupling and high interdependence.
Finally, data coupling exists when two modules share data, which can lead to coupling
between them.

To avoid the negative impacts of coupling on software systems, software engineers should
aim to minimize the degree of coupling between modules. This can be achieved by designing
cohesive, loosely coupled modules with well-defined interfaces. Loose coupling enables
each module to operate independently, reducing the impact of modifications made to other
modules. By implementing modular design principles, such as high cohesion and low
coupling, software engineers can develop systems that are easier to maintain, test, and
evolve.

In conclusion, coupling is a critical concept in software engineering that can have a
considerable impact on the maintainability, flexibility, and scalability of software
systems. By minimizing coupling between modules, software engineers can develop more
robust, adaptable systems that are easier to modify and evolve. Adopting modular design
principles can also facilitate the development of cohesive, loosely coupled modules that
enable the independent operation and reduce the impact of modifications made to other
modules.