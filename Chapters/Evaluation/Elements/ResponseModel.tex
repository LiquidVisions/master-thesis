\subsection{Converging the ResponseModel element}

\begin{table}[H]
    \begin{tabular}{ l | c | p{0.70\linewidth}}
        \toprule
        Data & \partconv & Both elements represent data objects that are part of the
        ontology or data schema of the application, and typically include attributes and
        relationship information. While they may contain a specific set of information as
        output for a Task or use case, both elements can also contain a full set of
        attributes and relationships. However, unlike the Data entity in \gls{ns}, which
        may include only a subset of information necessary for a specific Task or use
        case, it may also include the full set of information required for Tasks other
        purposes. \\
        \midrule

        Task & \noconv & The ResponseModel is not convergent with the Task element of
        \gls{ns}. However, the Tasks element might operate on RequestModels as input
        parameters to perform business logic. \\ \midrule
        
        Flow & \noconv & The ResponseModel and Flow are not convergent, as the Flow element
        represents the control between Tasks in \gls{ns}, while the ResponseModel in \gls{ca}
        represents (parts of) domain objects.\\ \midrule
        
        Connector & \noconv & The ResponseModel element and Connector element are not
        convergent, as the Connector element in \gls{ns} is involved in communication
        between components, whilst the ResponseModel in \gls{ca} represents (parts of)
        domain objects.\\ \midrule
        
        Trigger & \noconv & The ResponseModel element and Trigger element are not convergent,
        as the Trigger element in \gls{ns} is about event-based execution of Tasks, while
        the ResponseModel in \gls{ca} represents (parts of) domain objects.\\
        
        \bottomrule
    \end{tabular}
    \caption{The convergence of the ResponseModel element}
    \label{tab_convergence_responsemodel}
\end{table}