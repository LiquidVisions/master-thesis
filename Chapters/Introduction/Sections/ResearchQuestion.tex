\section{Research question} \label{sec_research_questions}

The objective of this research is to address violations of \gls{ns} Principles
and implementation elements on artifacts that are built with C\# programming language
based on the \gls{ca} approach. In order to test the hypothesized outcome
described in \ref{hypothesis}, the following research questions are formulated. Each
research question applies to both of the produced artifacts.

\subsection*{Research Question 1:} \label{rq1}
What violations of the \gls{ns} Theorems can be found as an effect of
following the \gls{ca} principles on both the expander- and expanded artifact?

\subsection*{Research Question 2:} \label{rq2}
Which of the \gls{ns} Theorems are excluded as an effect of following the
\gls{ca} principles on both the expander- and expanded artifact?

\subsection*{Research Question 3:} \label{rq3}
To what extent do the SOLID principles of \gls{ca} converge with each of the
\gls{ns} theorems, and how can these principles be effectively integrated to
improve software design?

When no violations or excludes on the \gls{ns} Principles are found in the
artifacts, it is assumed that there will be no combinatorial ripple effect on changing
both artifacts. Only in that case can it be concluded that \gls{ca} can be
used in conjunction with the \gls{ns} theorems. 