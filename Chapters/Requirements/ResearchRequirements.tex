\section{Research requirements} \label{sec_research_requirements} 

In order to address the research objectives we outlined in Chapter
\ref{sec_research_objectives}, we need to know what violations in the context of \ns
can occur in a software artifact. In chapters \ref{sec_ns_theory} we have shown that
\ns attempts to achieve evolvability by achieving stability. Through the extensive
and consistent use of the \ns theorems, a modular design emerges that is free of
instabilities. Alternatively, in terms used by the \ns theorem: the absence of
combinatorial effects. However, as described in section \ref{sec_artifact_requirements},
the artifacts are designed based on the principles and design approach of \ca. The
\ns theorems are not considered using the design phase of the artifacts. 

To be able to analyze the stability of the artifacts the following functional requirement
specifications are reused \parencite[254-259]{mannaert_normalized_2016}. They are noted
without the mathematical formulas that are present in cited sources.


\mycolorbox{An information system needs to be able to represent instances of data
entities. A data entity consists of several data fields. Such a field may be a basic
data field representing a value of a reference to another data entity.}
{Research Requirement 1}

\mycolorbox{An information system needs to be able to execute processing actions on
instances of data entities. A processing action consists of several consecutive processing
tasks. Such a task may be a basic task, i.e., a unit of processing that can change
independently, or an invocation of another processing action.} {Research Requirement 2}

\mycolorbox{An information system needs to be able to input or output values of instances of data entities through connectors.}
{Research Requirement 3}

\mycolorbox{An existing information system representing a set of data entities, needs
to be able to represent: 
\begin{itemize}
    \item a new version of a data entity that corresponds to including an additional
    data field
    \item an additional data entity 
\end{itemize}}
{Research Requirement 4}

\mycolorbox{An existing information system providing a set of processing actions,
needs to be able to provide:
\begin{itemize}
    \item a new version of a processing task, whose use may be mandatory
    \item a new version of a processing action, whose use may be mandatory
    \item an additional processing task
    \item an additional processing action
\end{itemize}}
{Research Requirement 5}