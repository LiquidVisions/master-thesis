\chapter{Introduction} \label{introduction}

\enquote{\emph{Pantha Rhei}} is, according to \emph{Plato}, one of the famous
philosophical statements first described by the Greek philosopher
\emph{Heraclitus}\footnote{\url{https://plato.stanford.edu/entries/process-philosophy/}}.
Translated as \enquote{everything flows} this statement is an unambiguous commitment to
the ubiquitous dynamics of everything that exists. \enquote{Life is flux}, one of the
constants in life, is change.

In the realms of Software Engineering, the \enquote{laws of software evolution}
\parencite[]{lehman_programs_1980} refers to a series of laws described by
\citeauthor{lehman_programs_1980}. With these Laws, he describes the balance between the
forces driving new developments on the one hand (a change) and the forces that slow down
progress on the other hand. Based on \emph{Heraclitus} philosophical statement, this paper
assumes that software engineering projects will frequently be subjected to change,
probably due to changing market demands or technological progress. 

More than a half of century of software engineering-, and architecture practices show that
the complexity of these software artifacts gradually increases over time. Eventually, this
will render most of the software artifacts obsolete, according to
\citeauthor{lehman_programs_1980} \parencite[]{lehman_programs_1980}.

We also observe that contemporary competitive corporate environments are changing
continuously. The speed at which these changes emerge is also increasing. To stay
competitive it is essential to be able to deal with corporate change and shifting market
demands. \todo{add a citation to the work of Steven de Haes}This is partly driven by
contemporary digital transformations. (IT) Organizations are attempting to cope with this
trend by adopting agility and maturing their agile practices
\parencite[]{2024_SIM_key_issues_and_trends}. Therefore, agility has been proposed as a
measure for contemporary organizations to adapt to new environments and cope with rapid
change \parencite[]{neumann_strategic_1994}.

Over the years, a lot of research is done to mitigate the risks of software artifact
degradation over time. There is also plenty of experience described in various books from
corporate experience. Two of these are part of this research are the Theorems of
Normalized Systems and Uncle Bob's Clean architecture.

\section{Introducing Normalized Systems Theorems}

The Normalized Systems theorems are a scientific approach to creating software systems
based on the laws for software evolvability. These theorems have resulted in a documented
track record of achieving software stability, in a scientific environment. Effectively, it
prevents the accumulation of combinatorial effects on anticipated change drivers. This
prevents the positive feedback loop and prevents the degradation of the software artifact.
Preventing positive feedback loops has a positive effect on the evolvability of software
artifacts.\parencite[]{mannaert_normalized_2009}. 

\citeauthor[]{mannaert_normalized_2009} have formulated the theorem of Normalized Systems
as prescriptive structures (elements) that lead to a modular architecture with low
coupling and high cohesion. The resulting software architecture will be designed to cope
with future change \parencites[]{mannaert_normalized_2009}.
\section{Introducing Clean Architecture}

Clean Architecture is the accumulation of more than half a century of coding, designing,
and architecting software systems by Martin, Robert C. He published experience in his book
\citetitle*[]{martin_clean_2018} in \citeyear[]{martin_clean_2018}. He states that
creating a programming artifact does not require much skill and knowledge. However, creating
stable and evolvable software artifacts is a different skill that requires
a lot of knowledge, skill, dedication, and time.

The book's goal is to have a software architecture that minimizes the human resources
required to build and maintain the information system. Just like Normalized Systems, it
has a prescribed design of software classes that will lead to a modular architecture with
low coupling and high cohesion \parencite{martin_clean_2018}.

\section{Research relevance} \label{sec:research_relevance} 


A lot of research has been done about the modularity, stability and evolvability of
software artifacts. Specifically in the \todo{finish this part}

In chapter~\ref{ns_theory}~\nameref{ns_theory} the Normalized Systems Theorems are
described to be grounded in science. Normalized Systems prescribe modular software
architectures by applying a strict set of design principles. Throughout extensive research
and practical experience, there is both scientific and practical proof that Normalized
Systems contribute to the stability and evolvability of a software artifact.

Clean Architecture can be described as a software architecture philosophy that has emerged
from the experience of its author. According to the author, a highly maintainable and
modular software architecture can be achieved by following the design principles and
applying architectural rules to the software artifact.

Since the introduction of Normalized Systems Theorems, Java EE (Java Enterprise Edition)
has been the prevalent programming language used in scientific research settings. It has
been used to create the evolvability of software architectures based on the stability
concepts of systems theory \parencite[]{mannaert_towards_2012}.

\citeauthor{mannaert_towards_2012} stated in the paper that the design theorems were
formulated as modular structures that are independent of any software language or
development paradigm. The applicability of Normalized Systems Theorems with

Java EE is still a prevalent programming language for enterprise- and IT organizations.
Many software solutions are created and maintained using this programming language.
\section{Hypothesised outcome} \label{hypothesis} 

The proposed hypothesis is that the \gls{ca} approach can be applied in conjunction with
the \gls{ns} theorems. The design principles of \gls{ca} converge with the theorems of
\gls{ns}. Consequently, the artifact that is part of this design research will lead to a
highly modular, stable, and evolvable C\# artifact that does not contradict Normalized
Systems theorems.

Both architectural approaches formulate modular structures independent of any programming
technology \parencite{mannaert_normalized_2009,robert_c_martin_clean_2018}. As such, the
C\# artifact produced as part of this research has similar trademarks of modularity,
evolvability, and stability compared to case studies where Java SE has been used
\parencite{oorts_building_2014, de_bruyn_enabling_2018}. Furthermore, the applicability of
\gls{ca} has no additional or negative effect when used in conjunction with the
\gls{ns} Theorems.

\begin{figure}[H]
    \centering
    \includegraphics[width=0.8\textwidth]{Figures/hypothesis.pdf}
    \caption[The hypothesis]{The hypothesis}
    \label{fig_hypothesis}
\end{figure}
\section{Research question} \label{sec:research_questions}
In order to test the hypothesis described in \ref{hypothesis} the following research
questions can be applied to both research objectives.

\subsubsection*{Research Question 1:}
What violations of the Normalized Systems Theorems can be found as an effect of
following the Clean Architecture principles on both the expander- and expanded artifact?

\subsubsection*{Research Question 2:}
Which of the Normalized Systems Theorems are excluded as an effect of following the
Clean Architecture principles on both the expander- and expanded artifact?

When no violations or excludes on the Normalized Systems Principles are found in the
artifacts, it is assumed that there will be no combinatorial ripple effect on changing
both artifacts. Only in that case, it can be concluded that Clean Architecture can be
used in conjunction with the Normalized Systems theorems. 
\section{Research model} \label{research_model}

When decomposing the statements made in section \ref{sec:research_relevance}
\nameref{sec:research_relevance} and the \nameref{sec:research_questions} the cause and
effect relationship between the independent-, and dependent variable can be determined. It
is clear that the independent variable 'Clean Architecture' has an effect on the outcome
of the design article that has been described in \ref{chap:artifact_design}
\nameref{chap:artifact_design}. 

Since there is plentiful scientific proof, the Normalized Systems theorems are used as the
baseline to measure the results. In the overall conceptual research framework, Normalized
Systems Theorems are considered to be the Moderator variable. In the context of this
research, Clean Architecture does not have a causal effect on Normalized Systems.

Since this research intends to demonstrate the level of convergence of Clean Architecture
with Normalized Systems, modularity, stability and evolvability are positioned as the
Mediator variable in this conceptual framework.

Figure \ref{fig:conceptual_framework} \nameref{fig:conceptual_framework} depicts a
graphical representation of the overall conceptual framework described above.

\begin{figure}[H]
    \centering
    \includegraphics[width=1\textwidth]{Figures/conceptual_framework}
    \caption[Overall conceptual framework]{Overall conceptual framework}
    \label{fig:conceptual_framework}
\end{figure}
