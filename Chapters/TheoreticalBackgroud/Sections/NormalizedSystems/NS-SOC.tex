\subsubsection*{Separation of Concerns} \label{subsubsec:soc}

\gls{soc} as a principle has first been mentioned by
\citeauthor{dijkstra_selected_1982}\footnote{\url{https://en.wikipedia.org/wiki/Separation_of_concerns}}
as the crucial principle to design modular software architecture
\parencite[]{dijkstra_selected_1982}. \gls{soc} promotes the idea that a program should be
divided into distinct sections, each addressing a separate concern or aspect of a design
problem. This allows for a more organized and maintainable source code. When implemented
correctly, a change to one concern does not affect the others. \gls{soc} should be applied
at the level of individual modules, rather than the level of an entire program.

\gls{soc} has been adopted as one of the design theorems of \gls{ns}, although it has a
stricter definition of this principle\parencite{mannaert_normalized_2016}.

\mycolorbox{A processing function can only contain a single task to achieve stability.}{Theorem I}