\chapter{Conclusions} \label{chap_conclusions}

\begin{enumerate}
    \color{red}

    \item \textbf{Literature Review}
    \begin{itemize}
        \item Ca offers structure, principles and guidelines on how to build something. On top
        of that, NST also offers guidelines in order to apply actual changes.
        \\gls{ca} has a strong emphasis on testability of code. Coupling is an important
        aspect on this.
        \item \gls{srp} differs fundamentally from \gls{soc} in definition, although in
        the artifact not that much (further explain). Komt vooral door verschillende
        definities in granulariteit.
    \end{itemize}
    
    \item \textbf{Architectural Desing}
    \begin{itemize}
        \item 
    \end{itemize}

    \item \textbf{Artifact Development}
    \begin{itemize}
        \item 
    \end{itemize}
    \begin{enumerate}[label*={\arabic*.}]
        
        \item \textbf{The Code Generator and Clean Architecture Expander}
        \begin{itemize}
            \item 
        \end{itemize}
        
        \item \textbf{Expanded Clean Architecture artifact}
        \begin{itemize}
            \item 
        \end{itemize}
        
    \end{enumerate}
    
    \item \textbf{Convergence Analysis:}
    \begin{itemize}
        \item 
    \end{itemize}
    \begin{enumerate}[label*={\arabic*.}]
        
        \item An analysis per principle of \gls{ca}, compared with each of the principles
        of \gls{ns}, indicating each level of convergence per principle
        
        \item An analysis per element of \gls{ca}, compared with each of the elements of
        \gls{ns}, indicating each level of convergence per principle
    
    \end{enumerate}
\end{enumerate}
