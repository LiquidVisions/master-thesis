\subsection{Coupling: The bad} \label{subsec:on_coupling}

Coupling is a central concept in software engineering that pertains to the degree of
interdependence among software modules or components. The level of coupling between
modules denotes the strength of their relationship, whereby a high level of coupling
implies a significant degree of interdependence. Conversely, low coupling signifies a
weaker relationship between modules, where modifications in one module are less likely to
impact others.

The negative impacts of excessive coupling on software systems are considerable. High
coupling can render software systems difficult to maintain, modify, or evolve. It can
impede the identification and resolution of errors within a system, leading to prolonged
debugging periods. Additionally, it can cause fragility in the system, where slight
modifications in one module can trigger cascading failures throughout the entire system.
Therefore, it is crucial for software engineers to minimize coupling between modules while
maintaining a cohesive design. By developing systems with low coupling, software engineers
can construct more maintainable, scalable, and adaptable systems that are easier to evolve.

Coupling, in software engineering, can take several forms, including content, common,
control, stamp, and data coupling. Content coupling occurs when one module accesses or
modifies the internal data or logic of another module, leading to high interdependence and
difficulty in isolating errors. Common coupling occurs when several modules access and use
the same global data, increasing their interdependence and reducing modularity. Control
coupling occurs when one module controls the execution flow of another module, making it
difficult to modify or reuse the controlled module. Stamp coupling arises when two modules
share a common data structure, leading to tight coupling and high interdependence.
Finally, data coupling exists when two modules share data, which can lead to coupling
between them.

To avoid the negative impacts of coupling on software systems, software engineers should
aim to minimize the degree of coupling between modules. This can be achieved by designing
cohesive, loosely coupled modules with well-defined interfaces. Loose coupling enables
each module to operate independently, reducing the impact of modifications made to other
modules. By implementing modular design principles, such as high cohesion and low
coupling, software engineers can develop systems that are easier to maintain, test, and
evolve.

In conclusion, coupling is a critical concept in software engineering that can have a
considerable impact on the maintainability, flexibility, and scalability of software
systems. By minimizing coupling between modules, software engineers can develop more
robust, adaptable systems that are easier to modify and evolve. Adopting modular design
principles can also facilitate the development of cohesive, loosely coupled modules that
enable the independent operation and reduce the impact of modifications made to other
modules.