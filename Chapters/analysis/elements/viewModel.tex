\subsection{The ViewModel Element} \label{converging_viewmodel_element}

\evaluateElementTable{ViewModel}{tab_convergence_viewemodel}{
    \addEvalRow{Data & \partialAlignment & The ViewModel and Data element of \gls{ns} is convergent to
        some degree. Both are involved in defining the structure of data used in the
        system. This could include required information about attributes and
        relationships. Additionally, the ViewModel could also represent information that
        is specifically intended for the representation of behavior for a user interface.}

    \addEvalRow{Task & \noAlignment & The ViewModel is not convergent with the Task element of
        \gls{ns}. The ViewModel is focused on presenting information, whilst the Task element
        is concerned with executing business logic.}

    \addEvalRow{Flow & \noAlignment & The ViewModel and Flow are not convergent, as the Flow element
        represents the control between Tasks in \gls{ns} and is not directly involved in
        the presentation of information.}

    \addEvalRow{Connector & \noAlignment & The ViewModel element and Connector element are not
        convergent, as the Connector element in \gls{ns} is involved in the communication
        between components, whilst the ViewModel in \gls{ca} is involved in the
        presentation of information.}
    
    \addEvalRow{Trigger & \noAlignment & The ViewModel element and Trigger element are not convergent,
        as the Trigger element in \gls{ns} is responsible for the event-based execution
        of Tasks, whilst the ViewModel in \gls{ca} is involved in the presentation of
        information.}
}