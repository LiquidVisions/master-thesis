\section{Dependency management} Dependency management is an extremely valuable aspect of
achieving stability and evolvability. Dependency management can be achieved by using
Dependency Injection. This research acknowledges the statement of
\textcite[215]{mannaert_normalized_2016} that Dependency Injection does not solve coupling
between classes. Working on the Artifact has shown that combinatorial effects can occur
when not careful. Nevertheless, Dependency Injection is a widely used pattern in building
the Artifact. In order to achieve stability and evolvability, the Dependency Injection
pattern \underline{must} be combined with various other principles of both \gls{ca} and
\gls{ns}. 

The goal is to centralize the management of dependencies and remove unwanted manual object
instantiations in the code. Al this while respecting the \gls{dip} principle so that each
Component Layer is responsible for managing its dependencies. The Artifact achieves this
by using extension methods as illustrated in Code Listing
\ref{list_DependencyInjectionExtension}
\parencite{koks_dependencyinjectionextension_2023}. Additionally, and quite significantly,
implementations primarily rely on abstractions or contracts (interfaces) instead of the
details of concrete implementations. 

Traditionally, Dependency Injection injects instantiations through constructor parameters
or class properties. Although there are benefits in this approach, doing so will
eventually lead to combinatorial effects, breaking the stability of a Software Artifact.
In order to solve this problem, the Artifact used the Service Locator pattern, a central
registry responsible for resolving dependencies \parencite{wikipedia_service_2023}. Many
frameworks are available from \gls{nuget}, but the Artifact uses the Service Registry that
is part of the .NET framework. This service registry is considered a cross-cutting
concern. The dependency on this technology is reduced by applying the principles of the
\gls{lsp} and \gls{isp}. The Artifact creates and uses separate interfaces to register
\parencite{koks_idependencymanagerinteractor_2023} and resolve
\parencite{koks_idependencyfactoryinteractor_2023} dependencies. As illustrated in Code
Listing \ref{list_DependencyManager}, the framework technology dependency is abstracted
behind implementing those interfaces \parencite{koks_dependencymanagerinteractor_2023}. 

Practically every class gets the \citecode{koks_idependencyfactoryinteractor_2023}
injected, on which further resolving is responsible for that class's inner workings. Code
Listing \ref{list_Injecting} illustrates how this is done in the
\citecode{koks_abstractexpander_2023} class. Finally, all the dependencies are
bootstrapped on application bootup, depicted in Code Listing \ref{list_dip}. 

The approach described here has many advantages in managing the stability and evolvability
of the Software Artifact. However, as for most things, there are also some drawbacks. For
example, a good amount of experience is required for developers to understand code that
incorporates abstractions, contracts, and Dependency Injection. Another drawback is that
dependency errors are detected in runtime rather than compile time. The benefits of the
Artifacts, however, outweigh the drawbacks.