\section{Normalized Systems: Impacting software stability} \label{sec_ns_theory}

\gls{ns} is a software development approach that prioritizes achieving software stability
through the use of standardized, modular components and interfaces. This theory is
informed by several scientific disciplines, including systems theory, mathematics, and
computer science, as well as some other software development approaches, such as agile
development and domain-driven design.

\gls{ns} originated in the field of software engineering. However, the underlying theory
of \gls{ns} can be applied to various other domains, such as Enterprise Engineering,
Business Process Modeling, and document management. This research acknowledges the
software engineering background of \gls{ns}. It consistently refers to software and
Information Systems when referring to \enquote*{artifacts.} However, the reader should
realize that the concepts and artifacts are not restricted to software artifacts alone.

\subsection{Stability} \label{subsec_on_stability}

\gls{ns} Theory considers stability a crucial property.
\textcite[269-270]{mannaert_normalized_2016} describe that stable software is not
excessively sensitive to small changes. The number of changes required for new versions of
the system is not dependent on the size of that system. Conversely, instabilities occur
when the total number of changes relies on the size of the system. The bigger the size of
the system, the more changes are required to implement the requirement. Mannaert et al.
\textcite[271]{mannaert_normalized_2016} refer to these instabilities' \gls{ce}
\subsection{Evolvability} \label{sec_on_evolvability}

In section \ref{subsec_on_stability}, it was highlighted that stability in a system is
achieved when it exhibits low sensitivity to minor changes, regardless of the system's
size. On the other hand, evolvability refers to a system's ability to adapt and adjust to
changing requirements continuously. Time is a critical factor here, as systems are often
easier to maintain during the initial stages. Evolvability, however, relates to the
system's ability to evolve and adapt independently of time.
\subsection{Modularity} \label{subsec_modularity}

The original material of \textcite[82]{robert_c_martin_clean_2018} describes a module as a
piece of code encapsulated in a source file with a cohesive set of functions and data
structures. According to \textcite[22]{mannaert_normalized_2016}, modularity is a
hierarchical or recursive concept that should exhibit high cohesion. While both design
approaches agree on the cohesiveness of a module's internal parts, there seems to be a
slight difference in granularity in their definitions.
\chapter{Component Cohesion Principles} \label{appendix_cohesion_principles}

\begin{table}[H]
    \small
    \begin{tabular}{ P{0.25\linewidth} | p{0.69\linewidth}} 
        \hline
        \textbf{Name} & \textbf{Description} \\ \hline
        \acrlong{rep} & \acrshort{rep} is a concept related to software development that
        refers to the balance between reusing existing software components and releasing
        new ones to ensure the efficient use of resources and time
        \parencite[104]{robert_c_martin_clean_2018}.\\ \midrule 
        
        \acrlong{ccp} & In the context of Clean Architecture, the \acrshort{ccp} states
        that classes or components that change together should be packaged together. In
        other words, if a group of classes is likely to be affected by the same kind of
        change, they should be grouped into the same package or module. This approach
        enhances the maintainability and modularity of the software
        \parencite[105]{robert_c_martin_clean_2018}.\\ \midrule 
        
        \acrlong{crp} & \acrshort{crp} states that classes or components that are reused
        together should be packaged together. It means that if a group of classes tends to
        be used together or has a high level of cohesion, they should be grouped into the
        same package or module. This approach aims to make it easier for developers to
        reuse components and understand their relationships
        \parencite[107]{robert_c_martin_clean_2018}.\\

        \bottomrule
    \end{tabular}
    \caption{The component Cohesion Principles}
    \label{appendix_tab_cohesion_principles}
\end{table}

Cohesion facilitates the reduction of complexity and interdependence among the components
of a system, thereby contributing to a more efficient, maintainable, and reliable system.
By organizing components around a shared purpose or function or by standardizing their
interfaces, data structures, and protocols, cohesion can offer the following benefits:

\begin{itemize}
    \item \textbf{Reduce redundancy and duplication of effort}: \\
    Cohesion ensures that components are arranged around a common purpose or function,
    reducing duplicates or redundant code. This simplifies system comprehension,
    maintenance, and modification.
    \item \textbf{Promoting code reuse:}\\
    Cohesion facilitates code reuse by making it easier to extract and reuse components
    designed for specific functions. This saves time and effort during development and
    enhances overall system quality.
    \item \textbf{Enhance maintainability:}\\
    Cohesion decreases the complexity and interdependence of system components, making it
    easier to identify and rectify bugs or errors in the code. This improves system
    maintainability and reduces the risk of introducing new errors during maintenance.
    \item \textbf{Increase scalability:}\\
    Cohesion improves a system's scalability by enabling it to be extended or modified
    effortlessly to accommodate changing requirements or conditions. By designing
    well-organized and well-defined components, developers can easily add or modify
    functionality as needed without disrupting the rest of the system.  
\end{itemize}
\subsection{Low Coupling} \label{subsec_on_coupling}

Coupling is an essential concept in software engineering related to the degree of
interdependence among software modules and components. High coupling between modules
indicates the strength of their relationship, whereby a high level of coupling implies a
significant degree of interdependence. Conversely, low coupling signifies a weaker
relationship between modules, where modifications in one module are less likely to impact
others. Although not always possible, the level of coupling between the various modules of
the system should be kept to a bare minimum. Both \textcite[23]{mannaert_normalized_2016}
and \textcite[130]{robert_c_martin_clean_2018} agree with the idea that modules should be
coupled as loosely as possible
\subsection{Expansion} \label{subsec_expansion}

According to \textcite[403]{mannaert_normalized_2016}, creating and maintaining a stable
and evolvable system is a particularly challenging, repetitive, and meticulous engineering
job. Developers must have a sound knowledge of NS while implementing new requirements in
an always consistent manner. \textcite[403]{mannaert_normalized_2016} propose automating
software structure instantiation by using code generation for recurring tasks. This
process is referred to as expansion.


\subsection{The Theoretical Framework} \label{subsec_ns_desing_theorems}

\gls{ns} consists of a theoretical framework describing a set of design principles. These
principles are the basis for achieving the concepts of stability, evolvability, and
modularity. \gls{ns} provides a rigorous and mathematical foundation for these theorems
and they offer guidelines for designing and developing software systems. In the following
sections, we will focus on the principles of \gls{ns} very briefly as they have been
extensively described in various scientific papers.

We know from Chapter \ref{sec_artifact_requirements} that the design artifacts, as a part
of this research, are implemented based on the \gls{ca} principles. Therefore, contrary to
Chapter \ref{sec_ca_theory}, there will be no references to the manifestations of the NS
design theorems in the design.

\subsubsection{Separation of Concerns} \label{subsubsec_soc}

\gls{soc} as a principle has first been mentioned by
\citeauthor{dijkstra_selected_1982}\footnote{\url{https://en.wikipedia.org/wiki/Separation_of_concerns}}
as the crucial principle to design modular software architecture
\parencite[]{dijkstra_selected_1982}. \gls{soc} promotes the idea that a program should be
divided into distinct sections, each addressing a particular concern or aspect of a design
problem. This allows for a more organized and maintainable source code. When implemented
correctly, a change to one concern does not affect the others. \gls{soc} should be applied
at the level of individual modules, rather than the level of an entire program.

\gls{soc} has been adopted as one of the design theorems of \gls{ns}, although it has a
stricter definition of this principle\parencite{mannaert_normalized_2016}.

\mycolorbox{A processing function can only contain a single task to achieve stability.}{Theorem I}
\subsubsection{Data version Transparency}

\gls{dvt} is the act of encapsulation of data entities for specific tasks at hand. This
results in the fact that data structures can have multiple versions often mentioned as
Data Transfer Objects in modern software engineering projects. In other words, it should
be possible to update the data entity without affecting the processing functions. This
leads to the following description of the theorem \parencite[280]{mannaert_normalized_2016}.

\mycolorbox{A data structure that is passed through the interface of a processing function
needs to exhibit version transparency to achieve stability.}{Theorem II}

\gls{dvt} is widely used in various technological applications. practically every web
service currently known supports some type of versioning. In restful APIs, for example, it
is common practice to support versioning over the URI. It is considered a best practice to
encapsulate breaking changes in a new version of the endpoint/service so that the
consumers are not (directly) affected by the change. In modern Object Oriented languages,
gls{dtv} is also supported by the ability to determine the scope of visibility of the
modifiers of the various programming constructs like fields, properties, interfaces, and
classes, also known as information hiding
\parencites{parnas_criteria_1972}[278]{mannaert_normalized_2016}.
\subsubsection{Action version Transparency}
\gls{avt} is the property of a system to modify existing processing functions without
affecting the existing ones. It should be possible to upgrade a function without affecting
the callers of those functions. This description leads to the following theorem
\parencite[282]{mannaert_normalized_2016}.

\mycolorbox{A processing function that is called by another processing function, needs to exhibit version transparency to achieve stability.}{Theorem III}

Most of the modern technology environments support some form of \gls{avt}. Polymorphism is
a widely used technique to support this theorem. Specifically, parametric
polymorphism \footnote{\url{https://en.wikipedia.org/wiki/Parametric_polymorphism}} allows
for a processing function to have multiple input parameters. There are also quite some
design patterns supporting this theorem. Some random examples are the state pattern
\footnote{\url{https://en.wikipedia.org/wiki/State_pattern}}, facade pattern
\footnote{\url{https://en.wikipedia.org/wiki/Facade_pattern}} and observer pattern
\footnote{\url{https://en.wikipedia.org/wiki/Observer_pattern}}.
\subsubsection{Separation of State}

\gls{sos} is a theorem that is based on the idea that processing functions should not
contain any state information but instead should rely on external data structures to store
state information. By separating state information from processing functions, Normalized
Systems can achieve a higher level of flexibility and adaptability. External data
structures can be updated or replaced without affecting the processing functions
themselves, which significantly reduces the change of unwanted ripple effects. This theorem is
described as followed: \parencite[258]{mannaert_normalized_2016}.

\mycolorbox{Calling a processing function within another processing function, needs to exhibit state keeping to achieve stability.}{Theorem IV}
\subsubsection{Separation of Concerns} \label{subsubsec_soc}

\gls{soc} as a principle has first been mentioned by
\citeauthor{dijkstra_selected_1982}\footnote{\url{https://en.wikipedia.org/wiki/Separation_of_concerns}}
as the crucial principle to design modular software architecture
\parencite[]{dijkstra_selected_1982}. \gls{soc} promotes the idea that a program should be
divided into distinct sections, each addressing a particular concern or aspect of a design
problem. This allows for a more organized and maintainable source code. When implemented
correctly, a change to one concern does not affect the others. \gls{soc} should be applied
at the level of individual modules, rather than the level of an entire program.

\gls{soc} has been adopted as one of the design theorems of \gls{ns}, although it has a
stricter definition of this principle\parencite{mannaert_normalized_2016}.

\mycolorbox{A processing function can only contain a single task to achieve stability.}{Theorem I}
\subsubsection{Data version Transparency}

\gls{dvt} is the act of encapsulation of data entities for specific tasks at hand. This
results in the fact that data structures can have multiple versions often mentioned as
Data Transfer Objects in modern software engineering projects. In other words, it should
be possible to update the data entity without affecting the processing functions. This
leads to the following description of the theorem \parencite[280]{mannaert_normalized_2016}.

\mycolorbox{A data structure that is passed through the interface of a processing function
needs to exhibit version transparency to achieve stability.}{Theorem II}

\gls{dvt} is widely used in various technological applications. practically every web
service currently known supports some type of versioning. In restful APIs, for example, it
is common practice to support versioning over the URI. It is considered a best practice to
encapsulate breaking changes in a new version of the endpoint/service so that the
consumers are not (directly) affected by the change. In modern Object Oriented languages,
gls{dtv} is also supported by the ability to determine the scope of visibility of the
modifiers of the various programming constructs like fields, properties, interfaces, and
classes, also known as information hiding
\parencites{parnas_criteria_1972}[278]{mannaert_normalized_2016}.
\subsubsection{Action version Transparency}
\gls{avt} is the property of a system to modify existing processing functions without
affecting the existing ones. It should be possible to upgrade a function without affecting
the callers of those functions. This description leads to the following theorem
\parencite[282]{mannaert_normalized_2016}.

\mycolorbox{A processing function that is called by another processing function, needs to exhibit version transparency to achieve stability.}{Theorem III}

Most of the modern technology environments support some form of \gls{avt}. Polymorphism is
a widely used technique to support this theorem. Specifically, parametric
polymorphism \footnote{\url{https://en.wikipedia.org/wiki/Parametric_polymorphism}} allows
for a processing function to have multiple input parameters. There are also quite some
design patterns supporting this theorem. Some random examples are the state pattern
\footnote{\url{https://en.wikipedia.org/wiki/State_pattern}}, facade pattern
\footnote{\url{https://en.wikipedia.org/wiki/Facade_pattern}} and observer pattern
\footnote{\url{https://en.wikipedia.org/wiki/Observer_pattern}}.
\subsubsection{Separation of State}

\gls{sos} is a theorem that is based on the idea that processing functions should not
contain any state information but instead should rely on external data structures to store
state information. By separating state information from processing functions, Normalized
Systems can achieve a higher level of flexibility and adaptability. External data
structures can be updated or replaced without affecting the processing functions
themselves, which significantly reduces the change of unwanted ripple effects. This theorem is
described as followed: \parencite[258]{mannaert_normalized_2016}.

\mycolorbox{Calling a processing function within another processing function, needs to exhibit state keeping to achieve stability.}{Theorem IV}
\subsection{The Design Elements} \label{subsec_design_elements} \todo{add cites to CA book}

\textcite{robert_c_martin_clean_2018} proposes the following elements to achieve
the goal of \enquote{Clean Architecture}.

\begin{table}[H]
    \begin{tabular}{ p{0.17\linewidth} p{0.74\linewidth}}
        \hline
        \textbf{Element} & \textbf{Description} \\ 
        \hline
        Entity & Entities are the core business objects, representing the domain's
        fundamental data.\\ \midrule

        Interactor & Interactors encapsulate business logic and represent specific actions
        that the system can perform. \\ \midrule

        RequestModel & RequestModels are used to represent the input data required by a specific
        interactor.\\ \midrule

        ViewModel & ViewModels are responsible for managing the data and behaviour of the
        user interface. \\ \midrule

        Controller & Controllers are responsible for handling requests from the user
        interface and routing them to the appropriate Interactor.\\ \midrule

        Presenter & Presenters are responsible for formatting and the data for the user
        interface.\\ \midrule

        Gateway & A Gateway provides an abstraction layer between the application and its
        external dependencies, such as databases, web services, or other external
        systems.\\ \midrule

        Boundary & Boundaries are used to separate the the different layers of the component.\\

        \bottomrule
    \end{tabular}
    \caption{The Elements proposed by Clean Architecture}
    \label{ca_element}
\end{table}