\section{Research method} \label{sec:research_method}

This research is a Design Science Method and relies on the Engineering Cycles as described
by Wieringa. The engineering cycle provides a structured approach to develop the
required artifacts to analyze the design problem \parencite{wieringa_design_2014}.

\begin{figure}[H]
    \centering
    \includegraphics[width=1\textwidth]{Figures/engineering_cycle.pdf}
    \caption[Engineering cycle]{Wieringa's engineering cycle}
    \label{fig:engineering_cycle}
\end{figure}

In the context of this research, the artifacts described in chapters
\ref{sec:generator_artifact} and \ref{sec:generated_artifact} are considered to be
information systems. \citeauthor{hevner_design_nodate} proposed a framework for research
in information systems by introducing the interacting relevance and rigor cycles.

Figure \ref{fig:dsr} depicts a specialized overview of Hevners Design Science Framework.
The rigor cycle is composed of the theories and knowledge from Normalized Systems
and Clean architecture. This is supplemented by the rigorous knowledge of modularity,
evolvability, and stability of software systems. The Relevance cycle represents the
business needs of the stakeholders. The business needs are described as research
objectives, research questions, and research requirements.

\begin{figure}[H]
    \centering
    \includegraphics[width=1\textwidth]{Figures/rigor_relevance_cycle.pdf}
    \caption[DSF]{The Design Science Framework for IS Research}
    \label{fig:dsr}
\end{figure}