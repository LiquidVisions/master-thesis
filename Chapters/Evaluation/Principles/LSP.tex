\subsection{Converging the Liskov Substitution Principle}

\begin{table}[H]
    \begin{tabular}{ l | c | p{0.80\linewidth}}
        \toprule
        \gls{soc} & \converges & Adhering to \gls{lsp} in the software design leads to a more
        modular design and separation of specific concerns. Therefore we state that \gls{lsp}
        converges with \gls{soc}. \gls{lsp} states that objects of a derived class should be
        able to replace objects of the base class without affecting the correctness of the
        program. This can only be achieved by a strict separation of concerns in combination
        with Action version Transparent implementations of the signature.
        \\
        \midrule
        \gls{dvt} & \noconvergence & \gls{lsp} has a limited alignment with the \gls{dvt}
        theorem. While \gls{lsp} focuses on the substitutability of objects in class
        hierarchies, \gls{dvt} aims to handle changes in data structures without impacting
        the system. By following \gls{lsp}, developers can ensure that derived classes can
        be substituted for their base classes, which may help reduce the impact of data
        structure changes on the system. However, \gls{lsp} does not explicitly address
        data versioning and thus does not guarantee full convergence with \gls{dvt}. \\
        \midrule
        \gls{avt} & \supports & The \gls{lsp} supports the \gls{avt} theorem. Both principles
        emphasize the importance of allowing the extensibility of the system, without
        negatively impacting the desired requirements. By adhering to \gls{lsp}, developers
        can create class hierarchies that can be easily extended to accommodate new actions or
        changes in existing ones, which may contribute to achieving \gls{avt}. However,
        adhering to \gls{lsp} alone may not guarantee full convergence with \gls{avt}. 
        \\
        \midrule
        \gls{sos} & \noconvergence & By designing class hierarchies according to \gls{lsp},
        developers can create components that are less prone to side effects caused by shared
        states. However, the alignment between \gls{lsp} and \gls{sos} is very weak, and
        adhering to \gls{lsp} alone may not guarantee full separation of states. 
        \\
        \bottomrule
    \end{tabular}
    \caption{Converge \gls{lsp} with \gls{ns}}
    \label{tab_convergence_lsp}
\end{table}