\subsubsection{Data version Transparency}

\gls{dvt} is the act of encapsulation of data entities for specific tasks at hand. This
results in the fact that data structures can have multiple versions often mentioned as
Data Transfer Objects in modern software engineering projects. In other words, it should
be possible to update the data entity without affecting the processing functions. This
leads to the following description of the theorem \parencite[280]{mannaert_normalized_2016}.

\mycolorbox{A data structure that is passed through the interface of a processing function
needs to exhibit version transparency to achieve stability.}{Theorem II}

\gls{dvt} is widely used in various technological applications. practically every web
service currently known supports some type of versioning. In restful APIs, for example, it
is common practice to support versioning over the URI. It is considered a best practice to
encapsulate breaking changes in a new version of the endpoint/service so that the
consumers are not (directly) affected by the change. In modern Object Oriented languages,
gls{dtv} is also supported by the ability to determine the scope of visibility of the
modifiers of the various programming constructs like fields, properties, interfaces, and
classes, also known as information hiding
\parencites{parnas_criteria_1972}[278]{mannaert_normalized_2016}.