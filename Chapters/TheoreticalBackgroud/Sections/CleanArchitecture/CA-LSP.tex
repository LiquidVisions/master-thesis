\subsubsection{The Liskov Substitution Principle} \label{subsubsec:lsp}


The \gls{lsp} is a fundamental concept in object-oriented programming that deals with the
behavior of subtype objects when used in place of their base type. It is named after
Barbara Liskov. She first introduced the principle in a paper she co-authored with
Jeannette Wing in 1987. Barbara Liskov wrote the following statement as a way of defining
subtypes \parencite{robert_c_martin_clean_2018}.

\mycolorbox{If for each	object o1 of type S there is an	object o2 of type T such that for
all programs P defined in terms of T, the behavior of P	is unchanged when o1 is
substituted for	o2 then S is a subtype of T.1}{\acrlong{lsp}}

The principle is based on the idea that a subtype should be semantically substitutable for
its base type. This means that the subtype should behave in a way that is consistent with
the expectations of the base type and should not introduce any new behaviors or violate
any of the constraints imposed by the base type.

The practical implications of \gls{lsp} are many. In software design, it means that we
should strive to create subtypes that are as similar as possible to their base types in
terms of their behavior and the constraints they impose. In testing, it means that we
should test subtypes to ensure that they behave correctly when used in place of their base
types.

Considered the implementation of the \citecode{koks_ilogger_2023} interface. This is
currently implemented in \citecode{koks_logger_2023} using the NLog framework.
Implementations that use logging only make use of the \code{koks_ilogger_2023} by making
proper use of the \acrlong{dip}. In the future, it might be needed to replace the NLog
framework with an arbitrary logging framework. This would be rather simple by just
replacing the \code{koks_logger_2023} implementation.

The \citecode{koks_icreategateway_2023} in Listing \ref{SipICreateGateway} is another
example. The artifact has two implementations of this interface. The data of the entities
are currently stored in the database, but harvest data is serialized to XML using the same
\code{koks_icreategateway_2023} interface. With this design decision, it is very easy to adapt
to a different type of storage mechanism if future requirement demand such a change.

\lstinputlisting[
    caption={The \citetitle{koks_icreategateway_2023}},
    label={SipICreateGateway} ]
    {Snippets/ICreateGateway.cs}

\lstinputlisting[
    caption={The \citetitle{koks_genericrepository_2023} and \citetitle{koks_harvestrepository_2023}},
    label={SipICreateGatewayImplementations} ]
    {Snippets/ICreateGatewayImplementations.cs}