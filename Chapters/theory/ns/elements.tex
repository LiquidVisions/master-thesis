\subsection{Normalized Elements} \label{subsec_ns_elements} 

In the context of the \gls{ns} Theory approach, the goal is to design evolvable software
independent of the underlying technology. Nevertheless, a particular technology must be
chosen when implementing the software and its components. For Object Oriented Programming
Languages like Java, the following Normalized Elements have been proposed
\parencite[363-398]{mannaert_normalized_2016}. It is essential to recognize that different
programming languages may necessitate alternative constructs
\parencite[364]{mannaert_normalized_2016}. The table describing each element uses the
definition from \textcite{mannaert_towards_2012}.

\begin{table}[H]
    \begin{tabular}{ p{0.15\linewidth} p{0.75\linewidth}}
        \hline
        \textbf{Element} & \textbf{Description} \\ 
        \hline
        Data & Based on \gls{dvt}, data elements have get- and set-methods for wide-sense
        data version transparency, or marshal -and parse- methods for strict-sense data
        version transparency. Supporting tasks can be added in a way which is consistent
        with the principles \gls{soc} and \gls{dvt}.\\
        \midrule

        Task & Based on \gls{soc}, the core action entity can only contain a single
        functional task, and not multiple tasks. Based on \gls{avt}, arguments and
        parameters must be encapsulated data entities. Based on \gls{soc} and \gls{sos},
        workflows need to be separated from action entities, and will therefore be
        encapsulated in a workflow element. Based on \gls{avt}, tasks need to be
        encapsulated in such a way that a separate action entity wraps the action entities
        representing task versions. Supporting tasks can be added in a way which is
        consistent with \gls{soc} and \gls{avt}.\\
        \midrule

        Workflow & Based on \gls{soc}, workflow elements cannot contain other functional
        tasks, as they are generally considered a separate change driver, often
        implemented in an external technology. Based on \gls{sos}, workflow elements must
        be stateful. This state is required for every instance of use of the action
        element, and therefore needs to be part of, or linked to, the instance of the data
        element that serves as argument.\\ \midrule

        Connector & Based on Theorem \gls{sos}, connector elements must ensure that
        external systems can interact with data elements, but that they cannot call an
        action element in a stateless way. Supporting tasks can be added in a way which is
        consistent with \gls{soc} and \gls{avt}.\\ \midrule

        Trigger & Based on \gls{soc}, trigger elements need to control the separated —both
        error and non- errorstates, and check whether an action element has to be
        triggered. Supporting tasks can be added in a way which is consistent with
        \gls{soc} and \gls{avt}.\\

        \bottomrule
    \end{tabular}
    \caption{The Elements proposed by Normalized Systems Theory}
    \label{ns_element}
\end{table}