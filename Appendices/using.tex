\chapter{Using Pantha Rhei} \label{appendix_run_flux}

Pantha Rhei is used by executing the \emph{flux} command with the parameters as described
in Table \ref{appendix_tab_commandline_parameters}

\lstinputlisting[
    caption={Example command executing Pantha Rhei},
    label={appendix_list_flux}]
    {Snippets/flux.txt}

\begin{table}[H]
    \begin{tabular}{ l | p{0.78\linewidth}}
        \toprule
        --root & A mandatory parameter that should contain the full path to the output
        directory. \\
        --db & A mandatory parameter that contains the connection string to the database. \\
        --app & A mandatory parameter indicating the unique identifier of the application that should be generated. \\
        --mode & An optional parameter that determines if a handler should be executed.
        \emph{Default} is the default fallback mode (see \ref{appendix_tab_generation_modes}). \\
        --reseed & An optional parameter that bypasses the expanding process. The model will
        be thoroughly cleaned and reseeded based on the entities of the expander
        artifact. This enables to a certain extent the meta-circularity and enables the
        expander artifact to generate itself. \\
        \bottomrule
    \end{tabular}
    \caption{The \emph{flux} command line parameters}
    \label{appendix_tab_commandline_parameters}
\end{table}



The following RunModes are available to isolate execution tasks.

\begin{table}[H]
    \begin{tabular}{ l | p{0.78\linewidth}}
        \toprule
Default & This is the default generation mode that executes all configured handlers of the
CleanArchitectureExpander. This will also install the required Visual Studio templates
which are needed for scaffolding the Solution and C\# Project files. Furthermore, it also
executes the Harvest and Rejuvenation handlers. This mode will clean up the entire output
folder prior after the Harvesting process is finished prior to the execution of the
handlers. \\
        
Extend & This mode will skip the installation of the Visual Studio templates and the
project scaffolding. It will not clean up the output folder but will overwrite any files
handled. This mode is often less time-consuming and can be used in scenarios to quickly
check the result of a part of the generation process. \\

Deploy & An optional mode that allows for expander handlers to run deployments in
isolation. For example, when a developer wants to deploy the output to an Azure App
Service. \\
       
Migrate & An optional mode that allows for expander handlers to run migrations in
isolation. For example, this  currently updates the database schema by running the Entity
Framework Commandline Interface (see
\url{https://learn.microsoft.com/en-us/ef/core/cli/dotnet}).\\
        \bottomrule
    \end{tabular}
    \caption{The available \emph{Generation modes}}
    \label{appendix_tab_generation_modes}
\end{table}