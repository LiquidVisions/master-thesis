\subsection{Component Architecture Requirements} \label{component_requirements}
 
The following requirements are applied to the Component Architecture of both the Generator
Artifact and the Generated Artifact.

\requirement{The Component Architecture}{table_component_requirements}{
    \addEvalRow{The solution is organized into separate Visual Studio projects for the Domain,
    Application, Infrastructure, and Presentation layers of the component. A detailed
    description of these layers can be found in Section \fullref{subsec_layers}}

    \addEvalRow{The Visual Studio projects representing the component layers comply with the naming
    conventions outlined in the appendix \fullref{appendix_component_naming_convention}}

    \addEvalRow{The dependencies between the component layers must follow an inward direction towards
    the higher-level components as illustrated in Figure \ref{fig_modulair_components}
    schematically, and cannot skip layers.}
}

\requirement{Technology Expander}{table_technology}{
    \addEvalRow{The Domain and Application layers have no dependencies on any infrastructure technologies, like web- or database
    technologies. }

    \addEvalRow{The Presentation Layer relies on various infrastructure technologies for
    facilitating end-user interaction. Examples of such technologies include Command
    Line Interfaces (CLIs), RESTful APIs, and web-based solutions. Each dependency is
    isolated and managed in separate Visual Studio Projects to ensure the stability and
    evolvability of the system.}

    \addEvalRow{The Infrastructure Layer may rely on other infrastructure components, such as
    databases or filesystems. Each infrastructure dependency is isolated and managed in
    separate Visual Studio Projects to promote stability and evolvability.}

    \addEvalRow{All Component Layers utilize the C\# programming language, explicitly targeting the
    .NET 7.0 framework.}

    \addEvalRow{Reusing existing functionality or technology (packages) is permitted only when
    adhering to the \gls{lsp} and utilizing the open-source package manager,
    \gls{nuget}.}
}