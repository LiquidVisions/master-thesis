\chapter{Introduction} \label{chap_introduction}

After my bachelor's degree in 2009, I started to work as a junior software engineer. I was
confident I was willing to accept any technical challenge as my experience up until that
point led me to believe that creating some software was not that difficult. I could not
have been more wrong. I quickly discovered it was a real challenge to apply new
requirements to existing pieces of (legacy) software or explain my craftings to the more
mature engineers. The craftsmanship of software engineering was enormously challenging to
me.

Determined to overcome the difficulties, I started reading and investigating and
immediately recognized the Law of Increasing Complexity of
\textcite{lehman_programs_1980}, where he explained the balance between the forces driving
new requirements and those that slow down progress. Other pioneers in software have
recognized these challenges in engineering also.

\textcite{d_mcilroy_nato_1968} proposed a vision where the systematic reuse of software
building blocks should lead to more reuse. \textcite{d_mcilroy_nato_1968} quoted,
\enquote{The real hero of programming is the one who writes negative code,} i.e., when a
change in a program source makes the number of lines of code decrease ('negative' code),
while its overall quality, readability or speed improves
\parencite{wikipedia_douglas_2023}. Perhaps very early concepts of modular software
constructs?

\textcite{dijkstra_letters_1968} argued against using unstructured control flow in
programming and advocated for using structured programming constructs to improve the
clarity and maintainability of the source code. In addition,
\Citeauthor{dijkstra_letters_1968} advocated structured programming techniques that
improved the modularity and evolvability of software artifacts.

\textcite{parnas_criteria_1972} continued with the principle of information hiding.
\Citeauthor{parnas_criteria_1972} stated that design decisions used multiple times by a
software artifact should be modularized to reduce complexity. 

Over the years, I got introduced to various software design principles and philosophies
like \gls{ca}, increasing my knowledge and craftsmanship. My career moved more toward the
arts of architecture and product management, and I have always retained my passion for
software engineering.

My obsession got re-ignited during my Master's degree introduction days at the Priory of
Corsendock. Jan Verelst introduced me to \gls{ns}, and I was intrigued by software
stability and evolvability. It was fascinating to learn that there is now empirical
scientific evidence for a quest I have been on for almost a decade. 

\gls{ns} Had to be the topic of my research. I was curious to compare what I knew
(\gls{ca}) with what science offered (\gls{ns}). In early investigations, I found
overlapping characteristics. Nevertheless, there were also a couple of differences. Could
these design approaches be used in conjunction with each other?

Java SE has primarily been used for case studies in order to develop the Normalized
Systems Theory \parencite{oorts_building_2014, de_bruyn_enabling_2018}. Although
sufficient in Java, I was pleased to read that both software design approaches have
formulated modular structures independent of any programming technology
\parencite{mannaert_normalized_2009,robert_c_martin_clean_2018}. So I could use my
favorite programming language C\#, to create a software artifact that supported my
research. 

Based on early investigations, I instinctively found that many applications of \gls{ca}
are a specialization of the \gls{ns} Theorems. Consequently, I hypothesized that \gls{ca}
and \gls{ns} could be used to achieve a modular, evolvable, and stable software artifact.

\begin{figure}[H]
    \centering
    \includegraphics[width=0.8\textwidth]{figures/hypothesis.pdf}
    \caption[The hypothesis]{The hypothesis}
    \label{fig_hypothesis}
\end{figure}

Since this research is investigating the convergence of gls{ca} and \gls{ns}, it is
relevant to introduce them and discuss the concepts mentioned in the following sections.

\section{Research Method} \label{sec_research_method}

This research is a design science method and relies on the engineering cycles described
by \textcite{wieringa_design_2014}. The engineering cycle provides a structured approach
to developing the required artifacts to analyze the design problem.

\begin{figure}[H]
    \centering
    \includegraphics[width=1\textwidth]{figures/engineering_cycle.pdf}
    \caption[Engineering cycle]{The Engineering Cycle of \textcite{wieringa_design_2014}}
    \label{fig_engineering_cycle}
\end{figure}

In this research, a significant component has been the development of a tangible software
artifact, providing a real-world illustration of the convergence between Clean
Architecture and Normalized Systems. \textcite{hevner_design_2004} proposed a framework
for research in information systems by introducing the interacting relevance and rigor
cycles.

Figure \ref{fig_dsr} depicts a specialization of the design Science Framework of
\textcite{hevner_design_2004}. The rigor cycle comprises the theories and knowledge from
\gls{ns} and \gls{ca}, supplemented by the rigorous knowledge of modularity, evolvability,
and stability of software systems. The relevance cycle represents the business needs of
the stakeholders. The research requirements are described as research objectives.

\begin{figure}[H]
    \centering
    \includegraphics[width=1\textwidth]{figures/rigor_relevance_cycle.pdf}
    \caption[Design Science Framework for IS Research]{The design Science Framework for IS Research \parencite{hevner_design_2004}}
    \label{fig_dsr}
\end{figure}
\section{Research Objectives} \label{sec_research_objectives}

In this Design Science Research, we will shift the focus from Research Questions to
Research Objects. The primary goal of this research is to determine the degree of
convergence of \gls{ca} with the \gls{ns} Theory. In order to achieve this goal, the
research is divided into the following objectives:

\begin{enumerate}
    \item \textbf{Literature Review} \\
    Conduct a literature review of \gls{ca} and \gls{ns}, focusing on their
    fundamental elements, principles, and real-world case studies. This review will
    provide a solid foundation for understanding the underlying concepts and their
    practical implications.
    
    \item \textbf{Architectural Desing} \\
    Create an Architectural Design fully and solely based on \gls{ca}. Implement the
    findings of the Literature Review in the Design. This design will be the basis for
    the Artifact Development.

    \item \textbf{Artifact Development} \\
    Construct two artifacts that facilitate the research of the convergence
    between \gls{ca} and \gls{ns} Theories.
    \begin{enumerate}[label*={\arabic*.}]
        
        \item \textbf{The Code Generator and Clean Architecture Expander} \\
        Inspired by \gls{ns}, create a code generator based on the \gls{ca} design. The
        generator will enable the rapid creation of software systems adhering to the
        principles and design of \gls{ca} and allows for efficient examination of their
        characteristics. The Clean Architecture expander expands a RESTful API based on
        the same \gls{ca} design as the code generator. 
        
        \item \textbf{Expanded Clean Architecture artifact} \\
        The expanded artifact will facilitate the analysis of a RESTful API implementation
        and its alignment with the \gls{ca} principles and design.
        
    \end{enumerate}
    
    \item \textbf{Convergence Analysis:} \\
    Analyze the artifacts to determine the degree of convergence between the principles
    and elements of \gls{ca} and \gls{ns} Theory. This analysis will involve the following:
    \begin{enumerate}[label*={\arabic*.}]
        
        \item An analysis per principle of \gls{ca}, compared with each of the principles
        of \gls{ns}, indicating each level of convergence per principle
        
        \item An analysis per element of \gls{ca}, compared with each of the elements of
        \gls{ns}, indicating each level of convergence per principle
    
    \end{enumerate}
\end{enumerate}
\section{Thesis Outline} \label{sec_structure}

The thesis is organized into seven main chapters, beginning with the introduction. The
introduction provides an overview of the study's research method and objectives and
includes this Section of the \nameref*{sec_structure}.

Chapter \ref{chap_theoreticalbackground} focuses on the study's theoretical background,
covering \gls{ca}, \gls{ns}, and some generic concepts. Chapter \ref{chap_requirements} is
dedicated to the requirements for software transformation and the artifacts built as part
of this study. Chapter \ref{chap_designing_artifacts} focuses on specific artifact design
Decisions, where chapter \ref{chap_evaluation} discusses the evaluation results of this
study. We conclude with the conclusion in \ref{chap_conclusions} and a personal reflection
on the journey of this research in \ref{chap_reflection}.