\chapter{Introduction} \label{chap_introduction}

Sinds the early days there have been challenges with creating and maintaining stable and
evolvable software. On the one hand, this is caused by constantly evolving requirements as
new business opportunities, technologies, methodologies, and best practices are developed
to meet the demands of modern corporate environments. On the other hand, changing software
can lead to deterioration in stability and evolvability, which can negatively impact the
quality of these systems. \textcite{lehman_programs_1980} has described this as one of his
laws of software evolution: The balance between the forces driving new requirements and
those that slow down progress. These challenges have been recognized by the following
pioneers in software engineering. 

\textcite{d_mcilroy_nato_1968} proposed a vision where the systematic reuse of software
building blocks should lead to more reuse. \textcite{d_mcilroy_nato_1968} quoted
\enquote{The real hero of programming is the one who writes negative code}, i.e. when a
change in a program source makes the number of lines of code decrease ('negative' code),
while its overall quality, readability or speed improves
\footnote{\url{https://en.wikipedia.org/wiki/Douglas_McIlroy}}. Perhaps very early concepts
of modular software constructs?

\textcite{dijkstra_letters_1968} argued against using unstructured control flow in
programming and advocated for using structured programming constructs to improve the
clarity and maintainability of the source code. He advocated structured programming
techniques that improved the modularity and evolvability of software artifacts.

\textcite{parnas_criteria_1972} continued with the principle of information hiding. He
stated that design decisions used multiple times by a software artifact should be
modularized to reduce complexity. 

Various programming paradigms, including procedural, object-oriented, and functional
programming, have emerged to enhance software programming capabilities that contribute to
stability and evolvability \parencite{mannaert_normalized_2016}. These paradigms have
impacted modern programming languages, such as Java and C\#, enabling the development of
more modular and evolvable software architectures.

Design principles, patterns, and theorems are, on top of all, additional measures to
enhance the modularity, stability, and evolvability of software artifacts. As a junior
software engineer, I was always intrigued by the concepts of quality and maintainable code
and quickly got introduced to the \gls{solid} principles. And later on with the complete
design approach derived from \gls{ca}. Starting my Master's degree, I got an
inspiring introduction from Jan Verelst and realized quality and maintainability were
essentially all about the concepts of stability and evolvability. For me, it was very
interesting that the \gls{ns} Theory is supported by empirical scientific evidence.
Although I'm not that active anymore in the field of software engineering, I immediately
knew the topic of my research. it is still a big passion of mine.

Given my experience with \gls{ca}, and what I was learning from \gls{ns} Theory, I noticed
a lot of similarities, but also some big differences. In early investigations, I found
overlapping characteristics. But it seemed there were also a couple of differences. I
wanted to know if the design approaches could be used in conjunction with each other,
perhaps bettering the result of stable and evolvable software.

Java SE has primarily been used for case studies in order to develop the Normalized
Systems Theory \parencite{oorts_building_2014, de_bruyn_enabling_2018}. Although
sufficient in Java, I was pleased to read that both software design approaches have
formulated their modular structures independent of any programming technology
\parencite{mannaert_normalized_2009,robert_c_martin_clean_2018}. So I was free to use my
favorite programming language C\# to create a software artifact that supported my
research, igniting my passion for programming again. 

Based on my early investigations of both design approaches I hypothesized that they can be
used in conjunction with each other. Consequently, an artifact that is designed based on
the principles of Clean Architecture will lead to a highly modular, stable, and evolvable
C\# artifact that does not contradict a design based on the Normalized Systems theorems.

\begin{figure}[H]
    \centering
    \includegraphics[width=0.8\textwidth]{figures/hypothesis.pdf}
    \caption[The hypothesis]{The hypothesis}
    \label{fig_hypothesis}
\end{figure}

\section{Introduction to Normalized Systems} \label{sec_inro_ns}

The \gls{ns} Theory is a scientific approach to creating software systems based on the
laws for software evolvability. Effectively, it prevents the accumulation of required
changes to implement new requirements \parencite[]{mannaert_normalized_2009}. 

\textcite[]{mannaert_normalized_2009} have formulated the Theories of \gls{ns} as a set of
principles accompanied by structures (elements) that will lead to a highly decouples and
modular software system. The result is a software architecture designed to cope with
future changes.
\section{Introduction to Clean Architecture} \label{sec_into_ca}

\gls{ca} is the accumulation of more than half a century of coding, designing, and
architecting software systems by \citeauthor*[]{robert_c_martin_clean_2018}. The book aims
for a software architecture that minimizes the human resources required to build and
maintain the information system. \gls{ca} has a prescribed design of software elements
that will lead to a modular architecture with low coupling and high cohesion.
Additionally, the book refers to a set of design principles that prescribes how those
elements should be structured or interact with each other.

\section{Research Objectives} \label{sec_research_objectives}

In this Design Science Research, we will shift the focus from Research Questions to
Research Objects. The primary goal of this research is to determine the degree of
convergence of \gls{ca} with the \gls{ns} Theory. In order to achieve this goal, the
research is divided into the following objectives:

\begin{enumerate}
    \item \textbf{Literature Review} \\
    Conduct a literature review of \gls{ca} and \gls{ns}, focusing on their
    fundamental elements, principles, and real-world case studies. This review will
    provide a solid foundation for understanding the underlying concepts and their
    practical implications.
    
    \item \textbf{Architectural Desing} \\
    Create an Architectural Design fully and solely based on \gls{ca}. Implement the
    findings of the Literature Review in the Design. This design will be the basis for
    the Artifact Development.

    \item \textbf{Artifact Development} \\
    Construct two artifacts that facilitate the research of the convergence
    between \gls{ca} and \gls{ns} Theories.
    \begin{enumerate}[label*={\arabic*.}]
        
        \item \textbf{The Code Generator and Clean Architecture Expander} \\
        Inspired by \gls{ns}, create a code generator based on the \gls{ca} design. The
        generator will enable the rapid creation of software systems adhering to the
        principles and design of \gls{ca} and allows for efficient examination of their
        characteristics. The Clean Architecture expander expands a RESTful API based on
        the same \gls{ca} design as the code generator. 
        
        \item \textbf{Expanded Clean Architecture artifact} \\
        The expanded artifact will facilitate the analysis of a RESTful API implementation
        and its alignment with the \gls{ca} principles and design.
        
    \end{enumerate}
    
    \item \textbf{Convergence Analysis:} \\
    Analyze the artifacts to determine the degree of convergence between the principles
    and elements of \gls{ca} and \gls{ns} Theory. This analysis will involve the following:
    \begin{enumerate}[label*={\arabic*.}]
        
        \item An analysis per principle of \gls{ca}, compared with each of the principles
        of \gls{ns}, indicating each level of convergence per principle
        
        \item An analysis per element of \gls{ca}, compared with each of the elements of
        \gls{ns}, indicating each level of convergence per principle
    
    \end{enumerate}
\end{enumerate}
%\input{chapters/introduction/sections/researchquestion}
\section{Research Method} \label{sec_research_method}

This research is a design science method and relies on the engineering cycles described
by \textcite{wieringa_design_2014}. The engineering cycle provides a structured approach
to developing the required artifacts to analyze the design problem.

\begin{figure}[H]
    \centering
    \includegraphics[width=1\textwidth]{figures/engineering_cycle.pdf}
    \caption[Engineering cycle]{The Engineering Cycle of \textcite{wieringa_design_2014}}
    \label{fig_engineering_cycle}
\end{figure}

In this research, a significant component has been the development of a tangible software
artifact, providing a real-world illustration of the convergence between Clean
Architecture and Normalized Systems. \textcite{hevner_design_2004} proposed a framework
for research in information systems by introducing the interacting relevance and rigor
cycles.

Figure \ref{fig_dsr} depicts a specialization of the design Science Framework of
\textcite{hevner_design_2004}. The rigor cycle comprises the theories and knowledge from
\gls{ns} and \gls{ca}, supplemented by the rigorous knowledge of modularity, evolvability,
and stability of software systems. The relevance cycle represents the business needs of
the stakeholders. The research requirements are described as research objectives.

\begin{figure}[H]
    \centering
    \includegraphics[width=1\textwidth]{figures/rigor_relevance_cycle.pdf}
    \caption[Design Science Framework for IS Research]{The design Science Framework for IS Research \parencite{hevner_design_2004}}
    \label{fig_dsr}
\end{figure}
\section{Thesis Outline} \label{sec_structure}

The thesis is organized into seven main chapters, beginning with the introduction. The
introduction provides an overview of the study's research method and objectives and
includes this Section of the \nameref*{sec_structure}.

Chapter \ref{chap_theoreticalbackground} focuses on the study's theoretical background,
covering \gls{ca}, \gls{ns}, and some generic concepts. Chapter \ref{chap_requirements} is
dedicated to the requirements for software transformation and the artifacts built as part
of this study. Chapter \ref{chap_designing_artifacts} focuses on specific artifact design
Decisions, where chapter \ref{chap_evaluation} discusses the evaluation results of this
study. We conclude with the conclusion in \ref{chap_conclusions} and a personal reflection
on the journey of this research in \ref{chap_reflection}.