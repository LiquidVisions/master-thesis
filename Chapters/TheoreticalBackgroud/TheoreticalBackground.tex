\chapter{Theoretical background} \label{chap_theoreticalbackground} 

The goal of this thesis is to investigate whether \gls{ca} approach aligns with the goals
of \gls{ns}. Therefore, it is essential to have a comprehensive understanding of software
stability and the key concepts, principles, and architectures that impact software
stability.

This chapter begins by examining the concepts of software stability, evolvability, and
modularity, highlighting their significance in achieving software stability in \gls{ns}.
This is followed by a brief overview of the design theorems and proposed architecture of
\gls{ns}.

The subsequent sections of the thesis explore the fundamental principles that underlie
\gls{ca}, as well as its proposed architectural designs. Finally, the thesis
concludes by discussing which aspects of \gls{ca} align with the principles of
\gls{ns} and contribute to achieving software stability in this approach.

\section{Normalized Systems: Impacting software stability} \label{sec:ns_theory}

\gls{ns} is a software development approach that prioritizes achieving software stability
through the use of standardized, modular components and interfaces. This theory is
informed by several scientific disciplines, including systems theory, mathematics, and
computer science, as well as some other software development approaches, such as agile
development and domain-driven design.

\gls{ns} originated in the field of software engineering. However, the underlying theory
of \gls{ns} can be applied to various other domains, such as Enterprise Engineering,
Business Process Modeling, and document management. This research acknowledges the
software engineering background of \gls{ns}. It consistently refers to software and
Information Systems when referring to \enquote*{artifacts.} However, the reader should
realize that the concepts and artifacts are not restricted to software artifacts alone.
\section{Clean architecture: A design philosophy}\label{sec_ca_theory}

\gls{ca} is a software design philosophy that emphasizes the organization of code
into independent, modular layers with distinct responsibilities. This approach aims to
create more flexible, maintainable, and testable software systems by enforcing the
separation of concerns and minimizing dependencies between components. The goal of clean
architecture is to provide a solid foundation for software development, allowing
developers to build applications that can adapt to changing requirements, scale
effectively, and remain resilient against the introduction of bugs
\parencite{robert_c_martin_clean_2018}.
