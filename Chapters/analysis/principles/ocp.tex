\subsection{Open/Closed Principle}

\evaluatePrincipleTable{\gls{ocp}}{table_ocp_convergence}{ 
    
\addEvalRow{\gls{soc} & \fullConvergence & The \gls{ocp} has a strong convergence with
    the \gls{soc} principle of \gls{ns}. \gls{ocp} states that software architectures
    should be open for extension but closed for modification. When applying \gls{ocp}
    correctly, the architecture supports new requirements built as an extension, affecting
    as few existing implementations as possible. Conversely, adhering to \gls{soc} does
    not guarantee the adherence of \gls{ocp}, as \gls{soc} focuses on modularization and
    encapsulation rather than the extensibility of functionality. The same example with
    the Tasks provided in sub-section \ref{srp} is also an excellent manifestation of this
    principle.} 
    
\addEvalRow{\gls{dvt} & \noConvergence & The \gls{ocp} indirectly relates to the \gls{dvt}
principle. The convergence of both principles is weak, and no manifestations are found in
the artifacts.}
    
\addEvalRow{\gls{avt} & \fullConvergence & The \gls{ocp} has a is strong convergence with
    the \gls{avt} principle of \gls{ns}, as both principles emphasize the importance of
    allowing changes or extensions to actions without affecting existing implementations.
    \gls{ocp} is also closely related to \gls{srp}. Besides \gls{srp}, \gls{ocp} have the
    most manifestations in the Artifact, some of which are already mentioned in previous
    examples. } 
    
\addEvalRow{\gls{sos} & \noConvergence & The \gls{ocp} indirectly relates to the \gls{sos}
    principle. The convergence of both principles is weak, and no manifestations are found
    in the artifacts. } }