\subsection{The study of Stability, Evolvability, and Combinatorics} \label{subsec_on_stability}

The \gls{ns} Theory considers stability a crucial property derived from the concept
\gls{bibo}: A bounded functional change must result in a bounded amount of work,
independent of the size of the system. Instabilities occur when the total number of
changes relies on the size of the system. The bigger the size of the system, the more
changes are required to implement the new requirement.
\textcite[271]{mannaert_normalized_2016} refer to these \enquote*{instabilities} as
Combinatorial Effects. Conversely, stability is achieved when a system is free from these
so-called Combinatorial Effects. Based on the concept of stability,
\textcite{mannaert_towards_2012} require information systems to be stable with respect to
a set of anticipated changes to exhibit high evolvability.