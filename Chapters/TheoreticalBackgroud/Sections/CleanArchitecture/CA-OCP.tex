\subsubsection{The Open-Closed Principle} \label{subsubsec:ocp}

The \gls{ocp} was first mentioned by \Citeauthor{meyer_object-oriented_1997}. He described
the principle as followed \parencite[79]{meyer_object-oriented_1997}. The reference here
is for the second edition of the book, the original version is from 1988.

\mycolorbox{A module should	be open for extension but closed for modification.}{\acrlong{ocp}}

\gls{ocp} emphasizes the importance of designing systems that are open for extension but
closed for modification. This means that the behavior of implementations can be extended
without modifying its source code. The OCP promotes the use of abstraction and
polymorphism to achieve this goal. By using interfaces, and abstract classes a system can
be designed to allow for new behaviors to be added through extension, without changing the
existing code. The OCP is one of the driving forces	behind the architecture	of systems.
The goal is	to make	the	system easy	to extend without incurring a high impact of change
\parencite[94]{robert_c_martin_clean_2018}.

A relevant manifestation of \gls{ocp} are all the different implementations of expander
handlers in figure \ref{fig:handlers}. The availability of the
\code{koks_iexpanderhandlerinteractor_2023} interface makes it possible to add more
functionality to the CleanArchtictureExpander without modifying any existing
implementation. New handlers are added by extension, and when implemented correctly, the
handler is automatically executed in the desired order and the required conditions.

\lstinputlisting[
    caption={The \citetitle{koks_iexpanderhandlerinteractor_2023}},
    label={SipIExpanderHandlerInteractor} ]
    {Snippets/IExpanderHandlerInteractor.cs}

\lstinputlisting[
    caption={The \citetitle{koks_iexecutioninteractor_2023}},
    label={SipIExecutionInteractor} ]
    {Snippets/IExecutionInteractor.cs}

The fact that \citecode{koks_iexpanderhandlerinteractor_2023} derives from
\citecode{koks_iexecutioninteractor_2023} is another manifestation of \gls{ocp}. This
design decision allows for object types that need to be treated as executables by the
\code{koks_codegeneratorinteractor_2023}. Examples are
\citecode{koks_regionharvesterinteractor_2023},
\citecode{koks_regionrejuvenatorinteractor_2023},
\citecode{koks_preprocessorinteractor_2023} and
\citecode{koks_postprocessorinteractor_2023}. 

Listing \ref{SipCodeGeneratorInteractor} shows the
\code{koks_codegeneratorinteractor_2023} that cohesively executes all of the
\code{koks_iexecutioninteractor_2023} in order. The software engineer only has to focus on
implementing the specific type of \code{koks_iexecutioninteractor_2023} without having to
affect the implementation. This is by definition an example of \enquote{open for
extension} and \enquote{closed for modifications}.