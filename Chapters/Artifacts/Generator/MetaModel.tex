\subsection{The meta-model} \label{sec_artifact_meta_model}

The meta-model is the model, of a model that is used to define a software system in the
context of this research. It contains the essential characteristics that are needed in
order to generate software applications and components. It contains the structures,
entities and relationships within the Generated Artifact. Having a meta-model ensures a
standardized way to describe and define the software components and their interactions,
enabling the design and analysis of modular, evolvable, and scalable software systems.

In this context, the meta-model serves as a blueprint for the software system, describing
its components, such as entities, fields, relationships, and expanders, along with their
attributes and constraints. The meta-model is used to generate the Generated Artifact
(\ref{sec_generated_artifact}). Moreover, it is potentially reusable for other types of
applications, which should lead to the same characteristics of stability and evolvability as
the one used for this research. 

The following sections describe all the entities that are part of the meta-model. These
entities are the basis of the \gls{erd} depicted in Appendix \ref{appendix:erd}. The ERD
is the official model representing the meta-model.

\subsubsection{The App entity}

The App entity represents an application and is regarded as the highest entry point for the
Generator artifact. The App Entity and subsequent entities contain all the information
needed to generate the Generated Artifact described in section \ref{sec_generated_artifact}. 

\begin{table}[H]
    \small
    \begin{tabular}{ p{0.21\linewidth} p{0.21\linewidth} p{0.49\linewidth} }
        \hline
        \textbf{Name} & \textbf{DataType} & \textbf{Description} \\
        \hline
        Id & Guid & Unique identifier of the application \\
        Name & string & Name of the application \\
        FullName & string & Full name of the application \\
        Expanders & List of Expanders & The Expanders that will be used during the
        generation process. \\
        Entities & List of Entities & The Entities that are applicable for the Generated artifact. \\
        ConnectionStrings & List of ConnectionStrings & The ConnectionString to the
        database that is used by the Generator Artifact. \\
        \hline
    \end{tabular}
    \caption{The fields of the App entity}
    \label{table:app_entity}
\end{table}

\subsubsection{The Component entity}

The Component entity represents a software component that can be part of an application.
Based on this entity the Generator Artifact can make design time on where to place certain
elements  

\begin{table}[H]
\small
\begin{tabular}{ p{0.20\linewidth} p{0.20\linewidth} p{0.50\linewidth} }
\hline
\textbf{Name} & \textbf{DataType} & \textbf{Description} \\
\hline
Id & Guid & Unique identifier of the component \\
Name & string & Name of the component \\
Description & string & Description of the component \\
Packages & List of Package & The Packages that should be applied to the component. \\
Expander & Expander & Navigation property to the Expander entity. \\
\hline
\end{tabular}
\caption{The fields of the Component entity}
\label{table:component_entity}
\end{table}

\subsubsection{The ConnectionString entity}

The ConnectionString entity represents a ConnectionString used by an application to
connect to a database or other external system.

\begin{table}[H]
\small
\begin{tabular}{ p{0.20\linewidth} p{0.20\linewidth} p{0.50\linewidth} }
\hline
\textbf{Name} & \textbf{DataType} & \textbf{Description} \\
\hline
Id & Guid & Unique identifier of the ConnectionString \\
Name & string & Name of the ConnectionString \\
Definition & string & Definition of the ConnectionString \\
App & App & Navigation property to the App entity \\
\hline
\end{tabular}
\caption{The fields of the ConnectionString entity}
\label{table:connectionstring_entity}
\end{table}

\subsubsection{The Entity entity}

The Entity entity represents an entity in the application's data model. 

\begin{table}[H]
\small
\begin{tabular}{ p{0.20\linewidth} p{0.24\linewidth} p{0.46\linewidth} }
\hline
\textbf{Name} & \textbf{DataType} & \textbf{Description} \\
\hline
Id & Guid & Unique identifier of the entity \\
Name & string & Name of the entity \\
Callsite & string & The source code location where the entity is defined. In the case of a C\#
artifact, this is to determine the name of the namespace.\\
Type & string & Type of the entity \\
Modifier & string & Modifier of the entity (e.g. public, private) \\
Behaviour & string & The behavior of the entity (e.g. abstract, virtual) \\
App & App & Navigation property to the App entity. \\
Fields & List of Fields & The Fields property represents a collection of the fields that
make up the entity. \\
ReferencedIn & List of Fields & Represents a navigation property that points to a Field
that uses the current entity as a return type. \\
Relations & List of Relationships & List of relationships involving this entity \\
IsForeignEntityOf & List of Relationships & List of relationships where this entity is the foreign entity \\
\hline
\end{tabular}
\caption{The fields of the Entity entity}
\label{table:entity_entity}
\end{table}

\subsubsection{The Expander entity}

The Expander entity represents an expander, which is responsible for generating code for
an application. The Generator Artifact attempts to execute all expanders that are related
to the selected App.

\begin{table}[H]
\small
\begin{tabular}{ p{0.20\linewidth} p{0.22\linewidth} p{0.48\linewidth} }
\hline
\textbf{Name} & \textbf{DataType} & \textbf{Description} \\
\hline
Id & Guid & Unique identifier of the expander \\
Name & string & Name of the expander \\
TemplateFolder & string & relative path to the templates that are used by the expander. \\
Order & int & The order in which the expander is executed \\
Apps & List of Apps & List of applications associated with the expander. \\
Components & List of Components & List of components associated with the expander \\
\hline
\end{tabular}
\caption{The fields of the Expander entity}
\label{table:expander_entity}
\end{table}

\subsubsection{The Field entity}

The Field entity represents a field or property of an entity in an application's data
model. Each field has a unique ID, name, and other properties such as its return type,
modifiers, and behavior. It can be associated with an entity and can have relationships
with other entities. The IsKey and IsIndex properties indicate whether the field is part
of the primary key or an index of the entity, respectively.

\begin{table}[H]
\small
\begin{tabular}{ p{0.23\linewidth} p{0.23\linewidth} p{0.44\linewidth} }
\hline
\textbf{Name} & \textbf{DataType} & \textbf{Description} \\
\hline
Id & Guid & Unique identifier of the field \\
Name & string & Name of the field \\
ReturnType & string & Return type of the field \\
IsCollection & bool & Whether the field is a collection or not \\
Modifier & string & Modifier of the field (e.g. public, private) \\
GetModifier & string & Modifier of the get accessor for the field \\
SetModifier & string & Modifier of the set accessor for the field \\
Behaviour & string & The behavior of the field (e.g. abstract, virtual) \\
Order & int & The order of the field within its entity \\
Size & int? & The size of the field \\
Required & bool & Whether the field is required or not \\
Reference & Entity & The entity that this field refers to\\
Entity & Entity & A navigation property to the parent entity \\
IsKey & bool & Indicates whether the field is part of the primary key \\
IsIndex & bool & Indicates whether the field is part of an index \\
RelationshipKeys & List of Relationships & List of relationships where this field is the key \\
IsForeignEntityKeyOf & List of Relationships & List of relationships where this field is the foreign key \\
\hline
\end{tabular}
\caption{The fields of the Field entity}
\label{table:field_entity}
\end{table}

\subsubsection{The Package entity}

The Package entity represents a software package that can be used by a component. This
could either be a Nuget package in the case of .NET projects, or for example npm packages
for web projects.

\begin{table}[H]
\small
\begin{tabular}{ p{0.20\linewidth} p{0.20\linewidth} p{0.50\linewidth} }
\hline
\textbf{Name} & \textbf{DataType} & \textbf{Description} \\
\hline
Id & Guid & Unique identifier of the package \\
Name & string & Name of the package \\
Version & string & Version of the package used \\
Component & Component & Component associated with the package \\
\hline
\end{tabular}
\caption{The fields of the Package entity}
\label{table:package_entity}
\end{table}

\subsubsection{The Relationship entity}

The Relationship entity represents a relationship between two entities in the App's data
model. The Relationship entity has proper cardinality support. Relationships are
bidirectional and can be navigated from either entity.

\begin{table}[H]
\small
\begin{tabular}{ p{0.24\linewidth} p{0.12\linewidth} p{0.55\linewidth} }
\hline
\textbf{Name} & \textbf{DataType} & \textbf{Description} \\
\hline
Id & Guid & Unique identifier of the relationship \\
Key & Field & The key field of the relationship \\
Entity & Entity & Navigation property to the parent Entity \\
Cardinality & string & The cardinality of the relationship \\
WithForeignEntityKey & Field & The foreign key field of the relationship, pointing to a
Field entity. \\
WithForeignEntity & Entity & The entity associated with the foreign key field \\
WithCardinality & string & The cardinality of the relationship with the foreign entity \\
Required & bool & indicates whether the relationship is required or not \\
\hline
\end{tabular}
\caption{The fields of the Relationship entity}
\label{table:relationship_entity}
\end{table}