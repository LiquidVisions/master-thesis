\subsection{Combinatorial effects versus Ripple effects} \label{subsec_ripple_effect}

\glspl{ce} and \glspl{ripple} are characteristics often used in \gls{ns} Theory. Both are
closely related to the challenges of maintaining evolvable systems but are certainly not
interchangeable. A \gls{ripple} is reffered to indicate the total amount of changes that
are required to adhere to new requirements. 

\textcite[271-272]{mannaert_normalized_2016} describe Combinatorial Effects as:
\enquote{instabilities in the evolution of an information system where the number of
additional software primitives is not only dependent on the amount of additional
functional requirements, but also on the set of existing software primitives —the size of
the system— at that point in time. These dependencies on the size of the system are caused
by the dimensions of variability due to the various versions. They are combinatorial
effects between the additional functional requirements, and the various existing versions
of software primitives.}