\section{The Convergence of Priniples} \label{sec_converging_principles}

In this section, we will apply a systematic cross-referencing approach to assess the level
of convergence between each of the \gls{solid} principles of \gls{ca} and the \gls{ns}
theories. Along with a brief explanation, the level of convergence is denoted as followed:

\begin{table}[H]
    \begin{tabular}{ l l p{0.57\linewidth}} Fully converges & \conv & This indicates
        a high degree of alignment between the respective \gls{solid} principle and
        \gls{ns} theorem. The application of either principle or theorem results in a
        similar impact on the software design. \\
        Supports convergence & \partconv & In this case, the \gls{solid} principle
        assists in implementing the \gls{ns} theorem through specific design choices.
        However, it is essential to note that applying the principle does
        not inherently ensure adherence to the corresponding theorem. \\
        No convergence & \noconv & This denotation signifies a lack of alignment between
        the \gls{solid} principle and the corresponding theorem. \\
    \end{tabular}
\end{table}

\subsection{Single Responsibility Principle} \label{srp}

\evaluatePrincipleTable{\gls{srp}}{table_srp_convergence}{ \addEvalRow{\gls{soc} &
    \fullConvergence & The main goal of both \gls{srp} and \gls{soc} is to promote and
    encourage modularity, low coupling, and high cohesion. While the definition has some
    differences, the two principles are practically interchangeable. Many examples in the
    Artifacts show a strong convergence between \gls{srp} and \gls{soc}. To name one, an
    Expander should be able to can perform multiple Tasks to complete the full
    instantiation of the model. Each of those Tasks can be implemented separately from
    each other. Figure \ref{fig_handlers} illustrates some of the Tasks implemented in the
    Clean Architecture Expander artifact. The Code Listing
    \ref{list_expandentitieshandlerinteractor} is an example of one implementation of such
    a task \citecode{koks_expandentitieshandlerinteractor_2023}.}
    
    \addEvalRow{\gls{dvt} & \npartialConvergence & Although using SRP does not implicitly
    guarantee \gls{dvt}, it supports \gls{dvt} by directing certain design choices.For
    example, \gls{ca} and \gls{ns} assign specific \gls{dto} objects to support specific
    use cases (Interactors or Tasks) or transfer (parts of) Data between architectural
    layers. \gls{ca} specifically assigned \glspl{dto} and guidelines on where and when to
    use them. These are also applied in the artifact of this study as ResponseModels,
    RequestModels, and ViewModels
    \parencites{koks_requestmodels_2023,koks_viewmodels_2023}. The separation of data
    structures specific to Use Cases minimizes the impact of data structure changes by
    preferring stamp coupling over data coupling. However, \gls{srp} is not a guaranteed
    measure for \gls{dvt}.}
    
    \addEvalRow{\gls{avt} & \npartialConvergence & While \gls{srp} emphasizes limiting the
    responsibility of each module, it does not explicitly require handling specific
    versions of use cases. Nevertheless, adhering to \gls{srp} can still indirectly
    contribute to achieving \gls{avt}. One way to achieve this is by separating versions
    of Actions into separate contracts, objects, or methods, enabling Action Version
    transparency to some degree. Although not yet available in the artifact, Code Listing
    \ref{list_versioning} shows that API versioning is a common standard practice fully
    supported by the open API specification and the .net core framework
    \parencites{github_aspnet-api-versioningprogramcs_2023, oas_versioning_2023}.
    Manifestations in the artifact can be located in the Logger (Code Listing
    \ref{list_logging}), amongst others \parencite{koks_logger_2023}.}
    
    \addEvalRow{\gls{sos} &\noConvergence & Following \gls{srp} might lead to separate modules
    that manage their state, indirectly contributing to \gls{sos}. However, the convergence
    is very weak, and no manifestations are found in the artifacts.}

}

\begin{figure}[H]
    \centering
    \includegraphics[width=0.6\textwidth]{figures/expander_handlers.pdf}
    \caption[handlers]{Each of the handlers handles an isolated part of the expanding process.}
    \label{fig_handlers}
\end{figure}
\subsection{Open/Closed Principle}

\evaluatePrincipleTable{\gls{ocp}}{table_ocp_alignment}{ 
    
\addEvalRow{\gls{soc} & \fullAlignment & The \gls{ocp} is strongly aligned with the \gls{soc}
    principle of \gls{ns}. \gls{ocp} states that software architectures should be open for
    extension but closed for modification. When applying \gls{ocp} correctly, the
    architecture supports new requirements built as an extension, affecting as few
    existing implementations as possible. Conversely, adhering to \gls{soc} does not
    guarantee the adherence of \gls{ocp}, as \gls{soc} focuses on modularization and
    encapsulation rather than the extensibility of functionality. The same example with
    the Tasks provided in sub-section \ref{srp} is also an excellent manifestation of this
    principle.} 
    
\addEvalRow{\gls{dvt} & \partialAlignment & While \gls{dvt} aims to handle changes in data
    structures without impacting the system, \gls{ocp} focuses on the extensibility of
    software architectures. Although \gls{ocp} does not explicitly address data version
    transparency, \gls{ocp} promotes architectures that are more adaptable to changes in
    data structures. Separating \glspl{dto} specific for a use case is a great example and
    manifestation in the Artifact
    \parencite{koks_requestmodels_2023,koks_viewmodels_2023}. However, there is no
    complete alignment because the use of \gls{ocp} does not guarantee \gls{dvt}.}
    
\addEvalRow{\gls{avt} & \fullAlignment &The \gls{ocp} is strongly aligned with the \gls{avt}
    principle of \gls{ns}, as both principles emphasize the importance of allowing changes
    or extensions to actions without affecting existing implementations. \gls{ocp} is also
    closely related to \gls{srp}. Besides \gls{srp}, \gls{ocp} have the most
    manifestations in the Artifact, some of which are already mentioned in previous
    examples. } 
    
\addEvalRow{\gls{sos} & \noAlignment & The \gls{ocp} indirectly relates to the \gls{sos}
    principle. The alignment of both principles is weak, and no
    manifestations are found in the artifacts. } 
}
\subsection{Liskov Substitution Principle}

\evaluatePrincipleTable{\gls{lsp}}{table_lsp_alignment}{ 
    
\addEvalRow{\gls{soc} & \fullAlignment & \gls{lsp} states that objects of a derived class should be
able to replace objects of the base class without affecting the program negatively.
Replacing objects can only be achieved by separating them, aligning the principles
inherritly. A good example is the implementation of the
\citecode{koks_itemplateinteractor_2023} where the template engine Scriban
\parencite{github_scriban_2023} is used to generate code instantiations as a result of the
Expanding the Model \parencite{koks_scribantemplateinteractor_2023}. We could easily
replace the Scriban template engine for an other engine with only impacting the Dependency
Injection Register.}
    
\addEvalRow{\gls{dvt} & \noAlignment & The alignment between \gls{lsp} and \gls{dvt} is weak,
and no manifestations are found in the artifacts.}
    
\addEvalRow{\gls{avt} & \partialAlignment & The \gls{lsp} supports the \gls{avt} principle.
Both principles emphasize the importance of allowing the extensibility of the software
system. By adhering to \gls{lsp}, the architecture allows for class hierarchies that can
be easily extended to accommodate new (versions of) actions, which can contribute to
achieving \gls{avt}. However, adhering to \gls{lsp} alone may not guarantee full adherence
with \gls{avt}. Considder \citecode{koks_icreategateway_2023} in Code Listing
\ref{list_ICreateGatewayExamples}. The artifact contains multiple implementations of this
interface. Each implementation could be considered a different version applied to the
interface.} 
    
\addEvalRow{\gls{sos} & \noAlignment & The \gls{lsp} does not relate to the \gls{sos}
principle. The alignment of both principles is weak, and no manifestations are found in
the artifacts.} }

\subsubsection{The Interface Segregation Principle} \label{subsubsec_isp}

The \gls{isp} suggests that software components should have narrow, specific interfaces
rather than broad, general-purpose ones. In addition, the \gls{isp} states that consumer
code should not be allowed to depend on methods it does not use. In other words,
interfaces should be designed to be as small and focused as possible, containing only the
methods relevant to the consumer code using them. This allows for the consumer code to use
only the needed methods without being forced to implement or depend on unnecessary methods
\parencite[104]{robert_c_martin_clean_2018}. 
\subsubsection{The Dependency Inversion Principle} \label{subsubsec_dip} 

The \gls{dip} prescribes that high-level modules should not depend on low-level modules,
and both should depend on abstractions. The principle emphasizes that the architecture
should be designed so that the flow of control between the different objects, layers, and
components is always from higher-level implementations to lower-level details. In other
words, high-level implementations, like business rules, should not be concerned about
low-level implementations, such as how the data is stored or presented to the end user.
Additionally, high-level and low-level implementations should only depend on abstractions
or interfaces that define a contract for how they should interact with each other
\parencite[91]{robert_c_martin_clean_2018}. 

This approach allows for great flexibility and a modular architecture. Modifications in
the low-level implementations will not affect the high-level implementations as long as
they still adhere to the contract defined by the abstractions and interfaces. Similarly,
changes to the high-level modules will not affect the low-level modules as long as they
still fulfill the contract. This reduces coupling and ensures the evolvability system over
time, as changes can be made to specific modules without affecting the rest of the system.