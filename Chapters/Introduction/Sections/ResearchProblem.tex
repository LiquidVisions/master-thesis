\section{Research Problem: The plethora of proposed solutions}
\label{sec:research_problem}

There is a wide variety of proposed solutions available for the challenges that occur in
modular and evolvable software architecture. There are plentiful experiences documented
throughout professional and personal blog posts on the internet. Many of those experiences
have mixed outcomes, some are opinionated and results are sometimes based on improper
interpretations of the proposed solutions.

Deciding on the best fit for one of the solutions is a recurring and often difficult task
for software architects. A popular and widely accepted solution from software engineering
literature is Clean Architecture. There is a wide supporting community and many corporate
solutions move toward architectures that have similarities with the Clean Architecture
approach. 

An architecture that derives from science and empirical evidence is Normalized Systems
\parencite*{mannaert_normalized_2009,mannaert_normalized_2016}. Deciding between the two
approaches can be a challenging task on which there is very little documentation and
research. Could it even be the case that combining the two approaches can be an approach
that leads to a highly modular and evolvable software artifact? Let's start with a small
introduction of both approaches.

\subsection{On Normalized Systems Theorems} \label{subsec:intro_to_ns}

The Normalized Systems theorems are a scientific approach to creating software systems
based on the laws for software evolvability. These theorems have resulted in a documented
track record of achieving software stability, in a scientific environment. Effectively, it
prevents the accumulation of combinatorial effects on anticipated change drivers. This
prevents the positive feedback loop and prevents the degradation of the software artifact.
Preventing positive feedback loops has a positive effect on the evolvability of software
artifacts.\parencite[]{mannaert_normalized_2009}. 

\citeauthor[]{mannaert_normalized_2009} have formulated the theorem of Normalized Systems
as prescriptive structures (elements) that lead to a modular architecture with low
coupling and high cohesion. The resulting software architecture will be designed to cope
with future change \parencites[]{mannaert_normalized_2009}.

\subsection{On Clean Architecture} \label{subsec:into_to_ca}

Clean Architecture is the accumulation of more than half a century of coding, designing,
and architecting software systems by \citeauthor*[]{martin_clean_2018}. He published his
experience in his book \citetitle*[]{martin_clean_2018} in \citeyear[]{martin_clean_2018}.
In this book, he states that creating a software artifact does not require that much skill
and knowledge. However, creating stable and evolvable software artifacts is a skill that
requires a lot of knowledge, skill, dedication, and time.

The book's goal is to have a software architecture that minimizes the human resources
required to build and maintain the information system. Just like Normalized Systems, it
has a prescribed design of software classes that will lead to a modular architecture with
low coupling and high cohesion \parencite{martin_clean_2018}.