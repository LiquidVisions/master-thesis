\subsection{The RequestModel Element} \label{converging_requestmodel_element}

\evaluateElementTable{RequestModel}{tab_convergence_requestmodel}{
    \addEvalRow{ Data & \partialAlignment & Both elements represent data objects that are part of the
        ontology or data schema of the application, and typically include attributes and
        relationship information. While they may contain a specific set of information as
        input for a Task or use case, both elements can also contain a full set of
        attributes and relationships. However, unlike the Data entity in \gls{ns}, which
        may include only a subset of information necessary for a specific Task or use
        case, it may also include the full set of information required for Tasks other
        purposes.}

    \addEvalRow{Task & \noAlignment & The RequestModel is not convergent with the Task element of
        \gls{ns}. However, the Tasks element might operate on RequestModels as input
        parameters to perform business logic.}

    \addEvalRow{Flow & \noAlignment & The RequestModel and Flow are not convergent, as the Flow element
        represents the control between Tasks in \gls{ns}, while the RequestModel in \gls{ca}
        represents (parts of) domain objects.}

    \addEvalRow{Connector & \noAlignment & The RequestModel element and Connector element are not
        convergent, as the Connector element in \gls{ns} is involved in the communication
        between components, whilst the RequestModel in \gls{ca} represents (parts of)
        domain objects.}
    
    \addEvalRow{Trigger & \noAlignment & The RequestModel element and Trigger element are not convergent,
        as the Trigger element in \gls{ns} is about event-based execution of Tasks, while
        the RequestModel in \gls{ca} represents (parts of) domain objects.}
}