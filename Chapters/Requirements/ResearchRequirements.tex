\section{Research requirements} \label{sec:research_requirements} 

Three types of changes are observed when new (functional) requirements are
introduced. These types of changes are applicable for all objects in the artifacts
representing the data, action, flow, connector and trigger elements from the Normalized
Systems Theorems.
\begin{itemize}
    \item Requirements that lead to the addition of objects;
    \item Requirements that lead to the modification of existing implementations;
    \item Requirements that lead to the removal of existing implementations;
\end{itemize} 

The main requirement for both the expander- and expanded artifact can be generalized. They
should be free of combinatorial effects on anticipated change drivers. In order to prove
or disprove the hypothesis described in chapter \ref{hypothesis} the following main
requirement can be specified:

\subsubsection*{Requirement 1}
Requirements that lead to the addition of objects can be systematically added one by one
without deteriorating the stability of the system.

\subsubsection*{Requirement 2}
Requirements that lead to the modification of existing implementations can be
systematically modified one by one without deteriorating the stability of the system.

\subsubsection*{Requirement 3}
Requirements that lead to the removal of existing implementation can be systematically
modified one by one without deteriorating the stability of the system.

In the case of the requirements described above, the systematic additions or modifications
without deteriorating the stability of the system can only be achieved when taking into
account the order of the dependencies of the system.