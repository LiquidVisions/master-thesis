\section{The Theoretical background of Normalized Systems} \label{ns_theory}
\begin{itemize}
    \item Inleiding
\end{itemize}


\subsection{Design Theorems of Normalized Systems} \label{subsec:ns_desing_theorems}

The theorems of Normalized Systems described in the following paragraphs are all
considered to be scientifically proven theorems \parencite{mannaert_normalized_2016}. The
authors of the book claim that the theories, when implemented correctly increase
the stability, and therefore the evolvability of software systems. The theorems are proven
mathematically followed by simple examples of describing them by \textit{reductio ad absurdum} 
\footnote{\url{https://en.wikipedia.org/wiki/Reductio_ad_absurdum}}.

\subsubsection{Separation of Concerns}
Since the early years, 'Separation of Concerns (SoC)' has been one of the most fundamental
software development principles. According to Wikipedia
\footnote{\url{https://en.wikipedia.org/wiki/Separation_of_concerns}} the principle has
first been mentioned by \citeauthor{dijkstra_selected_1982} as the crucial principle to
design modular software architecture \parencite[]{dijkstra_selected_1982}. The concept
itself was introduced by \citeauthor{broy_criteria_1972} in his book
\citetitle*{broy_criteria_1972}.

According to \citeauthor{dijkstra_selected_1982}, this principle promotes the idea that a
program should be divided into distinct sections, each addressing a separate concern or
aspect of the problem. This allows for a more organized and maintainable program, as
changes to one concern do not affect the others. Dijkstra also emphasized that the
separation of concerns should be applied at the level of individual modules or functions,
rather than at the level of the entire program.

Normalized Systems have a more specific definition of this principle. 
In the book of
\citeauthor{mannaert_normalized_2016} it is described as \textit{A processing function can
only contain a single task to achieve stability}. There are a couple of 
manifestation examples described like the application of an integration service bus. The
manifestation of external workflows is another example. One example of an external
workflow in the prototype of this research project is the
\citetitle*{koks_seedingboundary_2023}. \todo{add hyperlink}This class is responsible for the provision of
the initial data to the model, based on the elements of the meta-model itself. It
'seeds' the data by orchestrating individual entity seeders to execute in the correct
order. Another example is the segregation of the \citetitle*{koks_entity_2023} where each
CRUD operation is separated in its own interface.\todo{add hyperlink}\todo{Question:
should I add examples of the prototype in the chapter of theoretical background?}

\subsubsection{Data version transparancy}
\begin{itemize}
    \item Toelichting op data version transparency
\end{itemize}

\subsubsection{Action version transparancy}
\begin{itemize}
    \item Toelichting op action version transparency
\end{itemize}

\subsubsection{Separation of state}
\begin{itemize}
    \item toelichting op Separation of state.
\end{itemize}

\subsection{Construction elements}
\begin{itemize}
    \item Data elements
    \item task elements
    \item connector elements
    \item flow elements
    \item etc\dots
    \item In het hoofdstuk evaluation wordt
    besproken in hoeverre de CA construction elements convergeren met die van NS
\end{itemize}
