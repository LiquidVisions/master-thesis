\subsection{Coupling} \label{subsec_on_coupling}

Coupling is an essential concept in software engineering that pertains to the degree of
interdependence among software modules and components. The level of coupling between
modules denotes the strength of their relationship, whereby a high level of coupling
implies a significant degree of interdependence. Conversely, low coupling signifies a
weaker relationship between modules, where modifications in one module are less likely to
impact others. Although not always possible, the level of coupling between the various
modules of the system should be kept to a bare minimum.

The negative impact of excessive coupling on software systems is considerable. High
coupling can render software systems challenging to maintain, modify, or evolve. It can make
it considerably more challenging to find the root cause of potential bugs. Additionally, it
causes fragility in the system, where slight modifications in one module can trigger
cascading failures throughout the entire system. Therefore, it is crucial for software
engineers to minimize coupling between modules while maintaining a cohesive design. By
developing systems with low coupling, software engineers can construct more maintainable,
scalable, and adaptable systems that are easier to evolve.

Coupling in software engineering can take several forms, including content, common,
control, stamp, and data coupling. Content coupling occurs when one module accesses or
modifies the internal data or logic of another module, leading to high interdependence and
difficulty in isolating errors. Common coupling occurs when several modules access and use
the same global data, increasing their interdependence and reducing modularity. Control
coupling occurs when one module controls the execution flow of another module, making it
challenging to modify or reuse the controlled module. Stamp coupling arises when two modules
share a common data structure, leading to tight coupling and high interdependence.
Finally, data coupling exists when two modules share data, which can lead to coupling
between them.

One attempt to lower coupling in the expanded artifact is to prefer stamp coupling over
data coupling through the API interface. This is done by making use of RequestModels and
ViewModels, instead of the actual data element (see example in Listings \ref{SnipModelExamples}).
Depending on the use case, only the required data is passed down to the view, or in the
case of a command accepted as an input parameter.

\lstinputlisting[
    caption={The ViewModel \parencite{koks_componentviewmodel_2023} and RequestModel
    \parencite{koks_deletecomponentcommand_2023} of the Entity 'Component'
    \parencite{koks_component_2023}},
    label={SnipModelExamples}]
    {Snippets/ModelExamples.cs}