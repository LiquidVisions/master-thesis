% Chapter #1

\chapter{Problem statement}

A lot of research has been done to define the Normalized Systems Theorems and
design patterns for software architectures. Normalized systems theory has been
proposed as an approach to develop agile and evolvable software systems. Following
the research companies has arisen with a primary focus on software evolvability,
software agility and coping with change as part of the software architecture. 

Companies that apply the Normalized Systems Theory research into their products
are primarily using Java EE as a programming language. The company NSX for example
has implemented their generation tools, modelling suite (Prime Radiant) and
expander using this programming language. Java EE is still a very popular
programming language for enterprise-, and IT organisations. Many software
solutions are created and maintained using this programming language. The
Normalized Systems Theorem is not only applicable to Java EE. The principles and
design patterns that derive from the Normalized Systems Theorem are in fact
applicable for any object-oriented programming languages. Another example of a
popular programming language in enterprise-, and IT organisations is . 

There ishowever no documented research, or proof of experiences on software
projects using Normalized Systems Theory with the aspects of integration,
expansion and rejuvenation. asd asd a sd asd a sd a da d asdasdasdasd asd asd asd
a sd as da sd a d   asd asdasdasdasdasdasd asdasdasda dasdasdasdasdasdasd asd as d
asd as d asd as d asdasdasd asd asdasdasd adasdasd asdasdasd asda sdasdasd asdasda
adasdas dasda sda sdas da as das das da sd asdasdasd asd asda sda da sd a sd as
dasdasda sdas das d
